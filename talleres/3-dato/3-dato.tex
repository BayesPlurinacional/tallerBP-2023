\newif\ifen
\newif\ifes
\newif\iffr
\newcommand{\fr}[1]{\iffr#1 \fi}
\newcommand{\En}[1]{\ifen#1\fi}
\newcommand{\Es}[1]{\ifes#1\fi}
\estrue
\documentclass[shownotes,aspectratio=169]{beamer}

\usepackage{siunitx}
\input{../../auxiliar/tex/diapo_encabezado.tex}
\input{../../auxiliar/tex/tikzlibrarybayesnet.code.tex}
 \mode<presentation>
 {
 %   \usetheme{Madrid}      % or try Darmstadt, Madrid, Warsaw, ...
 %   \usecolortheme{default} % or try albatross, beaver, crane, ...
 %   \usefonttheme{serif}  % or try serif, structurebold, ...
  \usetheme{Antibes}
  \setbeamertemplate{navigation symbols}{}
 }
\estrue
\usepackage{todonotes}
\setbeameroption{show notes}
%
\newcommand{\gray}{\color{black!55}}
\usepackage{ulem} % sout
\usepackage{mdframed}
\usepackage{listings}
\lstset{
  aboveskip=3mm,
  belowskip=3mm,
  showstringspaces=true,
  columns=flexible,
  basicstyle={\ttfamily},
  breaklines=true,
  breakatwhitespace=true,
  tabsize=4,
  showlines=true
}


\begin{document}

\color{black!85}
\large
%
% \begin{frame}[plain,noframenumbering]
%
%
% \begin{textblock}{160}(0,0)
% \includegraphics[width=1\textwidth]{../../auxiliar/static/deforestacion}
% \end{textblock}
%
% \begin{textblock}{80}(18,9)
% \textcolor{black!15}{\fontsize{44}{55}\selectfont Verdades}
% \end{textblock}
%
% \begin{textblock}{47}(85,70)
% \centering \textcolor{black!15}{{\fontsize{52}{65}\selectfont Empíricas}}
% \end{textblock}
%
% \begin{textblock}{80}(100,28)
% \LARGE  \textcolor{black!15}{\rotatebox[origin=tr]{-3}{\scalebox{9}{\scalebox{1}[-1]{$p$}}}}
% \end{textblock}
%
% \begin{textblock}{80}(66,43)
% \LARGE  \textcolor{black!15}{\scalebox{6}{$=$}}
% \end{textblock}
%
% \begin{textblock}{80}(36,29)
% \LARGE  \textcolor{black!15}{\scalebox{9}{$p$}}
% \end{textblock}
%
% %
% %
% % \begin{textblock}{160}(01,81)
% % \footnotesize \textcolor{black!5}{\textbf{\small Seminario ``Acuerdos intersubjetivos''\\
% % Comunidad Bayesiana Plurinacional} \\}
% % \end{textblock}
%
% \end{frame}

%%%%%%%%%%%%%%%%%%%%%%%%%%%%%%%%%%%%%%%%%

\begin{frame}[plain,noframenumbering]
\begin{textblock}{160}(0,43)
\includegraphics[width=1\textwidth]{../../auxiliar/static/modelosGraficos}
\end{textblock}


\begin{textblock}{160}(4,4)
\LARGE \textcolor{black!85}{\fontsize{22}{0}\selectfont \textbf{Modelos gráficos e inferencia}}
\end{textblock}
% \begin{textblock}{160}(4,12)
% \LARGE \textcolor{black!85}{\fontsize{22}{0}\selectfont \textbf{algoritmos de inferencia}}
% \end{textblock}


\begin{textblock}{55}[0,0](72,23)
\begin{turn}{0}
\parbox{10cm}{\sloppy\setlength\parfillskip{0pt}
\textcolor{black!85}{Unidad 1} \\
\small\textcolor{black!85}{Acuerdos intersubjetivos en contextos de incertidumbre.} \\
\small\textcolor{black!85}{Especificación gráfica de modelos causales. Evaluación} \\
\small\textcolor{black!85}{de modelos causales. La emergencia del sobreajuste y el} \\
\small\textcolor{black!85}{balance natural de las reglas de la probabilidad.} \\
}
\end{turn}
\end{textblock}

\end{frame}



\begin{frame}[plain]
\begin{textblock}{160}(0,4)
\centering \LARGE Teoría de la información\\
\large \only<2->{El problema de la comunicación con la realidad}
\end{textblock}

\vspace{1.2cm}

\only<1>{
\begin{textblock}{160}(0,40) \centering
\Large \textbf{El problema de la comunicación con la realidad}
\end{textblock}
}

% \only<2->{
% \begin{textblock}{140}(10,48)
% Quisiéramos que lo recibido sea igual a lo enviado. \\[.05cm]
%
% \normalsize
%
% \hspace{0.5cm} $\bullet$ Solución física (me acerco para escuchar mejor)
%
% \only<3->{\hspace{0.5cm} $\bullet$ Solución inferencial (interpretar la señal con ruido)}
% \end{textblock}
% }


\only<2->{
\begin{textblock}{160}(0,28) \centering \Large

Quisiéramos que el mensaje recibido sea igual al enviado
\end{textblock}
}

\only<3->{
\begin{textblock}{140}(10,44) \centering
\Large Soluciones \\[0.05cm]

\large

Física: acercarme para escuchar mejor \\
\only<3>{Inferencia: interpretar la señal con ruido}\only<4->{\textbf{Inferencia: interpretar la señal con ruido} \\[1.1cm]}
\only<5->{\Large ¿Cuál es el dato de la realidad?}
\end{textblock}
}

\end{frame}

\begin{frame}[plain]
\begin{textblock}{160}(0,4) \centering
\LARGE Base empírica \\
\large \only<-9>{¿Cuál es el dato de la realidad?}\only<10>{\textbf{El dato se construye}}
\end{textblock}

\only<2>{
\begin{textblock}{160}(0,28) \centering
 \Large ¿Cuál es el conjunto de elementos del mundo \\ que sirven de evidencia indubitable (dato)?
\end{textblock}
}

\only<3-9>{
\begin{textblock}{130}(15,21)
 $\bullet$ BE Filosófica: $\emptyset$ \\
 \only<4->{$\bullet$ BE Epistemológica: Precios\\ }
 \only<5>{$\bullet$ Teoría: Inflación}\only<6->{$\bullet$ BE Metodológica$_1$: Inflación \\}
\only<7>{$\bullet$ Teoría: Producto Bruto Interno}\only<8->{$\bullet$ BE Metodológica$_2$: Producto Bruto Interno\\}
\only<9>{$\bullet$ Teoría: Política pública}
\end{textblock}
}

\only<10>{
\begin{textblock}{130}(15,25) \Large \centering
Depende del conjunto de supuestos \\
que una comunidad no pone en duda!
\end{textblock}
}

\only<4>{
\begin{textblock}{160}(0,48) \centering
\includegraphics[width=0.62\textwidth, page={1}]{figuras/baseEmpirica.pdf}
\end{textblock}}
\only<5>{
\begin{textblock}{160}(0,48) \centering
\includegraphics[width=0.62\textwidth, page={2}]{figuras/baseEmpirica.pdf}
\end{textblock}}
\only<6>{
\begin{textblock}{160}(0,48) \centering
\includegraphics[width=0.62\textwidth, page={3}]{figuras/baseEmpirica.pdf}
\end{textblock}}
\only<7>{
\begin{textblock}{160}(0,48) \centering
\includegraphics[width=0.62\textwidth, page={4}]{figuras/baseEmpirica.pdf}
\end{textblock}}
\only<8>{
\begin{textblock}{160}(0,48) \centering
\includegraphics[width=0.62\textwidth, page={5}]{figuras/baseEmpirica.pdf}
\end{textblock}}
\only<9->{
\begin{textblock}{160}(0,48) \centering
\includegraphics[width=0.62\textwidth, page={6}]{figuras/baseEmpirica.pdf}
\end{textblock}}


\end{frame}


\begin{frame}[plain]
 \begin{textblock}{160}(0,4)
 \centering \LARGE
 Los datos como funciones proposicionales
\end{textblock}
\vspace{0.75cm}

\end{frame}

\begin{frame}[plain]
 \begin{textblock}{160}(0,4)
 \centering \LARGE
 Los datos como funciones proposicionales
\end{textblock}
\vspace{0.75cm}

\begin{textblock}{160}(0,20)
\begin{equation*}
 f(x) = y
\end{equation*}
\end{textblock}

\begin{textblock}{160}(43,33)
\begin{itemize}
 \item[$x$]
    \textbf{\en{Unit of analysis}\es{Unidad de análisis}} (UA)
 \item[$f$]
   \en{\textbf{Variable} of the unit of analysis}
   \es{\textbf{Variable} de la unidad de análisis} (V)
 \item[$y$]
   \en{\textbf{Value} of the variable}
   \es{\textbf{Resultado} o valor de la variable} (R)
\end{itemize}
\end{textblock}


\only<2>{
\begin{textblock}{160}(0,65) \centering
 \textit{Altura}(Gustavo) = $1.78$m
\end{textblock}
}

\only<3-4>{
\begin{textblock}{160}(0,65) \centering
 \textit{Ideología}(Partido Comunista) = Comunista \\
 \only<4>{P(\textit{Ideología}(Partido Comunista) = Comunista) = 0.8}
\end{textblock}
}

\only<5>{
\begin{textblock}{160}(0,65) \centering
 \textit{Habilidad}(Maradona) $>$ \textit{Habilidad}(Messi)
\end{textblock}
}

\only<6>{
\begin{textblock}{140}(10,60)
\begin{framed} \centering
   \en{The meaning of data is implicit in their \textbf{operationalization}}
   \es{El significado preciso de la función depende de la \textbf{operacionalización}}
   \end{framed}
\end{textblock}
}
\end{frame}


\begin{frame}[plain]
\begin{textblock}{160}(0,4)
 \centering \LARGE La estructura invariante del dato empírico
\end{textblock}
\vspace{0.75cm}


\begin{textblock}{160}(0,16)
\centering
\tikz{

\node[det] (fuente) {};
\node[const, above=of fuente] (n_fuente) {$\hfrac{\text{Fuente de}}{\text{información}}$};
\node[const, right=of fuente, yshift=-0.35cm, xshift=0.05cm] (mensaje_enviado) {$\hfrac{\text{Mensaje}}{\text{enviado}}$};


\node[det, right=of fuente, xshift=0.6cm ] (transmisor) {};
\node[const, above=of transmisor] (n_transmisor) {\scriptsize Transmisor};
\node[const, right=of transmisor, yshift=-0.35cm, xshift=0.05cm] (senal_enviada) {$\hfrac{\text{Señal}}{\text{enviada}}$};


\node[det, right=of transmisor, xshift=0.7cm , minimum size=6pt] (perceptor) {};
\node[const, above=of perceptor] (n_perceptor) {\scriptsize Perceptor};
\node[const, right=of perceptor, yshift=-0.35cm, xshift=0.05cm] (senal_recibida) {$\hfrac{\text{Señal}}{\text{recibida}}$};

\node[det, below=of perceptor, yshift=0.2cm] (ruido) {};
\node[const, below=of ruido] (n_ruido) {\scriptsize Rudio};

\node[det, right=of perceptor, xshift=0.7cm] (receptor) {};
\node[const, above=of receptor] (n_receptor) {\scriptsize Receptor};
\node[const, right=of receptor, yshift=-0.35cm, xshift=0.05cm] (mensaje_recibido) {$\hfrac{\text{Mensaje}}{\text{recibido}}$};


\node[det, right=of receptor, xshift=0.6cm ] (destino) {};
\node[const, above=of destino] (n_destino) {$\hfrac{\text{Información}}{\text{en destino}}$};

\edge {fuente} {transmisor};
\edge {transmisor, ruido} {perceptor};
\edge {perceptor} {receptor};
\edge {receptor} {destino};
}

\end{textblock}

\begin{textblock}{140}(10,50)
\normalsize
$\bullet$ Fuente: el estado \textit{real} de variable \hfill \textit{habilidad}(Messi)\\

$\bullet$ Mensaje enviado: la dinámica del mundo \hfill Realidad Causal \\

$\bullet$ Transmisor: dimensiones perceptibles de la variable \hfill Ganar/Perder \\

$\bullet$ Perceptor: el cuerpo o instrumento de medición \hfill \textit{Scraper} de \texttt{fifa.com}  \\

$\bullet$ Receptor: el indicador, lo efectivamente registrado \hfill True/False \\

$\bullet$ Mensaje recibido: inferencia de la variable \hfill Modelo Causal \\

$\bullet$ Destino: la estimación  \hfill $P$(\textit{habilidad}(Messi) = y$|$I, M) \\


\end{textblock}


\end{frame}



\begin{frame}[plain]
\begin{textblock}{160}(0,4)
\centering \LARGE Siglo 20: Frecuentismo\\
\large \sout{Evaluación} Selección de hipótesis
\end{textblock}

\only<1>{
\begin{textblock}{140}(10,18)
\centering
\includegraphics[width=0.6\textwidth]{../../auxiliar/static/Elo1980.jpg}
\end{textblock}
}

 \only<2->{
 \begin{textblock}{140}(3,24)
 \normalsize
\tikz{
    \node[det, fill=black!10] (r) {$r$} ;
    \node[const, right=of r] (dr) {\normalsize $ P(r|d) = \mathbb{I}(r = d > 0)$};

    \node[latent, above=of r, yshift=-0.45cm] (d) {$d$} ; %
    \node[const, right=of d] (dd) {\normalsize $ P(d|p_a,p_b) = \delta(d = p_i-p_j)$};

    \node[latent, above=of d, xshift=-0.8cm, yshift=-0.45cm] (p1) {$p_a$} ; %
    \node[latent, above=of d, xshift=0.8cm, yshift=-0.45cm] (p2) {$p_b$} ; %


    \node[accion, above=of p1,yshift=0.3cm] (s1) {} ; %
    \node[const, right=of s1] (ds1) {$s_a$};
    \node[accion, above=of p2,yshift=0.3cm] (s2) {} ; %
    \node[const, right=of s2] (ds2) {$s_b$};

    \node[const, right=of p2] (dp2) {\normalsize $P(p_i|s_i) = \N(s_i,\beta^2)$};

    \node[const, above=of dr] (r_name) {\small Resultado};
    \node[const, above=of dd] (d_name) {\small Diferencia};
    \node[const, above=of dp2] (p_name) {\small Desempeño};

    \node[const, above=of ds2, yshift=0.1cm] (s_name) {\small Habilidad};

    \edge {d} {r};
    \edge {p1,p2} {d};
    \edge {s1} {p1};
    \edge {s2} {p2};

}
\end{textblock}
}



\only<3>{
\begin{textblock}{90}(75,24)
\includegraphics[width=0.75\textwidth]{figuras/probaOfWin_2D}
\end{textblock}
}

\only<4>{
\begin{textblock}{90}(80,38)
\includegraphics[width=0.8\textwidth]{figuras/probaOfWin}
\end{textblock}
}

\only<3-5>{
\begin{textblock}{90}(65,12)
\normalsize
\begin{flalign*}
 \hspace{-2cm}  P(d\,|\,p_a,p_b) = \iint &\delta(d=p_a-p_b)  \N(p_a|s_a,\beta^2)\N(p_b|s_b,\beta^2) \, dp_a \, dp_b \\
 \only<4->{&  = \N(\,d\,|\, s_a-s_b, \, 2\beta^2\,)}
 &&
 \end{flalign*}

\end{textblock}
}

\only<5->{
\begin{textblock}{90}(80,32)
\normalsize
\begin{flalign*}
& P(d>0 \,|\,s_a,s_b, \text{M}) =   1 - \Phi\left( \frac{s_a - s_b}{\sqrt{2}\beta} \right) &&
\end{flalign*}
\end{textblock}
}


\only<6->{
\begin{textblock}{60}(80,54)
\normalsize

\begin{equation*}
s_a^{\text{new}} = s_a^{\text{old}} + \Delta
\end{equation*}

\begin{equation*}
\hspace{0.3cm} \onslide<7->{\Delta = \underbrace{(1 - P(d > 0 \,|\,s_a,\, s_b))}_{\text{\footnotesize Sorpresa}}}
\onslide<8>{
\underbrace{(-1)^{(1-r)} }_{\text{\footnotesize Signo}}
}
\end{equation*}
\end{textblock}
}

\end{frame}




\begin{frame}[plain]
\begin{textblock}{160}(0,4)
\centering \LARGE Siglo 21: El retorno del enfoque bayesiano\\
\large Aplicación estricta de las reglas de la probabilidad
\end{textblock}


Teorema de Bayes.

\includegraphics[width=0.65\textwidth]{figuras/posterior_win.pdf}



\end{frame}

\begin{frame}[plain]


\begin{textblock}{140}(10,28)
\Large Ejercicio 1.1 \\[0.4cm]

\large La persona que da la pista a veces se confunde y señala la caja donde está el regalo o la caja que fue elegida previamente por la persona. \\[0.2cm]

\only<2>{
\large Descubrir el verdadero efecto causal que el regalo y la elección tienen sobre la pista en base a los datos ofrecidos en la práctica. \\[0.1cm]}

\end{textblock}


\end{frame}

\begin{frame}[plain,noframenumbering]
\centering \vspace{0.5cm}
\includegraphics[width=1\textwidth]{../../auxiliar/static/BP.png}
\end{frame}





%
% \begin{frame}[plain]
% \begin{textblock}{96}(0,6.5)\centering
% {\transparent{0.9}\includegraphics[width=0.8\textwidth]{../../auxiliar/static/inti.png}}
% \end{textblock}
%
% \begin{textblock}{160}(96,5.5)
% \includegraphics[width=0.35\textwidth]{../../auxiliar/static/pachacuteckoricancha}
% \end{textblock}
% \end{frame}





\end{document}



