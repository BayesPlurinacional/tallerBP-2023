\newif\ifen
\newif\ifes
\newif\iffr
\newcommand{\fr}[1]{\iffr#1 \fi}
\newcommand{\En}[1]{\ifen#1\fi}
\newcommand{\Es}[1]{\ifes#1\fi}
\estrue
\documentclass[shownotes,aspectratio=169]{beamer}

\usepackage{siunitx}

\usepackage{ragged2e} %\justifying
\usepackage{paracol}
\usepackage[utf8]{inputenc} %Para acentos en UTF8 (Prueba: á é í ó ú Á É Í Ó Ú ñ Ñ)
\usepackage{url}
%\usepackage{mathtools}
\usepackage{graphicx}
\usepackage{caption}
\usepackage{float} % para que los gr\'aficos se queden en su lugar con [H]
\usepackage[fleqn]{mathtools} % \coloneqq, flalign
\usepackage{subcaption}
\usepackage{wrapfig}
\usepackage{soul,color} %\st{Hellow world}
\usepackage{xcolor} %\st{Hellow world}
\usepackage[fleqn]{amsmath} %para escribir funci\'on partida
\usepackage{blkarray}
\usepackage{hyperref} % para inlcuir links dentro del texto
\usepackage{tabu} 
\usepackage{comment}
\usepackage{amsfonts} % mathbb{N} -> conjunto de los n\'umeros naturales  
\usepackage{enumerate}
\usepackage{listings}
\usepackage[shortlabels]{enumitem} %  shortlabels option to have compatibility with the enumerate-like scheme for label
\usepackage{framed}
\usepackage{mdframed}
\usepackage{multicol}
\usepackage{transparent} % \transparent{1.0}
\usepackage{bm} 
\usepackage[makeroom]{cancel} % \cancel{} \bcancel{} etc
\usepackage[absolute,overlay]{textpos} %no funciona
\setlength{\TPHorizModule}{1mm} %128mm  mitad: 64 
\setlength{\TPVertModule}{1mm}	%96mm  mitad 48

\newif\ifen
\newif\ifes
\newcommand{\en}[1]{\ifen#1\fi}
\newcommand{\es}[1]{\ifes#1\fi}
\estrue


\usepackage{todonotes}
\setbeameroption{show notes}
\usepackage{rotating}
\usepackage{transparent}


\newcommand{\E}{\en{S}\es{E}}
\newcommand{\A}{\en{E}\es{A}}
\newcommand{\Ee}{\en{s}\es{e}}
\newcommand{\Aa}{\en{e}\es{a}}

\hypersetup{
    colorlinks=true,
    linkcolor={red!50!black},
    citecolor={blue!35!black},
    urlcolor={blue!35!black}
}

\newcommand\hfrac[2]{\genfrac{}{}{0pt}{}{#1}{#2}} %\frac{}{} sin la linea del medio

\newcommand{\indep}{\perp \!\!\! \perp}
\newcommand{\N}{\mathcal{N}}
\newcommand{\vm}[1]{\mathbf{#1}}

\newtheorem{midef}{Definition}
\newtheorem{miteo}{Theorem}
\newtheorem{mipropo}{Proposition}

\usefonttheme[onlymath]{serif}


\usepackage{tikz} % Para graficar, por ejemplo bayes networks
%\usetikzlibrary{bayesnet} % Para que ande se necesita copiar el archivo  tikzlibrarybayesnet.code.tex en la misma carpeta

%%%%%%%%%%%%%%%%%%%%%%%%%%%%%%%%%5
%
% Incompatibles con textpos
%
%\usepackage{todonotes}
%\usepackage{tikz} % Para graficar, por ejemplo bayes networks
%
%%%%%%%%%%%%%%%%%%%%%%%%%%%%%%%%%%



\usepackage[absolute,overlay]{textpos} %no funciona
\setlength{\TPHorizModule}{1mm} %128mm  mitad: 64 
\setlength{\TPVertModule}{1mm}	%96mm  mitad 48
% 
% 
\captionsetup[figure]{labelformat=empty}

% 
% http://latexcolor.com/
\definecolor{lightseagreen}{rgb}{0.13, 0.7, 0.6.5}
\definecolor{greenblue}{rgb}{0.1, 0.55, 0.5}
\definecolor{redgreen}{rgb}{0.6, 0.4, 0.}
\definecolor{greenred}{rgb}{0.4, 0.7, 0.}
\definecolor{redblue}{rgb}{0.4, 0., .4}
\definecolor{tangelo}{rgb}{0.98, 0.3, 0.0}
\definecolor{git}{rgb}{0.94, 0.309, 0.2}
% 
\setbeamercolor{structure}{fg=greenblue}


%http://latexcolor.com/
\definecolor{azul}{rgb}{0.36, 0.54, 0.66}
\definecolor{rojo}{rgb}{0.7, 0.2, 0.116}
\definecolor{rojopiso}{rgb}{0.8, 0.25, 0.17}
\definecolor{verdeingles}{rgb}{0.12, 0.5, 0.17}
\definecolor{ubuntu}{rgb}{0.44, 0.16, 0.39}
\definecolor{debian}{rgb}{0.84, 0.04, 0.33}
\definecolor{dkgreen}{rgb}{0,0.6,0}
\definecolor{gray}{rgb}{0.5,0.5,0.5}
\definecolor{mauve}{rgb}{0.58,0,0.82}




\newcommand\Wider[2][3em]{%
\makebox[\linewidth][c]{%
  \begin{minipage}{\dimexpr\textwidth+#1\relax}
  \raggedright#2
  \end{minipage}%
  }%
}

\newenvironment{ejercicio}[1]{
% \setbeamercolor{block title}{bg=tangelo, fg=white}
\begin{exampleblock}{#1}
}{
\end{exampleblock}
}

\newenvironment{resumen}[1]{
\setbeamercolor{block title}{bg=git, fg=white}
\begin{block}{#1}
}{
\end{block}
}

\newenvironment{comando}{
\setbeamercolor{block body}{bg=git, fg=white}
\begin{block}{}
\begin{center}
\LARGE
\begin{texttt}
}{
\end{texttt}
\end{center}
\end{block}
}



% tikzlibrary.code.tex
%
% Copyright 2010-2011 by Laura Dietz
% Copyright 2012 by Jaakko Luttinen
%
% This file may be distributed and/or modified
%
% 1. under the LaTeX Project Public License and/or
% 2. under the GNU General Public License.
%
% See the files LICENSE_LPPL and LICENSE_GPL for more details.

% Load other libraries

%\newcommand{\vast}{\bBigg@{2.5}}
% newcommand{\Vast}{\bBigg@{14.5}}
% \usepackage{helvet}
% \renewcommand{\familydefault}{\sfdefault}

\usetikzlibrary{shapes}
\usetikzlibrary{fit}
\usetikzlibrary{chains}
\usetikzlibrary{arrows}

% Latent node
\tikzstyle{latent} = [circle,fill=white,draw=black,inner sep=1pt,
minimum size=20pt, font=\fontsize{10}{10}\selectfont, node distance=1]
% Observed node
\tikzstyle{obs} = [latent,fill=gray!25]
% Invisible node
\tikzstyle{invisible} = [latent,minimum size=0pt,color=white, opacity=0, node distance=0]
% Constant node
\tikzstyle{const} = [rectangle, inner sep=0pt, node distance=0.1]
%state
\tikzstyle{estado} = [latent,minimum size=8pt,node distance=0.4]
%action
\tikzstyle{accion} =[latent,circle,minimum size=5pt,fill=black,node distance=0.4]


% Factor node
\tikzstyle{factor} = [rectangle, fill=black,minimum size=10pt, draw=black, inner
sep=0pt, node distance=1]
% Deterministic node
\tikzstyle{det} = [latent, rectangle]

% Plate node
\tikzstyle{plate} = [draw, rectangle, rounded corners, fit=#1]
% Invisible wrapper node
\tikzstyle{wrap} = [inner sep=0pt, fit=#1]
% Gate
\tikzstyle{gate} = [draw, rectangle, dashed, fit=#1]

% Caption node
\tikzstyle{caption} = [font=\footnotesize, node distance=0] %
\tikzstyle{plate caption} = [caption, node distance=0, inner sep=0pt,
below left=5pt and 0pt of #1.south east] %
\tikzstyle{factor caption} = [caption] %
\tikzstyle{every label} += [caption] %

\tikzset{>={triangle 45}}

%\pgfdeclarelayer{b}
%\pgfdeclarelayer{f}
%\pgfsetlayers{b,main,f}

% \factoredge [options] {inputs} {factors} {outputs}
\newcommand{\factoredge}[4][]{ %
  % Connect all nodes #2 to all nodes #4 via all factors #3.
  \foreach \f in {#3} { %
    \foreach \x in {#2} { %
      \path (\x) edge[-,#1] (\f) ; %
      %\draw[-,#1] (\x) edge[-] (\f) ; %
    } ;
    \foreach \y in {#4} { %
      \path (\f) edge[->,#1] (\y) ; %
      %\draw[->,#1] (\f) -- (\y) ; %
    } ;
  } ;
}

% \edge [options] {inputs} {outputs}
\newcommand{\edge}[3][]{ %
  % Connect all nodes #2 to all nodes #3.
  \foreach \x in {#2} { %
    \foreach \y in {#3} { %
      \path (\x) edge [->,#1] (\y) ;%
      %\draw[->,#1] (\x) -- (\y) ;%
    } ;
  } ;
}

% \factor [options] {name} {caption} {inputs} {outputs}
\newcommand{\factor}[5][]{ %
  % Draw the factor node. Use alias to allow empty names.
  \node[factor, label={[name=#2-caption]#3}, name=#2, #1,
  alias=#2-alias] {} ; %
  % Connect all inputs to outputs via this factor
  \factoredge {#4} {#2-alias} {#5} ; %
}

% \plate [options] {name} {fitlist} {caption}
\newcommand{\plate}[4][]{ %
  \node[wrap=#3] (#2-wrap) {}; %
  \node[plate caption=#2-wrap] (#2-caption) {#4}; %
  \node[plate=(#2-wrap)(#2-caption), #1] (#2) {}; %
}

% \gate [options] {name} {fitlist} {inputs}
\newcommand{\gate}[4][]{ %
  \node[gate=#3, name=#2, #1, alias=#2-alias] {}; %
  \foreach \x in {#4} { %
    \draw [-*,thick] (\x) -- (#2-alias); %
  } ;%
}

% \vgate {name} {fitlist-left} {caption-left} {fitlist-right}
% {caption-right} {inputs}
\newcommand{\vgate}[6]{ %
  % Wrap the left and right parts
  \node[wrap=#2] (#1-left) {}; %
  \node[wrap=#4] (#1-right) {}; %
  % Draw the gate
  \node[gate=(#1-left)(#1-right)] (#1) {}; %
  % Add captions
  \node[caption, below left=of #1.north ] (#1-left-caption)
  {#3}; %
  \node[caption, below right=of #1.north ] (#1-right-caption)
  {#5}; %
  % Draw middle separation
  \draw [-, dashed] (#1.north) -- (#1.south); %
  % Draw inputs
  \foreach \x in {#6} { %
    \draw [-*,thick] (\x) -- (#1); %
  } ;%
}

% \hgate {name} {fitlist-top} {caption-top} {fitlist-bottom}
% {caption-bottom} {inputs}
\newcommand{\hgate}[6]{ %
  % Wrap the left and right parts
  \node[wrap=#2] (#1-top) {}; %
  \node[wrap=#4] (#1-bottom) {}; %
  % Draw the gate
  \node[gate=(#1-top)(#1-bottom)] (#1) {}; %
  % Add captions
  \node[caption, above right=of #1.west ] (#1-top-caption)
  {#3}; %
  \node[caption, below right=of #1.west ] (#1-bottom-caption)
  {#5}; %
  % Draw middle separation
  \draw [-, dashed] (#1.west) -- (#1.east); %
  % Draw inputs
  \foreach \x in {#6} { %
    \draw [-*,thick] (\x) -- (#1); %
  } ;%
}


 \mode<presentation>
 {
 %   \usetheme{Madrid}      % or try Darmstadt, Madrid, Warsaw, ...
 %   \usecolortheme{default} % or try albatross, beaver, crane, ...
 %   \usefonttheme{serif}  % or try serif, structurebold, ...
  \usetheme{Antibes}
  \setbeamertemplate{navigation symbols}{}
 }
\estrue
\usepackage{todonotes}
\setbeameroption{show notes}
%
\newcommand{\gray}{\color{black!55}}
\usepackage{ulem} % sout
\usepackage{mdframed}
\usepackage{listings}
\lstset{
  aboveskip=3mm,
  belowskip=3mm,
  showstringspaces=true,
  columns=flexible,
  basicstyle={\footnotesize\ttfamily},
  breaklines=true,
  breakatwhitespace=true,
  tabsize=4,
  showlines=true,
}

\begin{document}

\color{black!85}
\large
%
% \begin{frame}[plain,noframenumbering]
%
%
% \begin{textblock}{160}(0,0)
% \includegraphics[width=1\textwidth]{../../auxiliar/static/deforestacion}
% \end{textblock}
%
% \begin{textblock}{80}(18,9)
% \textcolor{black!15}{\fontsize{44}{55}\selectfont Verdades}
% \end{textblock}
%
% \begin{textblock}{47}(85,70)
% \centering \textcolor{black!15}{{\fontsize{52}{65}\selectfont Empíricas}}
% \end{textblock}
%
% \begin{textblock}{80}(100,28)
% \LARGE  \textcolor{black!15}{\rotatebox[origin=tr]{-3}{\scalebox{9}{\scalebox{1}[-1]{$p$}}}}
% \end{textblock}
%
% \begin{textblock}{80}(66,43)
% \LARGE  \textcolor{black!15}{\scalebox{6}{$=$}}
% \end{textblock}
%
% \begin{textblock}{80}(36,29)
% \LARGE  \textcolor{black!15}{\scalebox{9}{$p$}}
% \end{textblock}
%
% %
% %
% % \begin{textblock}{160}(01,81)
% % \footnotesize \textcolor{black!5}{\textbf{\small Seminario ``Acuerdos intersubjetivos''\\
% % Comunidad Bayesiana Plurinacional} \\}
% % \end{textblock}
%
% \end{frame}

%%%%%%%%%%%%%%%%%%%%%%%%%%%%%%%%%%%%%%%%%


\begin{frame}[plain,noframenumbering]
\begin{textblock}{160}(0,-4.3) \centering
\includegraphics[width=1\textwidth]{../../auxiliar/static/antartic}
\end{textblock}

\begin{textblock}{160}(0,0) \centering
\tikz{
\node[det, fill=black,draw=black] (k) {\textcolor{black}{--------------------------------------------------------------------------------------------------------------------------------------}} ;
}
\end{textblock}

\begin{textblock}{160}(5,0)
\tikz{
\node[det, fill=black,draw=black,text width=0.01cm] (k) {\textcolor{black}{--------------------------------------------------------------------------------------------------------------------------------------}} ;
}
\end{textblock}


\begin{textblock}{160}(0,4) \centering
\LARGE \hspace{1cm} \textcolor{black!20}{\fontsize{22}{0}\selectfont \textbf{Modelos de historia \\ \hspace{1cm} completa}}
\end{textblock}


\begin{textblock}{55}[0,1](8,70)
\begin{turn}{90}
\parbox{6cm}{\footnotesize
\textcolor{black!10}{Millones de km$^2$ de hielo Antártico}}
\end{turn}
\end{textblock}


\begin{textblock}{160}(20,63)
\textcolor{black!10}{Unidad 4 \\ \small
Redes bayesianas de historia completa. \\
El problema de usar el posterior como prior del siguiente evento\\
El algoritmo de inferencia por loopy belief propagation. \\
Consideraciones de inferencia causal en series temporales. \\
}
\end{textblock}

\end{frame}


\begin{frame}[plain]
\begin{textblock}{160}(0,4)
\centering \LARGE Series de tiempo \\
\large Creencias adaptativas
\end{textblock}


\only<1-4>{
\begin{textblock}{160}(0,26) \centering

\Large La función de costo epistémica

\large
\begin{equation*}
\underbrace{P(\text{Hipótesis},\text{\En{Data}\Es{Datos}})}_{\hfrac{\text{\footnotesize\En{Initial belief compatible}\Es{Creencia compatible }}}{\text{\footnotesize \En{with the data}\Es{con los datos}}}} = \underbrace{P(\text{Hipótesis})}_{\hfrac{\text{\footnotesize\En{Initial intersubjective}\Es{Acuerdo intersubjetivo}}}{\text{\footnotesize\En{agreement}\Es{inicial}}}} \underbrace{P(\text{dato}_1 |\text{Hipótesis})}_{\text{\footnotesize Predic\En{tion}\Es{ción} 1}} \, \underbrace{P(\text{dato}_2 | \text{dato}_1 , \text{Hipótesis})}_{\text{\footnotesize Predic\En{tion}\Es{ción} 2}} \dots
\end{equation*}

\vspace{0.8cm}


\only<2>{
\Large Un único 0 en la secuencia de predicciones

hace falsa la hipótesis para siempre.\\
}\only<3-4>{
Ejemplo

\Large Esa persona no está apta para realizar esa tarea.

\only<4>{\textbf{¿Para siempre?!}}
}

\end{textblock}
}
\only<5>{ \centering
\begin{textblock}{160}(0,-93)
\includegraphics[width=0.9\textwidth]{../../auxiliar/static/lifeHistory.jpeg}
\end{textblock}
}


\end{frame}

\begin{comment}

\begin{frame}[plain]
\begin{textblock}{160}(0,4)
\centering \LARGE Estimación de habilidad \\
\large en la industria del videojuego
\end{textblock}


 \only<1->{
 \begin{textblock}{140}(3,24)
 \normalsize
\tikz{
    \node[det, fill=black!10] (r) {$r$} ;
    \node[const, right=of r] (dr) {\normalsize $ P(r|d) = \mathbb{I}(r = d > 0)$};

    \node[latent, above=of r, yshift=-0.45cm] (d) {$d$} ; %
    \node[const, right=of d] (dd) {\normalsize $ p(d|p_a,p_b) = \delta(d = p_a-p_b)$};

    \node[latent, above=of d, xshift=-0.8cm, yshift=-0.45cm] (p1) {$p_a$} ; %
    \node[latent, above=of d, xshift=0.8cm, yshift=-0.45cm] (p2) {$p_b$} ; %


    \node[latent, above=of p1, yshift=-0.35cm] (s1) {$s_a$} ; %
    \node[latent, above=of p2, yshift=-0.35cm] (s2) {$s_b$} ; %
    \node[const, right=of s2] (ds2) {$p(s_i) = \N(s_i|\mu_i,\sigma_i^2)$};

    \node[const, right=of p2] (dp2) {\normalsize $p(p_i|s_i) = \N(p_i|s_i,\beta^2)$};

    \node[const, above=of dr] (r_name) {\small Resultado};
    \node[const, above=of dd] (d_name) {\small Diferencia};
    \node[const, above=of dp2] (p_name) {\small Desempeño};

    \node[const, above=of ds2, yshift=0.1cm] (s_name) {\small Habilidad};

    \edge {d} {r};
    \edge {p1,p2} {d};
    \edge {s1} {p1};
    \edge {s2} {p2};

}
\end{textblock}
}


\only<2->{
\begin{textblock}{100}(60,22) \centering
Prior \ $\rightarrow$ \ Posterior

\only<2>{ \includegraphics[width=0.85\textwidth,page=4]{../3-dato/figuras/posterior_win.pdf}}
\end{textblock}
}


\only<3->{
\begin{textblock}{100}(60,30) \centering
¿Cómo estimamos una habilidad en el tiempo?

\Large \vspace{0.8cm}

\only<4->{
¿Si usamos el último posterior como

prior del siguiente evento?

\vspace{0.6cm} \large

Posterior$_t$ $\rightarrow$ Prior$_{t+1}$
}

\end{textblock}
}
\end{frame}



\begin{frame}[plain]
\begin{textblock}{120}(40,4)
 \centering \LARGE Algoritmo suma-producto \\
 \large Las reglas de la probabilidad por pasaje de mensajes
\end{textblock}


\only<1->{
 \begin{textblock}{140}(2,6)
 \normalsize
\tikz{
    \node[det, fill=black!10] (r) {$r$} ;
    \node[factor, above=of r, yshift=-0.4cm] (fr) {};
    \node[const, left=of fr] (dr) {\normalsize $\mathbb{I}(r = d > 0)$};

    \node[latent, above=of fr, yshift=-0.55cm] (d) {$d$} ; %
    \node[factor, above=of d, yshift=-0.4cm] (fd) {};

    \node[const, left=of fd] (dd) {\normalsize $\delta(d = p_a-p_b)$};

    \node[latent, above=of fd, xshift=-0.8cm, yshift=-0.55cm] (p1) {$p_a$} ; %
    \node[latent, above=of fd, xshift=0.8cm, yshift=-0.55cm] (p2) {$p_b$} ; %

    \node[factor, above=of p1, yshift=-0.4cm] (fp1) {};
    \node[factor, above=of p2, yshift=-0.4cm] (fp2) {};

    \node[latent, above=of fp1, yshift=-0.55cm] (s1) {$s_a$} ; %
    \node[latent, above=of fp2, yshift=-0.55cm] (s2) {$s_b$} ; %

    \node[factor, above=of s1, yshift=-0.4cm] (fs1) {};
    \node[factor, above=of s2, yshift=-0.4cm] (fs2) {};


    \node[const, left=of fs1] (ds2) {$\N(s_i|\mu_i,\sigma_i^2)$};

    \node[const, left=of fp1] (dp2) {\normalsize $\N(p_i|s_i,\beta^2)$};

    \node[const, left=of r] (r_name) {\small Resultado};
    \node[const, left=of d] (d_name) {\small Diferencia};
    \node[const, left=of p1] (p_name) {\small Desempeño};

    \node[const, left=of s1, yshift=0.1cm] (s_name) {\small Habilidad \ };

    \edge {d} {r};
    \edge[-] {p1,p2} {fd};
    \edge {fd} {d};
    \edge[-] {s1} {fp1};
    \edge[-] {s2} {fp2};
    \edge {fp1} {p1};
    \edge {fp2} {p2};
    \edge {fs1} {s1};
    \edge {fs2} {s2};

    \uncover<2->{
      \path[draw, ->, fill=black!50] (fs2) edge[bend left,draw=black!50] node[right,color=black!75] {} (s2);
      \path[draw, ->, fill=black!50] (fp2) edge[bend left,draw=black!50] node[right,color=black!75] {} (s2);
    }
}
\end{textblock}
}



\only<2->{
\begin{textblock}{90}(55,18)
\begin{flalign*}
p(s_b,r) = \underbrace{m_{f_{s_b} \rightarrow s_b}(s_b)}_{\text{Prior}} \cdot \underbrace{m_{f_{p_b} \rightarrow s_b}(s_b)}_{\text{Likelihood}}
\end{flalign*}
\end{textblock}
}

\only<3->{
\begin{textblock}{90}(55,40)
\begin{equation*}\label{eq:m_v_f}
\underbrace{m_{x \rightarrow f}(x)}_{\hfrac{\text{\footnotesize Mensaje de}}{\text{\footnotesize las \textbf{variables}}}} = \underbrace{\prod_{g \in v(x) \setminus \{f\}} m_{g \rightarrow x}(x)}_{\hfrac{\text{\footnotesize El producto de}}{\text{\footnotesize lo que recibe de atrás}}}
\end{equation*}
\end{textblock}
}

\only<4->{
\begin{textblock}{90}(55,62)
\begin{equation*}\label{eq:m_f_v}
 \underbrace{m_{f \rightarrow x}(x)}_{\hfrac{\text{\footnotesize Mensaje de}}{\text{\footnotesize los \textbf{factores}}}} =  \underbrace{\sum_{\bm{y}} \Big(f(\bm{y},x) \prod_{y \in v(f) \setminus \{x\} } m_{y \rightarrow f}(y) \Big)}_{\hfrac{\text{\footnotesize Lo que recibe de atrás por el factor,}}{\text{\footnotesize integrando todas las variables de atrás}}}
\end{equation*}
\end{textblock}
}


\end{frame}


\begin{frame}[plain]
\begin{textblock}{160}(20,4)
\centering \LARGE Estimación de habilidad \\
\large en la industria del videojuego
\end{textblock}

\only<1->{
 \begin{textblock}{140}(2,6)
 \normalsize
\tikz{
    \node[det, fill=black!10] (r) {$r$} ;
    \node[factor, above=of r, yshift=-0.4cm] (fr) {};
    \node[const, left=of fr] (dr) {\normalsize $\mathbb{I}(d > 0)$};

    \node[latent, above=of fr, yshift=-0.55cm] (d) {$d$} ; %
    \node[factor, above=of d, yshift=-0.4cm] (fd) {};

    \node[const, left=of fd] (dd) {\normalsize $\delta(d = p_a-p_b)$};

    \node[latent, above=of fd, xshift=-0.8cm, yshift=-0.55cm] (p1) {$p_a$} ; %
    \node[latent, above=of fd, xshift=0.8cm, yshift=-0.55cm] (p2) {$p_b$} ; %

    \node[factor, above=of p1, yshift=-0.4cm] (fp1) {};
    \node[factor, above=of p2, yshift=-0.4cm] (fp2) {};

    \node[latent, above=of fp1, yshift=-0.55cm] (s1) {$s_a$} ; %
    \only<1-20>{\node[latent, above=of fp2, yshift=-0.55cm] (s2) {$s_b$} ; }
    \only<21->{\node[latent, above=of fp2, yshift=-0.55cm] (s2) {$\bm{s_b}$} ; }


    \node[factor, above=of s1, yshift=-0.4cm] (fs1) {};
    \node[factor, above=of s2, yshift=-0.4cm] (fs2) {};


    \node[const, left=of fs1] (ds2) {$\N(s_i|\mu_i,\sigma_i^2)$};

    \node[const, left=of fp1] (dp2) {\normalsize $\N(p_i|s_i,\beta^2)$};

    \node[const, left=of r] (r_name) {\small Resultado};
    \node[const, left=of d] (d_name) {\small Diferencia};
    \node[const, left=of p1] (p_name) {\small Desempeño};

    \node[const, left=of s1, yshift=0.1cm] (s_name) {\small Habilidad \ };

%     \only<12->{
%       \node[factor, right=of s2] (fs2_t2) {};
%       \node[latent, right=of fs2_t2] (s2_t2) {$s_{b_{t+1}}$};
%     }

    \edge {d} {r};
    \edge[-] {p1,p2} {fd};
    \edge {fd} {d};
    \edge[-] {s1} {fp1};
    \edge[-] {s2} {fp2};
    \edge {fp1} {p1};
    \edge {fp2} {p2};
    \edge {fs1} {s1};
    \edge {fs2} {s2};

    \uncover<2-11,21->{
      \path[draw, ->, fill=black!50] (fs2) edge[bend left,draw=black!50] node[right,color=black!75] {\large\only<2>{$\N(s_b|\mu_b,\sigma^2)$}} (s2);
    }
    \uncover<3-11>{
      \path[draw, ->, fill=black!50] (s2) edge[bend left,draw=black!50] node[right,color=black!75] {\large\only<3>{$\N(s_b|\mu_b,\sigma^2)$}} (fp2);
    }
    \uncover<4-11>{
      \path[draw, ->, fill=black!50] (fp2) edge[bend left,draw=black!50] node[right,color=black!75] {\large$\only<4-5>{\int \N(p_b|s_b,\beta^2) \N(s_b|\mu_b,\sigma^2) \, ds_b }\only<5>{= p(p_b)}$} (p2);
    }
    \uncover<12->{
      \path[draw, ->, fill=black!50] (fs1) edge[bend left,draw=black!50] node[right,color=black!75] {} (s1);
      \path[draw, ->, fill=black!50] (s1) edge[bend left,draw=black!50] node[right,color=black!75] {} (fp1);
      \path[draw, ->, fill=black!50] (fp1) edge[bend left,draw=black!50] node[right,color=black!75] {} (p1);
      \path[draw, ->, fill=black!50] (p1) edge[bend left,draw=black!50] node[right,color=black!75] {} (fd);
    }
    \uncover<13->{
      \path[draw, ->, fill=black!50] (fr) edge[bend right,draw=black!50] node[right,color=black!75] {\large$\only<13>{\mathbb{I}(d > 0)}$} (d);
    }
    \uncover<14->{
      \path[draw, ->, fill=black!50] (d) edge[bend right,draw=black!50] node[right,color=black!75] {\large$\only<14>{\mathbb{I}(d > 0)}$} (fd);
    }
    \uncover<15->{
      \path[draw, ->, fill=black!50] (fd) edge[bend right,draw=black!50] node[right,color=black!75] {\large \ $\only<15>{\int_{p_a} \int_d \delta(d=p_a - p_b) \mathbb{I}(d > 0) \N(p_a|\mu_a,\beta^2 + \sigma_a^2) \, dp_a \, dd}$$\only<16>{\int_{p_a} \int_d \delta(\bm{d=p_a - p_b}) \mathbb{I}(\bm{d} > 0) \N(p_a|\mu_a,\beta^2 + \sigma_a^2) \, dp_a \, dd}\only<17-18>{\int_{p_a} \mathbb{I}(p_a > p_b) \N(p_a|\mu_a,\beta^2 + \sigma_a^2) \, dp_a \only<18>{= 1 - \Phi(p_b|\mu_a, \beta^2 + \sigma_a^2)}}$} (p2);
    }
    \uncover<19->{
      \path[draw, ->, fill=black!50] (p2) edge[bend right,draw=black!50] node[right,color=black!75] {\large \ $\only<19-20>{1 - \Phi(p_b|\mu_a, \beta^2 + \sigma_a^2)}$} (fp2);
    }
    \uncover<20->{
      \path[draw, ->, fill=black!50] (fp2) edge[bend right,draw=black!50] node[right,color=black!75] {\large \ $\only<20>{1 - \Phi(\bm{s_b}|\mu_a, \bm{2}\beta^2 + \sigma_a^2)}$} (s2);
    }

}
\end{textblock}
}



\only<6-14>{
\begin{textblock}{120}(65,18)
  \begin{flalign*}
   \uncover<-11>{p(p_b) &= \int \N(p_b|s_b,\beta^2) \N(s_b|\mu_b,\sigma_b^2) ds_b \\}
    \only<9>{& \overset{*}{=} \int \N(p_a|\mu_a,\beta^2 + \sigma_a^2) \, \N(s_a | \mu_*,\sigma_{*}^2)ds_a}
    \only<10>{& \overset{*}{=} \int \underbrace{\N(p_b|\mu_b,\beta^2 + \sigma_b^2)}_{\text{const.}} \underbrace{\N(s_a | \mu_*,\sigma_{*}^2)ds_b}_{1}}
    \only<11>{ \phantom{p(p_a)} & \overset{*}{=} \phantom{\int} \, \N(p_b|\mu_b,\beta^2 + \sigma_b^2) \phantom{\N(s_b | \mu_*,\sigma_{*}^2)ds_b}}
    \only<12->{ p(p_a) & \overset{*}{=} \phantom{\int} \, \N(p_b|\mu_a,\beta^2 + \sigma_a^2) \phantom{\N(s_a | \mu_*,\sigma_{*}^2)ds_a}}
    &&
  \end{flalign*}
\end{textblock}
}


\only<7-8>{
\begin{textblock}{120}(40,34)
  \begin{figure}[H]
     \centering
     \uncover<7->{
     \begin{subfigure}[b]{0.4\textwidth}
       \includegraphics[page=1,width=1\textwidth]{figuras/paso_1_multiplicacion_normales_image}
     \end{subfigure}}
     \uncover<8->{
     \begin{subfigure}[b]{0.4\textwidth}
       \includegraphics[page=1,width=\textwidth]{figuras/paso_1_multiplicacion_normales_p}
     \end{subfigure}}
  \end{figure}
\end{textblock}
}


\only<17-19>{
\begin{textblock}{120}(40,48)
  \begin{figure}[H]
     \centering
     \begin{subfigure}[b]{0.58\textwidth}
       \includegraphics[page=1,width=\textwidth]{figuras/m_d_ta.pdf}
     \end{subfigure}
  \end{figure}
\end{textblock}
}



\only<21-23>{
\begin{textblock}{100}(55,15)
\begin{flalign*}
p(s_b, r^*) =  \overbrace{\N(s_b|\mu_b, \sigma_b^2)}^{\text{Prior}} \overbrace{1 - \Phi(0|\mu_a-s_b, 2\beta^2 + \sigma_a^2)}^{\text{Predicción o verosimilitud}}
\end{flalign*}
\end{textblock}
}

\only<21>{
\begin{textblock}{100}(60,35)
\includegraphics[page=1,width=0.8\textwidth]{figuras/posterior_lose.pdf}
\end{textblock}
}
\only<22>{
\begin{textblock}{100}(60,35)
\includegraphics[page=2,width=0.8\textwidth]{figuras/posterior_lose.pdf}
\end{textblock}
}
\only<23-25>{
\begin{textblock}{100}(60,35)
\includegraphics[page=3,width=0.8\textwidth]{figuras/posterior_lose.pdf}
\end{textblock}
}


\only<24-25>{
\begin{textblock}{100}(60,18)
\begin{flalign*}
& \, \widehat{p}(s_b| r^*) = \only<24>{\underset{\mu, \sigma}{\text{ arg min }} \ \ \text{KL}(\, p(s_b| r^*) \, || \,  \N(s_b|\mu, \sigma^2) \, )}\only<25>{\N(s_b\,|\,\widehat{\mu},\, \widehat{\sigma}^2)}
&&
\end{flalign*}
\end{textblock}
}


\only<26->{
\begin{textblock}{100}(60,18)
\begin{flalign*}
& \underbrace{\widehat{p}(s_b| r^*)}_{\text{Posterior}_1} =\N(s_b\,|\,\widehat{\mu},\, \widehat{\sigma}^2)
&&
\end{flalign*}
\end{textblock}
}

\only<26->{
\begin{textblock}{110}(60,40)
\begin{flalign*}
 & \widehat{\text{Posterior}}_t \rightarrow \text{Prior}_{t+1} %= \text{Prior}_1 \prod_{i=1}^t \text{Likelihood}_i
 &&
\end{flalign*}
\end{textblock}
}


\only<27>{
\begin{textblock}{100}(60,58)
\begin{flalign*}
& \underbrace{p(s_b)}_{\text{Prior}_2} = \N(s_b\,|\,\widehat{\mu}, \,  \widehat{\sigma}^2 + \gamma^2)
&&
\end{flalign*}
\end{textblock}
}



%
%
% \begin{textblock}{90}(60,56) \centering
%  \normalsize
% \tikz{
%     \node[factor] (fs_t) {};
%     \node[const, above=of fs_t] (dfs_t) {$\N(s_t|\mu_t,\sigma_t^2)$};
%     \node[latent, right=of fs_t, minimum size=0.75cm] (s_t) {$s_{t}$};
%     \node[factor, right=of s_t] (fs_t2) {};
%     \node[latent, right=of fs_t2] (s_t2) {$s_{t+1}$};
%  }
% \end{textblock}

\end{frame}



\begin{frame}[plain]
\begin{textblock}{160}(0,4)
\centering \LARGE Enfoque Posterior $\rightarrow$ Prior \\
\only<3->{\large Problemas de un procedimiento ad-hoc}
\end{textblock}

\only<1-2>{
\begin{textblock}{150}(0,22)
\begin{figure}[ht!]
\centering
\tikz{
    \node[const] (a1) {\includegraphics[page=1,width=0.2\linewidth]{figuras/trueskillthroughtime}};
    \node[const, left=of a1] (na) {Jugador A:};
    \node[const,below=of a1] (b1) {\includegraphics[page=3,width=0.2\linewidth]{figuras/trueskillthroughtime}};
    \uncover<2>{
    \node[const,right=of a1] (a2) {\includegraphics[page=2,width=0.2\linewidth]{figuras/trueskillthroughtime}};
      \node[const,below=of a2] (b2) {\includegraphics[page=4,width=0.2\linewidth]{figuras/trueskillthroughtime}};
      \node[const, above=of a2] (g2) {\small \ \ Gana B};
    }
    \node[const, above=of a1] (g1) {\small Gana A \ \ };
    \node[const, left=of b1] (nb) {Jugador B:};

}
\end{figure}
\end{textblock}
}

\only<3->{
\begin{textblock}{140}(10,42)
\Large Problemas. \\[0.2cm] \normalsize

$\bullet$ Ofrece resultados que van en contra de nuestra intuición

\onslide<4->{$\bullet$ Mucha incertidumbre al inicio de las series temporales}

\onslide<5->{$\bullet$ No aprovecha la información disponible (entre ramas paralelas)}

\onslide<6->{$\bullet$ No garantiza comparabilidad entre estimaciones lejanas en el tiempo y el espacio.}
\end{textblock}
}


\only<7>{
\begin{textblock}{160}(0,24)  \centering \Large
Propaga la información en una sola

dirección, del pasado al futuro.
\end{textblock}
}

\end{frame}



\begin{frame}[plain]
\only<4>{
 \begin{textblock}{160}(0,4) \normalsize \centering
\includegraphics[width=0.55\textwidth]{../../auxiliar/static/ttt_history.png}
\end{textblock}
}
\begin{textblock}{160}(0,4)
\centering \LARGE Modelos de historia completa \\
\large \only<1-4>{Estado del arte en la industria del videojuego.}\only<5->{Smoothing}
\end{textblock}



\only<1-3>{
 \begin{textblock}{160}(0,16) \normalsize \centering
 \scalebox{0.9}{
\tikz{
    \node[det, fill=black!10] (r) {$r$} ;
    \node[factor, above=of r, yshift=-0.4cm] (fr) {};

    \node[latent, above=of fr, yshift=-0.55cm] (d) {$d$} ; %
    \node[factor, above=of d, yshift=-0.4cm] (fd) {};


    \node[latent, above=of fd, xshift=-0.8cm, yshift=-0.55cm] (p1) {$p_a$} ; %
    \node[latent, above=of fd, xshift=0.8cm, yshift=-0.55cm] (p2) {$p_b$} ; %

    \node[factor, above=of p1, yshift=-0.4cm] (fp1) {};
    \node[factor, above=of p2, yshift=-0.4cm] (fp2) {};

    \node[latent, above=of fp1, yshift=-0.55cm] (s1) {$s_{a_1}$} ; %
    \node[latent, above=of fp2, yshift=-0.55cm] (s2) {$s_{b_1}$} ;

    \node[factor, above=of s1, yshift=-0.4cm] (fs1) {};
    \node[factor, above=of s2, yshift=-0.4cm] (fs2) {};

%     \node[const, left=of fr] (dr) {\normalsize $\mathbb{I}(d > 0)$};
%     \node[const, left=of fd] (dd) {\normalsize $\delta(d = p_a-p_b)$};
%     \node[const, left=of fs1] (ds2) {$\N(s_i|\mu_i,\sigma_i^2)$};
%     \node[const, left=of fp1] (dp2) {\normalsize $\N(p_i|s_i,\beta^2)$};

%     \node[const, left=of r] (r_name) {\small Resultado};
%     \node[const, left=of d] (d_name) {\small Diferencia};
%     \node[const, left=of p1] (p_name) {\small Desempeño};
%     \node[const, left=of s1, yshift=0.1cm] (s_name) {\small Habilidad \ };

    \edge {d} {r};
    \edge[-] {p1,p2} {fd};
    \edge {fd} {d};
    \edge[-] {s1} {fp1};
    \edge[-] {s2} {fp2};
    \edge {fp1} {p1};
    \edge {fp2} {p2};
    \edge {fs1} {s1};
    \edge {fs2} {s2};

    \only<2->{
      \node[factor, right=of s2, xshift=-0.3cm] (fs2_t2) {};
      \node[latent, right=of fs2_t2, xshift=-0.15cm] (s2_t2) {$s_{b_2}$};
      \node[factor, left=of s1, xshift=0.3cm] (fs1_t2) {};
      \node[latent, left=of fs1_t2, xshift=0.15cm] (s1_t2) {$s_{a_2}$};

      \node[const, above=of fs2_t2, yshift=0.1cm, xshift=0.1cm] (ds2_t2) {$\N(s_{b_2}|s_{b_1}, \gamma^2)$};
      \node[const, above=of fs1_t2, yshift=0.1cm, xshift=-0.1cm] (ds1_t2) {$\N(s_{a_2}|s_{a_1}, \gamma^2)$};

      \edge[-] {s1} {fs1_t2};
      \edge[-] {s2} {fs2_t2};
      \edge {fs1_t2} {s1_t2};
      \edge {fs2_t2} {s2_t2};
    }
    \only<3->{
      \node[factor, below=of s1_t2, yshift=0.55cm] (fp1_t2) {} ;
      \node[latent, below=of fp1_t2, yshift=0.4cm] (p1_t2) {$p_{a_2}$} ;
      \node[factor, below=of p1_t2, xshift=-0.8cm, yshift=0.55cm] (fd_t2) {} ;
      \node[latent, below=of fd_t2, yshift=0.4cm] (d_t2) {$d$} ;
      \node[factor, below=of d_t2, yshift=0.55cm] (fr_t2) {};
      \node[det, below=of fr_t2, yshift=0.4cm, fill=black!10] (r_t2) {$r$} ;

      \node[latent, above=of fd_t2, xshift=-0.8cm, yshift=-0.55cm] (p3) {$p_{e_t}$} ;
      \node[factor, above=of p3, yshift=-0.4cm] (fp3) {} ;
      \node[latent, above=of fp3,yshift=-0.55cm] (s3) {$s_{e_t}$} ;
      \node[factor, left=of s3,xshift=0.4cm] (fs3) {} ;
      \node[const, left=of fs3] (ds3) {$\dots$} ;


      \edge[-] {s1_t2} {fp1_t2};
      \edge {fp1_t2} {p1_t2};
      \edge[-] {p1_t2, p3} {fd_t2};
      \edge {fd_t2} {d_t2};
      \edge[-] {d_t2} {fr_t2};
      \edge {fr_t2} {r_t2};
      \edge {fs3} {s3};
      \edge[-] {s3} {fp3};
      \edge {fp3} {p3};

      \node[factor, below=of s2_t2, yshift=0.55cm] (fp2_t2) {} ;
      \node[latent, below=of fp2_t2, yshift=0.4cm] (p2_t2) {$p_{b_2}$} ;

      \node[factor, below=of p2_t2, xshift=0.8cm, yshift=0.55cm] (fdb_t2) {} ;
      \node[latent, below=of fdb_t2, yshift=0.4cm] (db_t2) {$d$} ;
      \node[factor, below=of db_t2, yshift=0.55cm] (frb_t2) {};
      \node[det, below=of frb_t2, yshift=0.4cm, fill=black!10] (rb_t2) {$r$} ;

      \node[latent, above=of fdb_t2, xshift=0.8cm, yshift=-0.55cm] (p4) {$p_{c_t}$} ;
      \node[factor, above=of p4, yshift=-0.4cm] (fp4) {} ;
      \node[latent, above=of fp4,yshift=-0.55cm] (s4) {$s_{c_t}$} ;
      \node[factor, right=of s4,xshift=-0.4cm] (fs4) {} ;
      \node[const, right=of fs4] (ds4) {$\dots$} ;


      \edge[-] {s2_t2} {fp2_t2};
      \edge {fp2_t2} {p2_t2};
      \edge[-] {p2_t2, p4} {fdb_t2};
      \edge {fdb_t2} {db_t2};
      \edge[-] {db_t2} {frb_t2};
      \edge {frb_t2} {rb_t2};
      \edge {fs4} {s4};
      \edge[-] {s4} {fp4};
      \edge {fp4} {p4};


    }

}
}
\end{textblock}
}




\only<5->{
\begin{textblock}{160}(0,16)  \centering
\scalebox{1}{
\tikz{ %
        \node[latent] (s0) {$s_{t-1}$} ; %
        \node[const, left=of s0] (ns0) {$\dots$}; %

        \node[factor, right=of s0,xshift=1cm ] (fs1) {} ;
        \node[const, above=of fs1] (nfs1) {$f_{s_t}$}; %

        \node[latent, right=of fs1, xshift=1.25cm] (s1) {$s_t$} ; %

        \node[factor, right=of s1, xshift=1.25cm ] (fs2) {} ;
        \node[const, above=of fs2] (nfs2) {\ \ \ $f_{s_{t+1}}$}; %

        \node[latent, right=of fs2,xshift=1cm] (s2) {$s_{t+1}$} ; %
        \node[const, right=of s2] (ns2) {$\dots$}; %


        \node[factor, below=of s1,xshift=-1.4cm,yshift=-1cm] (fp0) {} ;
        \node[const, right=of fp0] (nfp0) {$f_{p_t(1)}$}; %

        \node[factor, color=white, below=of s1] (fp1) {} ;
        \node[const, below=of fp1, yshift=0.2cm] (nfp1) {$\dots$}; %

        \node[factor, below=of s1,xshift=1.4cm,yshift=-1cm] (fp2) {} ;
        \node[const, left=of fp2] (nfp2) {$f_{p_t(k)}$}; %

        \node[latent, below=of fp0] (p0) {\footnotesize$p_t(1)$} ; %
        %\node[latent, below=of fp1] (p1) {\footnotesize$p_i^{t}(2)$} ; %
        \node[latent, below=of fp2] (p2) {\footnotesize$p_t(k)$} ; %

        \node[invisible, above=of s1, yshift=1cm] (ia) {};
%         %\draw[bend right=90] (fs1) arc (s1) node[midway,above]{label};
        %\draw[bend left,->]  (fs1) to node [auto] {Link} (s1);
        \edge[-] {s1} {fp0,fp1,fp2};
        \edge {fp0} {p0};
        %\edge[-] {fp1} {p1};
        \edge {fp2} {p2};
        \edge {fs1} {s1};
        \edge[-] {fs1} {s0};
        \edge[-] {fs2} {s1};
        \edge {fs2} {s2};
        %\edge[bend right] {s0} {fs1};
        \only<6-22>{
          \path[draw, -latex, fill=black!50] (s0) edge[bend right,draw=black!50] node[midway,below,color=black!75] {\scriptsize \textit{forwardPosterior}$(s_{t-1})$} (fs1);
        }
        \only<7-21>{
          \path[draw, -latex, fill=black!50] (fs1) edge[bend left,draw=black!50] node[midway,above,color=black!75] {\scriptsize \textit{forwardPrior}$(s_t)$} (s1);
        }
        \only<8,16->{
          \path[draw, -latex, fill=black!50] (s2) edge[bend left,draw=black!50] node[midway,below,color=black!75] {\scriptsize \textit{\ \ backwardPosterior}$(s_{t+1})$} (fs2);
        }
        \only<8,17->{
          \path[draw, -latex, fill=black!50] (fs2) edge[bend right,draw=black!50] node[midway,above,color=black!75] {\scriptsize \textit{backwardPrior}$(s_t)$} (s1);
        }
        \only<10,18>{
          \path[draw, -latex, fill=black!50,sloped] (s1) edge[bend right,draw=black!50] node[midway,above,color=black!75] {\scriptsize \textit{\ \ withinPrior}$_k(s_t)$} (fp0);
        }
        \only<11-17,19->{
          \path[draw, -latex, fill=black!50,sloped] (fp0) edge[bend left,draw=black!50] node[midway,above,color=black!75] {\scriptsize \textit{likelihood}$_1(s_t)$} (s1);
        }
        \only<12,20>{
          \path[draw, -latex, fill=black!50,sloped] (s1) edge[bend left,draw=black!50] node[midway,above,color=black!75] {\scriptsize \textit{\ \ withinPrior}$_k(s_t)$} (fp2);
        }
        \only<13-19,21->{
          \path[draw, -latex, fill=black!50,sloped] (fp2) edge[bend right,draw=black!50] node[midway,above,color=black!75] {\scriptsize \textit{\ \ likelihood}$_k(s_t)$} (s1);
        }
        \only<14-15>{
          \path[draw, -latex, fill=black!50] (s1) edge[bend left,draw=black!50] node[midway,above,color=black!75] {\scriptsize \textit{forwardPosterior}$(s_t)$} (fs2);
        }
        \only<15>{
          \path[draw, -latex, fill=black!50] (fs2) edge[bend right,draw=black!50] node[midway,below,color=black!75] {\scriptsize \textit{\ \ forwardPrior}$(s_{t+1})$} (s2);
        }
        \only<22->{
          \path[draw, -latex, fill=black!50] (s1) edge[bend right,draw=black!50] node[midway,above,color=black!75] {\scriptsize \textit{backwardPosterior}$(s_t)$} (fs1);
        }
        \only<23->{
          \path[draw, -latex, fill=black!50] (fs1) edge[bend left,draw=black!50] node[midway,below,color=black!75] {\scriptsize \textit{backwardPrior}$(s_{t-1})$} (s0);
        }
}
}
\end{textblock}
}

\end{frame}


\begin{frame}[plain]
\begin{textblock}{160}(0,4)
\centering \LARGE Modelos de historia completa \\
\large Smoothing
\end{textblock}

\begin{textblock}{160}(0,20)
\begin{figure}[ht!]
  \centering
  \scalebox{.95}{
    \tikz{ %
      \node[latent] (s10) {$s_{a_0}$} ;
      %
      \node[latent,  below=of s10,yshift=-0.7cm] (s11) {$s_{a_1}$} ;

      \node[latent, right=of s11, xshift=3cm] (p11) {$p_{a_1}$} ;
      %
      \node[latent, below=of s11,yshift=-0.4cm] (s12) {$s_{a_2}$} ;
      \node[latent, right=of s12, xshift=3cm] (p12) {$p_{a_2}$} ;

      \node[const, right=of p11,xshift=0.5cm] (r1) {$\bm{>}$} ;
      \node[const, above=of r1, yshift=0.3cm] (nr1) {\footnotesize \ \  Observed result} ;
      \node[const, right=of p12,xshift=0.5cm] (r2) {$\bm{<}$} ;
      \node[const, above=of r2, yshift=0.3cm] (nr2) {\footnotesize \ \ Observed result} ;

      \node[latent, left=of s10, xshift=13.4cm] (s20) {$s_{b_0}$} ;
      \node[latent, below=of s20,yshift=-0.7cm] (s21) {$s_{b_1}$} ;
      \node[latent, left=of s21, xshift=-3cm] (p21) {$p_{b_1}$} ;

      \node[latent, below=of s21, yshift=-0.4cm] (s22) {$s_{b_2}$} ;
      \node[latent, left=of s22, xshift=-3cm] (p22) {$p_{b_2}$} ;


      \edge {s10} {s11};
      \edge {s11} {s12};
      \edge {s20} {s21};
      \edge {s21} {s22};
      \edge {s11} {p11};
      \edge {s12} {p12};
      \edge {s21} {p21};
      \edge {s22} {p22};

      \node[const, right=of s10, yshift=0cm,xshift=-0.15cm ] (wp10) {\includegraphics[page={73},width=.125\linewidth]{figuras/smoothing-por-pasos}} ;
      \node[const, left=of s20, yshift=0cm,xshift=0.15cm ] (wp20) {\includegraphics[page={73},width=.125\linewidth]{figuras/smoothing-por-pasos}} ;


      \node[const, left=of s11, yshift=0.6cm ] (post11) {\only<1-3>{\phantom{\includegraphics[page={1},width=.125\linewidth]{figuras/smoothing-por-pasos}}}\only<4-9>{\includegraphics[page={1},width=.125\linewidth]{figuras/smoothing-por-pasos}}\only<10-13>{\includegraphics[page={13},width=.125\linewidth]{figuras/smoothing-por-pasos}}\only<14>{\includegraphics[page={25},width=.125\linewidth]{figuras/smoothing-por-pasos}}\only<15>{\includegraphics[page={37},width=.125\linewidth]{figuras/smoothing-por-pasos}}\only<16>{\includegraphics[page={49},width=.125\linewidth]{figuras/smoothing-por-pasos}}\only<17>{\includegraphics[page={61},width=.125\linewidth]{figuras/smoothing-por-pasos}} } ;
      \node[const, right=of s11, yshift=0.6cm, xshift=-0.15cm ] (wp11) {\only<2-7>{\includegraphics[page={2},width=.125\linewidth]{figuras/smoothing-por-pasos}}\only<8-13>{\includegraphics[page={14},width=.125\linewidth]{figuras/smoothing-por-pasos}}\only<14>{\includegraphics[page={26},width=.125\linewidth]{figuras/smoothing-por-pasos}}\only<15>{\includegraphics[page={38},width=.125\linewidth]{figuras/smoothing-por-pasos}}\only<16>{\includegraphics[page={50},width=.125\linewidth]{figuras/smoothing-por-pasos}}\only<17>{\includegraphics[page={62},width=.125\linewidth]{figuras/smoothing-por-pasos}}} ;
      \node[const, left=of p11, yshift=0.6cm, xshift=0.15cm ] (lh11) {\only<3-8>{\includegraphics[page={3},width=.125\linewidth]{figuras/smoothing-por-pasos}}\only<9-13>{\includegraphics[page={15},width=.125\linewidth]{figuras/smoothing-por-pasos}}\only<14>{\includegraphics[page={27},width=.125\linewidth]{figuras/smoothing-por-pasos}}\only<15>{\includegraphics[page={39},width=.125\linewidth]{figuras/smoothing-por-pasos}}\only<16>{\includegraphics[page={51},width=.125\linewidth]{figuras/smoothing-por-pasos}}\only<17>{\includegraphics[page={63},width=.125\linewidth]{figuras/smoothing-por-pasos}}} ;

      \node[const, left=of s12, yshift=0.6cm ] (post12) {\only<7-12>{\includegraphics[page={4},width=.125\linewidth]{figuras/smoothing-por-pasos}}\only<13>{\includegraphics[page={16},width=.125\linewidth]{figuras/smoothing-por-pasos}}\only<14>{\includegraphics[page={28},width=.125\linewidth]{figuras/smoothing-por-pasos}}\only<15>{\includegraphics[page={40},width=.125\linewidth]{figuras/smoothing-por-pasos}}\only<16>{\includegraphics[page={52},width=.125\linewidth]{figuras/smoothing-por-pasos}}\only<17>{\includegraphics[page={64},width=.125\linewidth]{figuras/smoothing-por-pasos}} } ;
      \node[const, right=of s12, yshift=0.6cm, xshift=-0.15cm  ] (wp12) {\only<5-10>{\includegraphics[page={5},width=.125\linewidth]{figuras/smoothing-por-pasos}}\only<11-13>{\includegraphics[page={17},width=.125\linewidth]{figuras/smoothing-por-pasos}}\only<14>{\includegraphics[page={29},width=.125\linewidth]{figuras/smoothing-por-pasos}}\only<15>{\includegraphics[page={41},width=.125\linewidth]{figuras/smoothing-por-pasos}}\only<16>{\includegraphics[page={53},width=.125\linewidth]{figuras/smoothing-por-pasos}}\only<17>{\includegraphics[page={65},width=.125\linewidth]{figuras/smoothing-por-pasos}} } ;
      \node[const, left=of p12, yshift=0.6cm, xshift=0.15cm  ] (lh12) {\only<6-11>{\includegraphics[page={6},width=.125\linewidth]{figuras/smoothing-por-pasos}}\only<12-13>{\includegraphics[page={18},width=.125\linewidth]{figuras/smoothing-por-pasos}}\only<14>{\includegraphics[page={30},width=.125\linewidth]{figuras/smoothing-por-pasos}}\only<15>{\includegraphics[page={42},width=.125\linewidth]{figuras/smoothing-por-pasos}}\only<16>{\includegraphics[page={54},width=.125\linewidth]{figuras/smoothing-por-pasos}}\only<17>{\includegraphics[page={66},width=.125\linewidth]{figuras/smoothing-por-pasos}} } ;


      \node[const, right=of s21, yshift=0.6cm ] (post21) {\only<1-3>{\phantom{\includegraphics[page={7},width=.125\linewidth]{figuras/smoothing-por-pasos}}}\only<4-9>{\includegraphics[page={7},width=.125\linewidth]{figuras/smoothing-por-pasos}}\only<10-13>{\includegraphics[page={19},width=.125\linewidth]{figuras/smoothing-por-pasos}}\only<14>{\includegraphics[page={31},width=.125\linewidth]{figuras/smoothing-por-pasos}}\only<15>{\includegraphics[page={43},width=.125\linewidth]{figuras/smoothing-por-pasos}}\only<16>{\includegraphics[page={55},width=.125\linewidth]{figuras/smoothing-por-pasos}}\only<17>{\includegraphics[page={67},width=.125\linewidth]{figuras/smoothing-por-pasos}}}  ;
      \node[const, left=of s21, yshift=0.6cm, xshift=0.15cm  ] (wp21) {\only<2-7>{\includegraphics[page={8},width=.125\linewidth]{figuras/smoothing-por-pasos}}\only<8-13>{\includegraphics[page={20},width=.125\linewidth]{figuras/smoothing-por-pasos}}\only<14>{\includegraphics[page={32},width=.125\linewidth]{figuras/smoothing-por-pasos}}\only<15>{\includegraphics[page={44},width=.125\linewidth]{figuras/smoothing-por-pasos}}\only<16>{\includegraphics[page={56},width=.125\linewidth]{figuras/smoothing-por-pasos}}\only<17>{\includegraphics[page={68},width=.125\linewidth]{figuras/smoothing-por-pasos}}} ;
      \node[const, right=of p21, yshift=0.6cm, xshift=-0.15cm  ] (lh21) {\only<3-8>{\includegraphics[page={9},width=.125\linewidth]{figuras/smoothing-por-pasos}}\only<9-13>{\includegraphics[page={21},width=.125\linewidth]{figuras/smoothing-por-pasos}}\only<14>{\includegraphics[page={33},width=.125\linewidth]{figuras/smoothing-por-pasos}}\only<15>{\includegraphics[page={45},width=.125\linewidth]{figuras/smoothing-por-pasos}}\only<16>{\includegraphics[page={57},width=.125\linewidth]{figuras/smoothing-por-pasos}}\only<17>{\includegraphics[page={69},width=.125\linewidth]{figuras/smoothing-por-pasos}} } ;


      \node[const, right=of s22, yshift=0.6cm ] (post22) {\only<7-12>{\includegraphics[page={10},width=.125\linewidth]{figuras/smoothing-por-pasos}}\only<13>{\includegraphics[page={22},width=.125\linewidth]{figuras/smoothing-por-pasos}}\only<14>{\includegraphics[page={34},width=.125\linewidth]{figuras/smoothing-por-pasos}}\only<15>{\includegraphics[page={46},width=.125\linewidth]{figuras/smoothing-por-pasos}}\only<16>{\includegraphics[page={58},width=.125\linewidth]{figuras/smoothing-por-pasos}}\only<17>{\includegraphics[page={70},width=.125\linewidth]{figuras/smoothing-por-pasos}}} ;
      \node[const, left=of s22, yshift=0.6cm, xshift=0.15cm  ] (wp22) {\only<5-10>{\includegraphics[page={11},width=.125\linewidth]{figuras/smoothing-por-pasos}}\only<11-13>{\includegraphics[page={23},width=.125\linewidth]{figuras/smoothing-por-pasos}}\only<14>{\includegraphics[page={35},width=.125\linewidth]{figuras/smoothing-por-pasos}}\only<15>{\includegraphics[page={47},width=.125\linewidth]{figuras/smoothing-por-pasos}}\only<16>{\includegraphics[page={59},width=.125\linewidth]{figuras/smoothing-por-pasos}}\only<17>{\includegraphics[page={71},width=.125\linewidth]{figuras/smoothing-por-pasos}} } ;
      \node[const, right=of p22, yshift=0.6cm, xshift=-0.15cm  ] (lh22) {\only<6-11>{\includegraphics[page={12},width=.125\linewidth]{figuras/smoothing-por-pasos}}\only<12-13>{\includegraphics[page={24},width=.125\linewidth]{figuras/smoothing-por-pasos}}\only<14>{\includegraphics[page={36},width=.125\linewidth]{figuras/smoothing-por-pasos}}\only<15>{\includegraphics[page={48},width=.125\linewidth]{figuras/smoothing-por-pasos}}\only<16>{\includegraphics[page={60},width=.125\linewidth]{figuras/smoothing-por-pasos}}\only<17>{\includegraphics[page={72},width=.125\linewidth]{figuras/smoothing-por-pasos}}} ;

      \only<4->{\node[const, above=of post11] (npost11) {\scriptsize \only<10>{\phantom}{Posterior}} ;}
      \only<10>{\node[const, above=of post11] (npost11) {\scriptsize \textbf{Posterior}} ;}
      \only<2->{\node[const, above=of wp11] (nwp11) {\scriptsize \only<8>{\phantom}{Prior}} ;}
      \only<8>{\node[const, above=of wp11] (nwp11) {\scriptsize \textbf{Prior}} ;}
      \only<3->{\node[const, above=of lh11] (nlh11) {\scriptsize \only<9>{\phantom}{Likelihood}} ;}
      \only<9>{\node[const, above=of lh11] (nlh11) {\scriptsize \textbf{Likelihood}} ;}
      \only<4->{\node[const, above=of post21] (npost21) {\scriptsize \only<10>{\phantom}{Posterior}} ;}
      \only<10>{\node[const, above=of post21] (npost21) {\scriptsize \textbf{Posterior}} ;}
      \only<2->{\node[const, above=of wp21] (nwp21) {\scriptsize \only<8>{\phantom}{Prior}} ;}
      \only<8>{\node[const, above=of wp21] (nwp21) {\scriptsize \textbf{Prior}} ;}
      \only<3->{\node[const, above=of lh21] (nlh21) {\scriptsize \only<9>{\phantom}{Likelihood}} ;}
      \only<9>{\node[const, above=of lh21] (nlh21) {\scriptsize \textbf{Likelihood}} ;}

      \only<7->{\node[const, above=of post12] (npost12) {\scriptsize \only<13>{\phantom}{Posterior}} ;}
      \only<13>{\node[const, above=of post12] (npost12) {\scriptsize \textbf{Posterior}} ;}
      \only<5->{\node[const, above=of wp12] (nwp12) {\scriptsize \only<11>{\phantom}{Prior}} ;}
      \only<11>{\node[const, above=of wp12] (nwp12) {\scriptsize \textbf{Prior}} ;}
      \only<6->{\node[const, above=of lh12] (nlh12) {\scriptsize \only<12>{\phantom}{Likelihood}} ;}
      \only<12>{\node[const, above=of lh12] (nlh12) {\scriptsize \textbf{Likelihood}} ;}

      \only<7->{\node[const, above=of post22] (npost22) {\scriptsize \only<13>{\phantom}{Posterior}} ;}
      \only<13>{\node[const, above=of post22] (npost22) {\scriptsize \textbf{Posterior}} ;}
      \only<5->{\node[const, above=of wp22] (nwp22) {\scriptsize \only<11>{\phantom}{Prior}} ;}
      \only<11>{\node[const, above=of wp22] (nwp22) {\scriptsize \textbf{Prior}} ;}
      \only<6->{\node[const, above=of lh22] (nlh22) {\scriptsize \only<12>{\phantom}{Likelihood}} ;}
      \only<12>{\node[const, above=of lh22] (nlh22) {\scriptsize \textbf{Likelihood}} ;}

      \node[const, above=of wp10,yshift=-0.55cm] (nwp10) {\scriptsize Prior} ;
      \node[const, above=of wp20,yshift=-0.55cm] (nwp20) {\scriptsize Prior} ;

      }
  }
\end{figure}
\end{textblock}

\end{frame}


\begin{frame}[plain]
\begin{textblock}{160}(0,4)
\centering \LARGE TrueSkill Through Time \\
\large Estado del arte en la industria del videojuego.
\end{textblock}

\begin{textblock}{50}(5,28)
\centering
\includegraphics[width=0.7\textwidth]{../../auxiliar/static/julia.png}
\end{textblock}
\begin{textblock}{50}(55,28)
\centering
\includegraphics[width=1\textwidth]{../../auxiliar/static/python.png}
\end{textblock}
\begin{textblock}{50}(115,28) \raggedright
\includegraphics[width=0.55\textwidth]{../../auxiliar/static/R.png}
\end{textblock}


\end{frame}



\begin{frame}[plain]
\begin{textblock}{160}(0,4)
\centering \LARGE TrueSkill Through Time \\
\large Estado del arte en la industria del videojuego.
\end{textblock}


\begin{textblock}{155}(0,20) \centering
\includegraphics[width=0.9\textwidth]{figuras/atp.pdf}
\end{textblock}


\end{frame}


\begin{frame}[plain]
\begin{textblock}{160}(0,4)
\centering \LARGE Efecto causal temporal \\
\large \only<5>{El control sintético}
\end{textblock}

\only<1>{
\begin{textblock}{155}(0,20) \centering
\includegraphics[page=1,width=0.7\textwidth]{figuras/ICP}
\end{textblock}
}
\only<2>{
\begin{textblock}{155}(0,20) \centering
\includegraphics[page=2,width=0.7\textwidth]{figuras/ICP}
\end{textblock}
}
\only<3>{
\begin{textblock}{155}(0,20) \centering
\includegraphics[page=3,width=0.7\textwidth]{figuras/ICP}
\end{textblock}
}
\only<4->{
\begin{textblock}{155}(0,20) \centering
\includegraphics[page=4,width=0.7\textwidth]{figuras/ICP}
\end{textblock}
}

\only<5>{
\begin{textblock}{150}(45,28) \Large
\textbf{Necesitamos modelar \\ el contrafactual}
% Modelos lineales de series de tiempo \\[0.2cm]
%
% \large
%
% \ \ \ $\bullet$ El comportamiento de la serie temporal previa a la intervención
%
% \ \ \ $\bullet$ El comportamiento de otras series temporales (misma serie en otras regiones)
%
% \ \ \ $\bullet$ Información del conocimiento experto
\end{textblock}
}


\end{frame}

\end{comment}
%
% \begin{frame}[plain]
% \begin{textblock}{160}(0,4)
% \centering \LARGE Modelos lineas para series temporales \\
% \large Modelo lineal básico
% \end{textblock}
% %
%
% \begin{textblock}{80}(80,18)
% \begin{equation*}
% \begin{split}
% y &= w_0 + w_1 \, x_1 + w_2 \, x_2^2 + w_3 \, x_3^3 + \varepsilon \\
% & = \sum_{i=0}^{m-1} w_i \phi_i(x) = \bm{w}^T \bm{\phi}(x) \\[0.2cm]
% p(\varepsilon) &= \N(\varepsilon \,|\, 0, \beta^2) \\[0.6cm]
% p(w_i) &= \N(w_i \,|\, 0, \sigma_{i}^2) \\[0cm]
% \end{split}
% \end{equation*}
% \end{textblock}
%
% \only<1->{
% \begin{textblock}{80}(0,24)
% \begin{figure}[H]
%     \centering
%     \tikz{
%     \node[latent, fill=black!20] (y) {$y$} ; %
%     \node[latent, above=of y, xshift=-1cm, yshift=-0.33cm ,fill=black!100, minimum size=2pt] (x) {} ; %
%     \node[const, left=of x] (c_x) {$\bm{\phi}(x)$};
%     \node[latent, fill=black!100, left=of y,minimum size=2pt] (beta)
%     {} ; %
%     \node[const, left=of beta] (c_beta) {$\beta$};
%     \node[latent, fill=black!0, above=of y] (w) {$\bm{w}$};
%     \node[latent, above=of w, fill=black!100, minimum size=2pt] (alpha) {} ; %
%     \node[const, right=of alpha] (c_alpha) {$\bm{\sigma}$};
%
%     \edge {x,beta,w} {y};
%     \edge {alpha} {w};
%
%     \node[invisible, fill=black!0, minimum size=0pt, xshift=-0.52cm] (data_inv) {} ; %
%
%     %\plate {no} {(x)(y)(data_inv)(c_x)} {$i: $ Datos\vspace{1cm}}; %
%     }
% \end{figure}
% \end{textblock}
% }
%
%
% \end{frame}
%

\begin{frame}[plain]
\begin{textblock}{160}(0,4)
\centering \LARGE Modelos lineas para series temporales \\
\large \only<1-4>{Modelo lineal básico}\only<5->{Structural state-space models}
\end{textblock}

\only<1->{
\begin{textblock}{155}(5,36)
\raggedright
\tikz{
    \node[latent, fill=black!20] (y) {$y\only<5->{_0}$} ; %
    \node[factor, above=of y, yshift=-0.4cm] (fy) {} ; %
    \only<1-4>{
      \node[const, right=of fy] (py) {$p(y|\bm{w}, x) = \N(\, y \, | \, \only<1>{w_0 + w_1 \, x_1 + w_2 \, x_2^2 + w_3 \, x_3^3}\only<2>{\sum_i w_i \phi_i(x)}\only<3->{\bm{w}^T \bm{\phi}(x)}, \, \beta^2) $};
    }


    \node[latent, left=of fy, xshift=0.75cm ,fill=black!100, minimum size=2pt] (x) {} ;

    \node[const, left=of x] (c_x) {$x\only<5->{_0}$};

    \node[latent, above=of fy, yshift=-0.55cm] (w) {$\bm{w}\only<5->{_0}$};

    \only<1-4>{\node[factor, above=of w, yshift=-0.4cm] (fw) {};}
    \only<5->{\node[factor, left=of w, xshift=0.4cm] (fw) {};}
    \only<1-4>{
      \node[const, right=of fw] (py) {$p(\bm{w}) = \only<1-3>{\prod_i \N(\, w_i \, | \, 0 , \, \sigma^2)}\only<4->{\N(\, \bm{w} \, | \, 0 , \, \bm{S})}$};
    }
    \only<5->{
      \node[const, above=of fw] (py) {$p(\bm{w})$};
    }

%     \only<5->{
%       \node[const, right=of fy] (py) {$p(y|\bm{w}, x)$};
%     }

    \node[invisible, above=of w, yshift=1cm] (ia) {};
    \node[invisible, left=of y, xshift=-1.3cm] (il) {};

    \only<5->{
      \node[factor, right=of w] (fw1) {};
      \node[latent, right=of fw1] (w1) {$\bm{w_1}$};
      \node[latent, below=of fw1, yshift=0.75cm,fill=black!100, minimum size=2pt] (z1) {} ;
      \node[const, left=of z1] (nz1) {$z_1$};
      \node[factor, below=of w1, yshift=0.55cm] (fy1) {};
      \node[latent, below=of fy1, fill=black!15, yshift=0.4cm] (y1) {$y_1$};
       \node[latent, left=of fy1, xshift=0.75cm ,fill=black!100, minimum size=2pt] (x1) {} ;
      \node[const, left=of x1] (nx1) {$x_1$};
      \node[const, above=of fw1] (nfw1) {$p(\bm{w}_{t+1}|\bm{w}_t, z_t)$};
      \node[const, right=of fy1] (nfy1) {$p(y_y|\bm{w}_t, x_t) $};
    }

    \edge {fw} {w};
    \edge[-] {x,w} {fy};
    \edge {fy} {y};
    \only<5->{
      \edge[-] {w,z1} {fw1};
      \edge {fw1} {w1};
      \edge[-] {x1,w1} {fy1};
      \edge {fy1} {y1};
    }

    %\plate {no} {(x)(y)(data_inv)(c_x)} {$i: $ Datos\vspace{1cm}}; %
}
\end{textblock}
}


\only<6->{
\begin{textblock}{160}(20,18)
\begin{flalign*}
 %\bm{w}_{t+1} = \Phi(z_t) \bm{w}_{t} + R_t & \nu\\ & p(\nu)
& p(\bm{w}_{t+1}| \bm{w}_{t}, z_t) =  \N(\, \bm{w}_{t+1} \, | \,  \Phi(z_t) \bm{w}_{t}, \, Q)
&&
\end{flalign*}
\end{textblock}
}



\end{frame}



\begin{frame}[plain]
\begin{textblock}{160}(0,4)
\centering \LARGE Modelos lineas para series temporales \\
\large Structural state-space models
\end{textblock}


\begin{textblock}{150}(5,74)
\raggedright
\tikz{
  \node[latent, fill=black!10] (y0) {$y_{t-1}$};
  \node[latent, fill=black!10, right=of y0] (y1) {$y_{t}$};
  \node[latent, right=of y1] (y2) {$y_{t+1}$};
}
\end{textblock}


\begin{textblock}{80}(100,24)
$y_t  =  \bm{\phi}(x)_t^T \, \bm{w}_t + \varepsilon_t$
\end{textblock}

\begin{textblock}{80}(100,34)
$\alpha_{t+1} = T_t \, \alpha_t + R_t  \nu_t$
\end{textblock}


\begin{textblock}{80}(100,44)
$\bullet$ $X_t \in \mathbb{R}^{d \times d}$: matriz de transición.
$\bullet$ $T_t \in \mathbb{R}^{d \times d}$: matriz de transición.
\end{textblock}


\end{frame}

\begin{frame}[plain,noframenumbering]
\centering \vspace{0.5cm}
\includegraphics[width=1\textwidth]{../../auxiliar/static/BP.png}
\end{frame}





%
% \begin{frame}[plain]
% \begin{textblock}{96}(0,6.5)\centering
% {\transparent{0.9}\includegraphics[width=0.8\textwidth]{../../auxiliar/static/inti.png}}
% \end{textblock}
%
% \begin{textblock}{160}(96,5.5)
% \includegraphics[width=0.35\textwidth]{../../auxiliar/static/pachacuteckoricancha}
% \end{textblock}
% \end{frame}





\end{document}



