\newif\ifen
\newif\ifes
\newif\iffr
\newcommand{\fr}[1]{\iffr#1 \fi}
\newcommand{\En}[1]{\ifen#1\fi}
\newcommand{\Es}[1]{\ifes#1\fi}
\estrue
\documentclass[shownotes,aspectratio=169]{beamer}

\usepackage{siunitx}

\usepackage{ragged2e} %\justifying
\usepackage{paracol}
\usepackage[utf8]{inputenc} %Para acentos en UTF8 (Prueba: á é í ó ú Á É Í Ó Ú ñ Ñ)
\usepackage{url}
%\usepackage{mathtools}
\usepackage{graphicx}
\usepackage{caption}
\usepackage{float} % para que los gr\'aficos se queden en su lugar con [H]
\usepackage[fleqn]{mathtools} % \coloneqq, flalign
\usepackage{subcaption}
\usepackage{wrapfig}
\usepackage{soul,color} %\st{Hellow world}
\usepackage{xcolor} %\st{Hellow world}
\usepackage[fleqn]{amsmath} %para escribir funci\'on partida
\usepackage{blkarray}
\usepackage{hyperref} % para inlcuir links dentro del texto
\usepackage{tabu} 
\usepackage{comment}
\usepackage{amsfonts} % mathbb{N} -> conjunto de los n\'umeros naturales  
\usepackage{enumerate}
\usepackage{listings}
\usepackage[shortlabels]{enumitem} %  shortlabels option to have compatibility with the enumerate-like scheme for label
\usepackage{framed}
\usepackage{mdframed}
\usepackage{multicol}
\usepackage{transparent} % \transparent{1.0}
\usepackage{bm} 
\usepackage[makeroom]{cancel} % \cancel{} \bcancel{} etc
\usepackage[absolute,overlay]{textpos} %no funciona
\setlength{\TPHorizModule}{1mm} %128mm  mitad: 64 
\setlength{\TPVertModule}{1mm}	%96mm  mitad 48

\newif\ifen
\newif\ifes
\newcommand{\en}[1]{\ifen#1\fi}
\newcommand{\es}[1]{\ifes#1\fi}
\estrue


\usepackage{todonotes}
\setbeameroption{show notes}
\usepackage{rotating}
\usepackage{transparent}


\newcommand{\E}{\en{S}\es{E}}
\newcommand{\A}{\en{E}\es{A}}
\newcommand{\Ee}{\en{s}\es{e}}
\newcommand{\Aa}{\en{e}\es{a}}

\hypersetup{
    colorlinks=true,
    linkcolor={red!50!black},
    citecolor={blue!35!black},
    urlcolor={blue!35!black}
}

\newcommand\hfrac[2]{\genfrac{}{}{0pt}{}{#1}{#2}} %\frac{}{} sin la linea del medio

\newcommand{\indep}{\perp \!\!\! \perp}
\newcommand{\N}{\mathcal{N}}
\newcommand{\vm}[1]{\mathbf{#1}}

\newtheorem{midef}{Definition}
\newtheorem{miteo}{Theorem}
\newtheorem{mipropo}{Proposition}

\usefonttheme[onlymath]{serif}


\usepackage{tikz} % Para graficar, por ejemplo bayes networks
%\usetikzlibrary{bayesnet} % Para que ande se necesita copiar el archivo  tikzlibrarybayesnet.code.tex en la misma carpeta

%%%%%%%%%%%%%%%%%%%%%%%%%%%%%%%%%5
%
% Incompatibles con textpos
%
%\usepackage{todonotes}
%\usepackage{tikz} % Para graficar, por ejemplo bayes networks
%
%%%%%%%%%%%%%%%%%%%%%%%%%%%%%%%%%%



\usepackage[absolute,overlay]{textpos} %no funciona
\setlength{\TPHorizModule}{1mm} %128mm  mitad: 64 
\setlength{\TPVertModule}{1mm}	%96mm  mitad 48
% 
% 
\captionsetup[figure]{labelformat=empty}

% 
% http://latexcolor.com/
\definecolor{lightseagreen}{rgb}{0.13, 0.7, 0.6.5}
\definecolor{greenblue}{rgb}{0.1, 0.55, 0.5}
\definecolor{redgreen}{rgb}{0.6, 0.4, 0.}
\definecolor{greenred}{rgb}{0.4, 0.7, 0.}
\definecolor{redblue}{rgb}{0.4, 0., .4}
\definecolor{tangelo}{rgb}{0.98, 0.3, 0.0}
\definecolor{git}{rgb}{0.94, 0.309, 0.2}
% 
\setbeamercolor{structure}{fg=greenblue}


%http://latexcolor.com/
\definecolor{azul}{rgb}{0.36, 0.54, 0.66}
\definecolor{rojo}{rgb}{0.7, 0.2, 0.116}
\definecolor{rojopiso}{rgb}{0.8, 0.25, 0.17}
\definecolor{verdeingles}{rgb}{0.12, 0.5, 0.17}
\definecolor{ubuntu}{rgb}{0.44, 0.16, 0.39}
\definecolor{debian}{rgb}{0.84, 0.04, 0.33}
\definecolor{dkgreen}{rgb}{0,0.6,0}
\definecolor{gray}{rgb}{0.5,0.5,0.5}
\definecolor{mauve}{rgb}{0.58,0,0.82}




\newcommand\Wider[2][3em]{%
\makebox[\linewidth][c]{%
  \begin{minipage}{\dimexpr\textwidth+#1\relax}
  \raggedright#2
  \end{minipage}%
  }%
}

\newenvironment{ejercicio}[1]{
% \setbeamercolor{block title}{bg=tangelo, fg=white}
\begin{exampleblock}{#1}
}{
\end{exampleblock}
}

\newenvironment{resumen}[1]{
\setbeamercolor{block title}{bg=git, fg=white}
\begin{block}{#1}
}{
\end{block}
}

\newenvironment{comando}{
\setbeamercolor{block body}{bg=git, fg=white}
\begin{block}{}
\begin{center}
\LARGE
\begin{texttt}
}{
\end{texttt}
\end{center}
\end{block}
}



% tikzlibrary.code.tex
%
% Copyright 2010-2011 by Laura Dietz
% Copyright 2012 by Jaakko Luttinen
%
% This file may be distributed and/or modified
%
% 1. under the LaTeX Project Public License and/or
% 2. under the GNU General Public License.
%
% See the files LICENSE_LPPL and LICENSE_GPL for more details.

% Load other libraries

%\newcommand{\vast}{\bBigg@{2.5}}
% newcommand{\Vast}{\bBigg@{14.5}}
% \usepackage{helvet}
% \renewcommand{\familydefault}{\sfdefault}

\usetikzlibrary{shapes}
\usetikzlibrary{fit}
\usetikzlibrary{chains}
\usetikzlibrary{arrows}

% Latent node
\tikzstyle{latent} = [circle,fill=white,draw=black,inner sep=1pt,
minimum size=20pt, font=\fontsize{10}{10}\selectfont, node distance=1]
% Observed node
\tikzstyle{obs} = [latent,fill=gray!25]
% Invisible node
\tikzstyle{invisible} = [latent,minimum size=0pt,color=white, opacity=0, node distance=0]
% Constant node
\tikzstyle{const} = [rectangle, inner sep=0pt, node distance=0.1]
%state
\tikzstyle{estado} = [latent,minimum size=8pt,node distance=0.4]
%action
\tikzstyle{accion} =[latent,circle,minimum size=5pt,fill=black,node distance=0.4]


% Factor node
\tikzstyle{factor} = [rectangle, fill=black,minimum size=10pt, draw=black, inner
sep=0pt, node distance=1]
% Deterministic node
\tikzstyle{det} = [latent, rectangle]

% Plate node
\tikzstyle{plate} = [draw, rectangle, rounded corners, fit=#1]
% Invisible wrapper node
\tikzstyle{wrap} = [inner sep=0pt, fit=#1]
% Gate
\tikzstyle{gate} = [draw, rectangle, dashed, fit=#1]

% Caption node
\tikzstyle{caption} = [font=\footnotesize, node distance=0] %
\tikzstyle{plate caption} = [caption, node distance=0, inner sep=0pt,
below left=5pt and 0pt of #1.south east] %
\tikzstyle{factor caption} = [caption] %
\tikzstyle{every label} += [caption] %

\tikzset{>={triangle 45}}

%\pgfdeclarelayer{b}
%\pgfdeclarelayer{f}
%\pgfsetlayers{b,main,f}

% \factoredge [options] {inputs} {factors} {outputs}
\newcommand{\factoredge}[4][]{ %
  % Connect all nodes #2 to all nodes #4 via all factors #3.
  \foreach \f in {#3} { %
    \foreach \x in {#2} { %
      \path (\x) edge[-,#1] (\f) ; %
      %\draw[-,#1] (\x) edge[-] (\f) ; %
    } ;
    \foreach \y in {#4} { %
      \path (\f) edge[->,#1] (\y) ; %
      %\draw[->,#1] (\f) -- (\y) ; %
    } ;
  } ;
}

% \edge [options] {inputs} {outputs}
\newcommand{\edge}[3][]{ %
  % Connect all nodes #2 to all nodes #3.
  \foreach \x in {#2} { %
    \foreach \y in {#3} { %
      \path (\x) edge [->,#1] (\y) ;%
      %\draw[->,#1] (\x) -- (\y) ;%
    } ;
  } ;
}

% \factor [options] {name} {caption} {inputs} {outputs}
\newcommand{\factor}[5][]{ %
  % Draw the factor node. Use alias to allow empty names.
  \node[factor, label={[name=#2-caption]#3}, name=#2, #1,
  alias=#2-alias] {} ; %
  % Connect all inputs to outputs via this factor
  \factoredge {#4} {#2-alias} {#5} ; %
}

% \plate [options] {name} {fitlist} {caption}
\newcommand{\plate}[4][]{ %
  \node[wrap=#3] (#2-wrap) {}; %
  \node[plate caption=#2-wrap] (#2-caption) {#4}; %
  \node[plate=(#2-wrap)(#2-caption), #1] (#2) {}; %
}

% \gate [options] {name} {fitlist} {inputs}
\newcommand{\gate}[4][]{ %
  \node[gate=#3, name=#2, #1, alias=#2-alias] {}; %
  \foreach \x in {#4} { %
    \draw [-*,thick] (\x) -- (#2-alias); %
  } ;%
}

% \vgate {name} {fitlist-left} {caption-left} {fitlist-right}
% {caption-right} {inputs}
\newcommand{\vgate}[6]{ %
  % Wrap the left and right parts
  \node[wrap=#2] (#1-left) {}; %
  \node[wrap=#4] (#1-right) {}; %
  % Draw the gate
  \node[gate=(#1-left)(#1-right)] (#1) {}; %
  % Add captions
  \node[caption, below left=of #1.north ] (#1-left-caption)
  {#3}; %
  \node[caption, below right=of #1.north ] (#1-right-caption)
  {#5}; %
  % Draw middle separation
  \draw [-, dashed] (#1.north) -- (#1.south); %
  % Draw inputs
  \foreach \x in {#6} { %
    \draw [-*,thick] (\x) -- (#1); %
  } ;%
}

% \hgate {name} {fitlist-top} {caption-top} {fitlist-bottom}
% {caption-bottom} {inputs}
\newcommand{\hgate}[6]{ %
  % Wrap the left and right parts
  \node[wrap=#2] (#1-top) {}; %
  \node[wrap=#4] (#1-bottom) {}; %
  % Draw the gate
  \node[gate=(#1-top)(#1-bottom)] (#1) {}; %
  % Add captions
  \node[caption, above right=of #1.west ] (#1-top-caption)
  {#3}; %
  \node[caption, below right=of #1.west ] (#1-bottom-caption)
  {#5}; %
  % Draw middle separation
  \draw [-, dashed] (#1.west) -- (#1.east); %
  % Draw inputs
  \foreach \x in {#6} { %
    \draw [-*,thick] (\x) -- (#1); %
  } ;%
}


 \mode<presentation>
 {
 %   \usetheme{Madrid}      % or try Darmstadt, Madrid, Warsaw, ...
 %   \usecolortheme{default} % or try albatross, beaver, crane, ...
 %   \usefonttheme{serif}  % or try serif, structurebold, ...
  \usetheme{Antibes}
  \setbeamertemplate{navigation symbols}{}
 }
\estrue
\usepackage{todonotes}
\setbeameroption{show notes}
%
\newcommand{\gray}{\color{black!55}}
\usepackage{ulem} % sout
\usepackage{mdframed}
\usepackage{listings}
\lstset{
  aboveskip=3mm,
  belowskip=3mm,
  showstringspaces=true,
  columns=flexible,
  basicstyle={\footnotesize\ttfamily},
  breaklines=true,
  breakatwhitespace=true,
  tabsize=4,
  showlines=true,
}

\begin{document}

\color{black!85}
\large
%
% \begin{frame}[plain,noframenumbering]
%
%
% \begin{textblock}{160}(0,0)
% \includegraphics[width=1\textwidth]{../../auxiliar/static/deforestacion}
% \end{textblock}
%
% \begin{textblock}{80}(18,9)
% \textcolor{black!15}{\fontsize{44}{55}\selectfont Verdades}
% \end{textblock}
%
% \begin{textblock}{47}(85,70)
% \centering \textcolor{black!15}{{\fontsize{52}{65}\selectfont Empíricas}}
% \end{textblock}
%
% \begin{textblock}{80}(100,28)
% \LARGE  \textcolor{black!15}{\rotatebox[origin=tr]{-3}{\scalebox{9}{\scalebox{1}[-1]{$p$}}}}
% \end{textblock}
%
% \begin{textblock}{80}(66,43)
% \LARGE  \textcolor{black!15}{\scalebox{6}{$=$}}
% \end{textblock}
%
% \begin{textblock}{80}(36,29)
% \LARGE  \textcolor{black!15}{\scalebox{9}{$p$}}
% \end{textblock}
%
% %
% %
% % \begin{textblock}{160}(01,81)
% % \footnotesize \textcolor{black!5}{\textbf{\small Seminario ``Acuerdos intersubjetivos''\\
% % Comunidad Bayesiana Plurinacional} \\}
% % \end{textblock}
%
% \end{frame}

%%%%%%%%%%%%%%%%%%%%%%%%%%%%%%%%%%%%%%%%%


\begin{frame}[plain,noframenumbering]
\begin{textblock}{160}(0,-4.3) \centering
\includegraphics[width=1\textwidth]{../../auxiliar/static/antartic}
\end{textblock}

\begin{textblock}{160}(0,0) \centering
\tikz{
\node[det, fill=black,draw=black] (k) {\textcolor{black}{--------------------------------------------------------------------------------------------------------------------------------------}} ;
}
\end{textblock}

\begin{textblock}{160}(5,0)
\tikz{
\node[det, fill=black,draw=black,text width=0.01cm] (k) {\textcolor{black}{--------------------------------------------------------------------------------------------------------------------------------------}} ;
}
\end{textblock}


\begin{textblock}{160}(0,4) \centering
\LARGE \hspace{1cm} \textcolor{black!20}{\fontsize{22}{0}\selectfont \textbf{Modelos de historia \\ \hspace{1cm} completa}}
\end{textblock}


\begin{textblock}{55}[0,1](8,70)
\begin{turn}{90}
\parbox{6cm}{\footnotesize
\textcolor{black!10}{Millones de km$^2$ de hielo Antártico}}
\end{turn}
\end{textblock}


\begin{textblock}{160}(20,63)
\textcolor{black!10}{Unidad 4 \\ \small
Redes bayesianas de historia completa. \\
El problema de usar el posterior como prior del siguiente evento\\
El algoritmo de inferencia por loopy belief propagation. \\
Consideraciones de inferencia causal en series temporales. \\
}
\end{textblock}

\end{frame}


\begin{frame}[plain]
\begin{textblock}{160}(0,4)
\centering \LARGE Series de tiempo \\
\large Creencias adaptativas
\end{textblock}


\only<1-4>{
\begin{textblock}{160}(0,26) \centering

\Large La función de costo epistémica

\large
\begin{equation*}
\underbrace{P(\text{Hipótesis},\text{\En{Data}\Es{Datos}})}_{\hfrac{\text{\footnotesize\En{Initial belief compatible}\Es{Creencia compatible }}}{\text{\footnotesize \En{with the data}\Es{con los datos}}}} = \underbrace{P(\text{Hipótesis})}_{\hfrac{\text{\footnotesize\En{Initial intersubjective}\Es{Acuerdo intersubjetivo}}}{\text{\footnotesize\En{agreement}\Es{inicial}}}} \underbrace{P(\text{dato}_1 |\text{Hipótesis})}_{\text{\footnotesize Predic\En{tion}\Es{ción} 1}} \, \underbrace{P(\text{dato}_2 | \text{dato}_1 , \text{Hipótesis})}_{\text{\footnotesize Predic\En{tion}\Es{ción} 2}} \dots
\end{equation*}

\vspace{0.8cm}


\only<2>{
\Large Un único 0 en la secuencia de predicciones

hace falsa la hipótesis para siempre.\\
}\only<3-4>{
\Large Esa persona no está apta para realizar esa tarea.

\only<4>{\textbf{¿Para siempre?!}}
}

\end{textblock}
}
\only<5>{ \centering
\begin{textblock}{160}(0,-93)
\includegraphics[width=0.9\textwidth]{../../auxiliar/static/lifeHistory.jpeg}
\end{textblock}
}


\end{frame}


\begin{frame}[plain]
\begin{textblock}{160}(0,4)
\centering \LARGE Estimación de habilidad \\
\large en la industria del videojuego
\end{textblock}


 \only<1->{
 \begin{textblock}{140}(3,24)
 \normalsize
\tikz{
    \node[det, fill=black!10] (r) {$r$} ;
    \node[const, right=of r] (dr) {\normalsize $ P(r|d) = \mathbb{I}(r = d > 0)$};

    \node[latent, above=of r, yshift=-0.45cm] (d) {$d$} ; %
    \node[const, right=of d] (dd) {\normalsize $ p(d|p_a,p_b) = \delta(d = p_a-p_b)$};

    \node[latent, above=of d, xshift=-0.8cm, yshift=-0.45cm] (p1) {$p_a$} ; %
    \node[latent, above=of d, xshift=0.8cm, yshift=-0.45cm] (p2) {$p_b$} ; %


    \node[latent, above=of p1, yshift=-0.35cm] (s1) {$s_a$} ; %
    \node[latent, above=of p2, yshift=-0.35cm] (s2) {$s_b$} ; %
    \node[const, right=of s2] (ds2) {$p(s_i) = \N(s_i|\mu_i,\sigma_i^2)$};

    \node[const, right=of p2] (dp2) {\normalsize $p(p_i|s_i) = \N(p_i|s_i,\beta^2)$};

    \node[const, above=of dr] (r_name) {\small Resultado};
    \node[const, above=of dd] (d_name) {\small Diferencia};
    \node[const, above=of dp2] (p_name) {\small Desempeño};

    \node[const, above=of ds2, yshift=0.1cm] (s_name) {\small Habilidad};

    \edge {d} {r};
    \edge {p1,p2} {d};
    \edge {s1} {p1};
    \edge {s2} {p2};

}
\end{textblock}
}


\only<2>{
\begin{textblock}{100}(60,22) \centering
Prior \ $\rightarrow$ \ Posterior

\includegraphics[width=0.85\textwidth,page=4]{../3-dato/figuras/posterior_win.pdf}
\end{textblock}
}


\only<3->{
\begin{textblock}{100}(60,24) \centering
¿Cómo estimamos una habilidad en el tiempo?

\Large \vspace{0.8cm}

\only<4->{
¿Si usamos el último posterior como

prior del siguiente evento?

\vspace{0.6cm} \large

Posterior$_t$ $\rightarrow$ Prior$_{t+1}$
}

\only<5->{
\begin{equation*}
\underbrace{\N(\text{Habilidad}_{t+1} \, | \, \text{Habilidad}_{t} , \gamma^2)}_{\text{Prior}_{t+1}}
\end{equation*}
}
\end{textblock}
}
\end{frame}


\begin{frame}[plain]
\begin{textblock}{160}(20,4)
\centering \LARGE Estimación de habilidad \\
\large en la industria del videojuego
\end{textblock}

\only<1->{
 \begin{textblock}{140}(2,6)
 \normalsize
\tikz{
    \node[det, fill=black!10] (r) {$r$} ;
    \node[factor, above=of r, yshift=-0.4cm] (fr) {};
    \node[const, left=of fr] (dr) {\normalsize $\mathbb{I}(r = d > 0)$};

    \node[latent, above=of fr, yshift=-0.55cm] (d) {$d$} ; %
    \node[factor, above=of d, yshift=-0.4cm] (fd) {};

    \node[const, left=of fd] (dd) {\normalsize $\delta(d = p_a-p_b)$};

    \node[latent, above=of fd, xshift=-0.8cm, yshift=-0.55cm] (p1) {$p_a$} ; %
    \node[latent, above=of fd, xshift=0.8cm, yshift=-0.55cm] (p2) {$p_b$} ; %

    \node[factor, above=of p1, yshift=-0.4cm] (fp1) {};
    \node[factor, above=of p2, yshift=-0.4cm] (fp2) {};

    \node[latent, above=of fp1, yshift=-0.55cm] (s1) {$s_a$} ; %
    \node[latent, above=of fp2, yshift=-0.55cm] (s2) {$s_b$} ; %

    \node[factor, above=of s1, yshift=-0.4cm] (fs1) {};
    \node[factor, above=of s2, yshift=-0.4cm] (fs2) {};


    \node[const, left=of fs1] (ds2) {$\N(s_i|\mu_i,\sigma_i^2)$};

    \node[const, left=of fp1] (dp2) {\normalsize $\N(p_i|s_i,\beta^2)$};

    \node[const, left=of r] (r_name) {\small Resultado};
    \node[const, left=of d] (d_name) {\small Diferencia};
    \node[const, left=of p1] (p_name) {\small Desempeño};

    \node[const, left=of s1, yshift=0.1cm] (s_name) {\small Habilidad \ };

    \edge {d} {r};
    \edge[-] {p1,p2} {fd};
    \edge {fd} {d};
    \edge[-] {s1} {fp1};
    \edge[-] {s2} {fp2};
    \edge {fp1} {p1};
    \edge {fp2} {p2};
    \edge {fs1} {s1};
    \edge {fs2} {s2};

    \onslide<2->{
      \path[draw, ->, fill=black!50] (fs2) edge[bend left,draw=black!50] node[right,color=black!75] {\large\only<2>{$\N(s_b|\mu_b,\sigma^2)$}} (s2);
    }
    \onslide<3->{
      \path[draw, ->, fill=black!50] (s2) edge[bend left,draw=black!50] node[right,color=black!75] {\large\only<3>{$\N(s_b|\mu_b,\sigma^2)$}} (fp2);
    }
    \onslide<4->{
      \path[draw, ->, fill=black!50] (fp2) edge[bend left,draw=black!50] node[right,color=black!75] {\large$\only<4-5>{\int \N(p_b|s_b,\beta^2) \N(s_b|\mu_b,\sigma^2) \, ds_b }\only<5>{= p(p_b)}$} (p2);
    }
}
\end{textblock}
}



\only<6->{
\begin{textblock}{120}(65,18)
  \begin{flalign*}
   p(p_b) &= \int \N(p_b|s_b,\beta^2) \N(s_b|\mu_b,\sigma_b^2) ds_b \\
    \only<9>{& \overset{*}{=} \int \N(p_a|\mu_a,\beta^2 + \sigma_a^2) \, \N(s_a | \mu_*,\sigma_{*}^2)ds_a}
    \only<10>{& \overset{*}{=} \int \underbrace{\N(p_a|\mu_a,\beta^2 + \sigma_a^2)}_{\text{const.}} \underbrace{\N(s_a | \mu_*,\sigma_{*}^2)ds_a}_{1}}
    \only<11>{& \overset{*}{=} \phantom{\int} \, \N(p_a|\mu_a,\beta^2 + \sigma_a^2) \phantom{\N(s_a | \mu_*,\sigma_{*}^2)ds_a}}
    &&
  \end{flalign*}
\end{textblock}
}


\only<7-8>{
\begin{textblock}{120}(40,34)
  \begin{figure}[H]
     \centering
     \onslide<7->{
     \begin{subfigure}[b]{0.4\textwidth}
       \includegraphics[page=1,width=1\textwidth]{figuras/paso_1_multiplicacion_normales_image}
     \end{subfigure}}
     \onslide<8->{
     \begin{subfigure}[b]{0.4\textwidth}
       \includegraphics[page=1,width=\textwidth]{figuras/paso_1_multiplicacion_normales_p}
     \end{subfigure}}
  \end{figure}
\end{textblock}
}


\end{frame}



\begin{frame}[plain,noframenumbering]
\centering \vspace{0.5cm}
\includegraphics[width=1\textwidth]{../../auxiliar/static/BP.png}
\end{frame}





%
% \begin{frame}[plain]
% \begin{textblock}{96}(0,6.5)\centering
% {\transparent{0.9}\includegraphics[width=0.8\textwidth]{../../auxiliar/static/inti.png}}
% \end{textblock}
%
% \begin{textblock}{160}(96,5.5)
% \includegraphics[width=0.35\textwidth]{../../auxiliar/static/pachacuteckoricancha}
% \end{textblock}
% \end{frame}





\end{document}



