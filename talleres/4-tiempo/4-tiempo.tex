\newif\ifen
\newif\ifes
\newif\iffr
\newcommand{\fr}[1]{\iffr#1 \fi}
\newcommand{\En}[1]{\ifen#1\fi}
\newcommand{\Es}[1]{\ifes#1\fi}
\estrue
\documentclass[shownotes,aspectratio=169]{beamer}

\usepackage{siunitx}
\input{../../auxiliar/tex/diapo_encabezado.tex}
\input{../../auxiliar/tex/tikzlibrarybayesnet.code.tex}
 \mode<presentation>
 {
 %   \usetheme{Madrid}      % or try Darmstadt, Madrid, Warsaw, ...
 %   \usecolortheme{default} % or try albatross, beaver, crane, ...
 %   \usefonttheme{serif}  % or try serif, structurebold, ...
  \usetheme{Antibes}
  \setbeamertemplate{navigation symbols}{}
 }
\estrue
\usepackage{todonotes}
\setbeameroption{show notes}
%
\newcommand{\gray}{\color{black!55}}
\usepackage{ulem} % sout
\usepackage{mdframed}
\usepackage{listings}
\lstset{
  aboveskip=3mm,
  belowskip=3mm,
  showstringspaces=true,
  columns=flexible,
  basicstyle={\footnotesize\ttfamily},
  breaklines=true,
  breakatwhitespace=true,
  tabsize=4,
  showlines=true,
}

\begin{document}

\color{black!85}
\large
%
% \begin{frame}[plain,noframenumbering]
%
%
% \begin{textblock}{160}(0,0)
% \includegraphics[width=1\textwidth]{../../auxiliar/static/deforestacion}
% \end{textblock}
%
% \begin{textblock}{80}(18,9)
% \textcolor{black!15}{\fontsize{44}{55}\selectfont Verdades}
% \end{textblock}
%
% \begin{textblock}{47}(85,70)
% \centering \textcolor{black!15}{{\fontsize{52}{65}\selectfont Empíricas}}
% \end{textblock}
%
% \begin{textblock}{80}(100,28)
% \LARGE  \textcolor{black!15}{\rotatebox[origin=tr]{-3}{\scalebox{9}{\scalebox{1}[-1]{$p$}}}}
% \end{textblock}
%
% \begin{textblock}{80}(66,43)
% \LARGE  \textcolor{black!15}{\scalebox{6}{$=$}}
% \end{textblock}
%
% \begin{textblock}{80}(36,29)
% \LARGE  \textcolor{black!15}{\scalebox{9}{$p$}}
% \end{textblock}
%
% %
% %
% % \begin{textblock}{160}(01,81)
% % \footnotesize \textcolor{black!5}{\textbf{\small Seminario ``Acuerdos intersubjetivos''\\
% % Comunidad Bayesiana Plurinacional} \\}
% % \end{textblock}
%
% \end{frame}

%%%%%%%%%%%%%%%%%%%%%%%%%%%%%%%%%%%%%%%%%


\begin{frame}[plain,noframenumbering]
\begin{textblock}{160}(0,-4.3) \centering
\includegraphics[width=1\textwidth]{../../auxiliar/static/antartic}
\end{textblock}

\begin{textblock}{160}(0,0) \centering
\tikz{
\node[det, fill=black,draw=black] (k) {\textcolor{black}{--------------------------------------------------------------------------------------------------------------------------------------}} ;
}
\end{textblock}

\begin{textblock}{160}(5,0)
\tikz{
\node[det, fill=black,draw=black,text width=0.01cm] (k) {\textcolor{black}{--------------------------------------------------------------------------------------------------------------------------------------}} ;
}
\end{textblock}


\begin{textblock}{160}(0,4) \centering
\LARGE \hspace{1cm} \textcolor{black!20}{\fontsize{22}{0}\selectfont \textbf{Modelos de historia \\ \hspace{1cm} completa}}
\end{textblock}


\begin{textblock}{55}[0,1](8,70)
\begin{turn}{90}
\parbox{6cm}{\footnotesize
\textcolor{black!10}{Millones de km$^2$ de hielo Antártico}}
\end{turn}
\end{textblock}


\begin{textblock}{160}(20,63)
\textcolor{black!10}{Unidad 4 \\ \small
Redes bayesianas de historia completa. \\
El problema de usar el posterior como prior del siguiente evento\\
El algoritmo de inferencia por loopy belief propagation. \\
Consideraciones de inferencia causal en series temporales. \\
}
\end{textblock}

\end{frame}


\begin{frame}[plain]
\begin{textblock}{160}(0,4)
\centering \LARGE Series de tiempo \\
\large Creencias adaptativas
\end{textblock}


\only<1-4>{
\begin{textblock}{160}(0,26) \centering

\Large La función de costo epistémica

\large
\begin{equation*}
\underbrace{P(\text{Hipótesis},\text{\En{Data}\Es{Datos}})}_{\hfrac{\text{\footnotesize\En{Initial belief compatible}\Es{Creencia compatible }}}{\text{\footnotesize \En{with the data}\Es{con los datos}}}} = \underbrace{P(\text{Hipótesis})}_{\hfrac{\text{\footnotesize\En{Initial intersubjective}\Es{Acuerdo intersubjetivo}}}{\text{\footnotesize\En{agreement}\Es{inicial}}}} \underbrace{P(\text{dato}_1 |\text{Hipótesis})}_{\text{\footnotesize Predic\En{tion}\Es{ción} 1}} \, \underbrace{P(\text{dato}_2 | \text{dato}_1 , \text{Hipótesis})}_{\text{\footnotesize Predic\En{tion}\Es{ción} 2}} \dots
\end{equation*}

\vspace{0.8cm}


\only<2>{
\Large Un único 0 en la secuencia de predicciones

hace falsa la hipótesis para siempre.\\
}\only<3-4>{
\Large Esa persona no está apta para realizar esa tarea.

\only<4>{\textbf{¿Para siempre?!}}
}

\end{textblock}
}
\only<5>{ \centering
\begin{textblock}{160}(0,-93)
\includegraphics[width=0.9\textwidth]{../../auxiliar/static/lifeHistory.jpeg}
\end{textblock}
}


\end{frame}


\begin{frame}[plain]
\begin{textblock}{160}(0,4)
\centering \LARGE Estimación de habilidad \\
\large en la industria del videojuego
\end{textblock}


 \only<1->{
 \begin{textblock}{140}(3,24)
 \normalsize
\tikz{
    \node[det, fill=black!10] (r) {$r$} ;
    \node[const, right=of r] (dr) {\normalsize $ P(r|d) = \mathbb{I}(r = d > 0)$};

    \node[latent, above=of r, yshift=-0.45cm] (d) {$d$} ; %
    \node[const, right=of d] (dd) {\normalsize $ p(d|p_a,p_b) = \delta(d = p_a-p_b)$};

    \node[latent, above=of d, xshift=-0.8cm, yshift=-0.45cm] (p1) {$p_a$} ; %
    \node[latent, above=of d, xshift=0.8cm, yshift=-0.45cm] (p2) {$p_b$} ; %


    \node[latent, above=of p1, yshift=-0.35cm] (s1) {$s_a$} ; %
    \node[latent, above=of p2, yshift=-0.35cm] (s2) {$s_b$} ; %
    \node[const, right=of s2] (ds2) {$p(s_i) = \N(s_i|\mu_i,\sigma_i^2)$};

    \node[const, right=of p2] (dp2) {\normalsize $p(p_i|s_i) = \N(p_i|s_i,\beta^2)$};

    \node[const, above=of dr] (r_name) {\small Resultado};
    \node[const, above=of dd] (d_name) {\small Diferencia};
    \node[const, above=of dp2] (p_name) {\small Desempeño};

    \node[const, above=of ds2, yshift=0.1cm] (s_name) {\small Habilidad};

    \edge {d} {r};
    \edge {p1,p2} {d};
    \edge {s1} {p1};
    \edge {s2} {p2};

}
\end{textblock}
}


\only<2>{
\begin{textblock}{100}(60,22) \centering
Prior \ $\rightarrow$ \ Posterior

\includegraphics[width=0.85\textwidth,page=4]{../3-dato/figuras/posterior_win.pdf}
\end{textblock}
}


\only<3->{
\begin{textblock}{100}(60,24) \centering
¿Cómo estimamos una habilidad en el tiempo?

\Large \vspace{0.8cm}

\only<4->{
¿Si usamos el último posterior como

prior del siguiente evento?

\vspace{0.6cm} \large

Posterior$_t$ $\rightarrow$ Prior$_{t+1}$
}

\only<5->{
\begin{equation*}
\underbrace{\N(\text{Habilidad}_{t+1} \, | \, \text{Habilidad}_{t} , \gamma^2)}_{\text{Prior}_{t+1}}
\end{equation*}
}
\end{textblock}
}
\end{frame}


\begin{frame}[plain]
\begin{textblock}{160}(20,4)
\centering \LARGE Estimación de habilidad \\
\large en la industria del videojuego
\end{textblock}

\only<1->{
 \begin{textblock}{140}(2,6)
 \normalsize
\tikz{
    \node[det, fill=black!10] (r) {$r$} ;
    \node[factor, above=of r, yshift=-0.4cm] (fr) {};
    \node[const, left=of fr] (dr) {\normalsize $\mathbb{I}(r = d > 0)$};

    \node[latent, above=of fr, yshift=-0.55cm] (d) {$d$} ; %
    \node[factor, above=of d, yshift=-0.4cm] (fd) {};

    \node[const, left=of fd] (dd) {\normalsize $\delta(d = p_a-p_b)$};

    \node[latent, above=of fd, xshift=-0.8cm, yshift=-0.55cm] (p1) {$p_a$} ; %
    \node[latent, above=of fd, xshift=0.8cm, yshift=-0.55cm] (p2) {$p_b$} ; %

    \node[factor, above=of p1, yshift=-0.4cm] (fp1) {};
    \node[factor, above=of p2, yshift=-0.4cm] (fp2) {};

    \node[latent, above=of fp1, yshift=-0.55cm] (s1) {$s_a$} ; %
    \node[latent, above=of fp2, yshift=-0.55cm] (s2) {$s_b$} ; %

    \node[factor, above=of s1, yshift=-0.4cm] (fs1) {};
    \node[factor, above=of s2, yshift=-0.4cm] (fs2) {};


    \node[const, left=of fs1] (ds2) {$\N(s_i|\mu_i,\sigma_i^2)$};

    \node[const, left=of fp1] (dp2) {\normalsize $\N(p_i|s_i,\beta^2)$};

    \node[const, left=of r] (r_name) {\small Resultado};
    \node[const, left=of d] (d_name) {\small Diferencia};
    \node[const, left=of p1] (p_name) {\small Desempeño};

    \node[const, left=of s1, yshift=0.1cm] (s_name) {\small Habilidad \ };

    \edge {d} {r};
    \edge[-] {p1,p2} {fd};
    \edge {fd} {d};
    \edge[-] {s1} {fp1};
    \edge[-] {s2} {fp2};
    \edge {fp1} {p1};
    \edge {fp2} {p2};
    \edge {fs1} {s1};
    \edge {fs2} {s2};

    \onslide<2->{
      \path[draw, ->, fill=black!50] (fs2) edge[bend left,draw=black!50] node[right,color=black!75] {\large\only<2>{$\N(s_b|\mu_b,\sigma^2)$}} (s2);
    }
    \onslide<3->{
      \path[draw, ->, fill=black!50] (s2) edge[bend left,draw=black!50] node[right,color=black!75] {\large\only<3>{$\N(s_b|\mu_b,\sigma^2)$}} (fp2);
    }
    \onslide<4->{
      \path[draw, ->, fill=black!50] (fp2) edge[bend left,draw=black!50] node[right,color=black!75] {\large$\only<4-5>{\int \N(p_b|s_b,\beta^2) \N(s_b|\mu_b,\sigma^2) \, ds_b }\only<5>{= p(p_b)}$} (p2);
    }
}
\end{textblock}
}



\only<6->{
\begin{textblock}{120}(65,18)
  \begin{flalign*}
   p(p_b) &= \int \N(p_b|s_b,\beta^2) \N(s_b|\mu_b,\sigma_b^2) ds_b \\
    \only<9>{& \overset{*}{=} \int \N(p_a|\mu_a,\beta^2 + \sigma_a^2) \, \N(s_a | \mu_*,\sigma_{*}^2)ds_a}
    \only<10>{& \overset{*}{=} \int \underbrace{\N(p_a|\mu_a,\beta^2 + \sigma_a^2)}_{\text{const.}} \underbrace{\N(s_a | \mu_*,\sigma_{*}^2)ds_a}_{1}}
    \only<11>{& \overset{*}{=} \phantom{\int} \, \N(p_a|\mu_a,\beta^2 + \sigma_a^2) \phantom{\N(s_a | \mu_*,\sigma_{*}^2)ds_a}}
    &&
  \end{flalign*}
\end{textblock}
}


\only<7-8>{
\begin{textblock}{120}(40,34)
  \begin{figure}[H]
     \centering
     \onslide<7->{
     \begin{subfigure}[b]{0.4\textwidth}
       \includegraphics[page=1,width=1\textwidth]{figuras/paso_1_multiplicacion_normales_image}
     \end{subfigure}}
     \onslide<8->{
     \begin{subfigure}[b]{0.4\textwidth}
       \includegraphics[page=1,width=\textwidth]{figuras/paso_1_multiplicacion_normales_p}
     \end{subfigure}}
  \end{figure}
\end{textblock}
}


\end{frame}



\begin{frame}[plain,noframenumbering]
\centering \vspace{0.5cm}
\includegraphics[width=1\textwidth]{../../auxiliar/static/BP.png}
\end{frame}





%
% \begin{frame}[plain]
% \begin{textblock}{96}(0,6.5)\centering
% {\transparent{0.9}\includegraphics[width=0.8\textwidth]{../../auxiliar/static/inti.png}}
% \end{textblock}
%
% \begin{textblock}{160}(96,5.5)
% \includegraphics[width=0.35\textwidth]{../../auxiliar/static/pachacuteckoricancha}
% \end{textblock}
% \end{frame}





\end{document}



