\newif\ifen
\newif\ifes
\newif\iffr
\newcommand{\fr}[1]{\iffr#1 \fi}
\newcommand{\En}[1]{\ifen#1\fi}
\newcommand{\Es}[1]{\ifes#1\fi}
\estrue
\documentclass[shownotes,aspectratio=169]{beamer}

\usepackage{siunitx}
\usepackage{transparent}


\usepackage{ragged2e} %\justifying
\usepackage{paracol}
\usepackage[utf8]{inputenc} %Para acentos en UTF8 (Prueba: á é í ó ú Á É Í Ó Ú ñ Ñ)
\usepackage{url}
%\usepackage{mathtools}
\usepackage{graphicx}
\usepackage{caption}
\usepackage{float} % para que los gr\'aficos se queden en su lugar con [H]
\usepackage[fleqn]{mathtools} % \coloneqq, flalign
\usepackage{subcaption}
\usepackage{wrapfig}
\usepackage{soul,color} %\st{Hellow world}
\usepackage{xcolor} %\st{Hellow world}
\usepackage[fleqn]{amsmath} %para escribir funci\'on partida
\usepackage{blkarray}
\usepackage{hyperref} % para inlcuir links dentro del texto
\usepackage{tabu} 
\usepackage{comment}
\usepackage{amsfonts} % mathbb{N} -> conjunto de los n\'umeros naturales  
\usepackage{enumerate}
\usepackage{listings}
\usepackage[shortlabels]{enumitem} %  shortlabels option to have compatibility with the enumerate-like scheme for label
\usepackage{framed}
\usepackage{mdframed}
\usepackage{multicol}
\usepackage{transparent} % \transparent{1.0}
\usepackage{bm} 
\usepackage[makeroom]{cancel} % \cancel{} \bcancel{} etc
\usepackage[absolute,overlay]{textpos} %no funciona
\setlength{\TPHorizModule}{1mm} %128mm  mitad: 64 
\setlength{\TPVertModule}{1mm}	%96mm  mitad 48

\newif\ifen
\newif\ifes
\newcommand{\en}[1]{\ifen#1\fi}
\newcommand{\es}[1]{\ifes#1\fi}
\estrue


\usepackage{todonotes}
\setbeameroption{show notes}
\usepackage{rotating}
\usepackage{transparent}


\newcommand{\E}{\en{S}\es{E}}
\newcommand{\A}{\en{E}\es{A}}
\newcommand{\Ee}{\en{s}\es{e}}
\newcommand{\Aa}{\en{e}\es{a}}

\hypersetup{
    colorlinks=true,
    linkcolor={red!50!black},
    citecolor={blue!35!black},
    urlcolor={blue!35!black}
}

\newcommand\hfrac[2]{\genfrac{}{}{0pt}{}{#1}{#2}} %\frac{}{} sin la linea del medio

\newcommand{\indep}{\perp \!\!\! \perp}
\newcommand{\N}{\mathcal{N}}
\newcommand{\vm}[1]{\mathbf{#1}}

\newtheorem{midef}{Definition}
\newtheorem{miteo}{Theorem}
\newtheorem{mipropo}{Proposition}

\usefonttheme[onlymath]{serif}


\usepackage{tikz} % Para graficar, por ejemplo bayes networks
%\usetikzlibrary{bayesnet} % Para que ande se necesita copiar el archivo  tikzlibrarybayesnet.code.tex en la misma carpeta

%%%%%%%%%%%%%%%%%%%%%%%%%%%%%%%%%5
%
% Incompatibles con textpos
%
%\usepackage{todonotes}
%\usepackage{tikz} % Para graficar, por ejemplo bayes networks
%
%%%%%%%%%%%%%%%%%%%%%%%%%%%%%%%%%%



\usepackage[absolute,overlay]{textpos} %no funciona
\setlength{\TPHorizModule}{1mm} %128mm  mitad: 64 
\setlength{\TPVertModule}{1mm}	%96mm  mitad 48
% 
% 
\captionsetup[figure]{labelformat=empty}

% 
% http://latexcolor.com/
\definecolor{lightseagreen}{rgb}{0.13, 0.7, 0.6.5}
\definecolor{greenblue}{rgb}{0.1, 0.55, 0.5}
\definecolor{redgreen}{rgb}{0.6, 0.4, 0.}
\definecolor{greenred}{rgb}{0.4, 0.7, 0.}
\definecolor{redblue}{rgb}{0.4, 0., .4}
\definecolor{tangelo}{rgb}{0.98, 0.3, 0.0}
\definecolor{git}{rgb}{0.94, 0.309, 0.2}
% 
\setbeamercolor{structure}{fg=greenblue}


%http://latexcolor.com/
\definecolor{azul}{rgb}{0.36, 0.54, 0.66}
\definecolor{rojo}{rgb}{0.7, 0.2, 0.116}
\definecolor{rojopiso}{rgb}{0.8, 0.25, 0.17}
\definecolor{verdeingles}{rgb}{0.12, 0.5, 0.17}
\definecolor{ubuntu}{rgb}{0.44, 0.16, 0.39}
\definecolor{debian}{rgb}{0.84, 0.04, 0.33}
\definecolor{dkgreen}{rgb}{0,0.6,0}
\definecolor{gray}{rgb}{0.5,0.5,0.5}
\definecolor{mauve}{rgb}{0.58,0,0.82}




\newcommand\Wider[2][3em]{%
\makebox[\linewidth][c]{%
  \begin{minipage}{\dimexpr\textwidth+#1\relax}
  \raggedright#2
  \end{minipage}%
  }%
}

\newenvironment{ejercicio}[1]{
% \setbeamercolor{block title}{bg=tangelo, fg=white}
\begin{exampleblock}{#1}
}{
\end{exampleblock}
}

\newenvironment{resumen}[1]{
\setbeamercolor{block title}{bg=git, fg=white}
\begin{block}{#1}
}{
\end{block}
}

\newenvironment{comando}{
\setbeamercolor{block body}{bg=git, fg=white}
\begin{block}{}
\begin{center}
\LARGE
\begin{texttt}
}{
\end{texttt}
\end{center}
\end{block}
}



% tikzlibrary.code.tex
%
% Copyright 2010-2011 by Laura Dietz
% Copyright 2012 by Jaakko Luttinen
%
% This file may be distributed and/or modified
%
% 1. under the LaTeX Project Public License and/or
% 2. under the GNU General Public License.
%
% See the files LICENSE_LPPL and LICENSE_GPL for more details.

% Load other libraries

%\newcommand{\vast}{\bBigg@{2.5}}
% newcommand{\Vast}{\bBigg@{14.5}}
% \usepackage{helvet}
% \renewcommand{\familydefault}{\sfdefault}

\usetikzlibrary{shapes}
\usetikzlibrary{fit}
\usetikzlibrary{chains}
\usetikzlibrary{arrows}

% Latent node
\tikzstyle{latent} = [circle,fill=white,draw=black,inner sep=1pt,
minimum size=20pt, font=\fontsize{10}{10}\selectfont, node distance=1]
% Observed node
\tikzstyle{obs} = [latent,fill=gray!25]
% Invisible node
\tikzstyle{invisible} = [latent,minimum size=0pt,color=white, opacity=0, node distance=0]
% Constant node
\tikzstyle{const} = [rectangle, inner sep=0pt, node distance=0.1]
%state
\tikzstyle{estado} = [latent,minimum size=8pt,node distance=0.4]
%action
\tikzstyle{accion} =[latent,circle,minimum size=5pt,fill=black,node distance=0.4]


% Factor node
\tikzstyle{factor} = [rectangle, fill=black,minimum size=10pt, draw=black, inner
sep=0pt, node distance=1]
% Deterministic node
\tikzstyle{det} = [latent, rectangle]

% Plate node
\tikzstyle{plate} = [draw, rectangle, rounded corners, fit=#1]
% Invisible wrapper node
\tikzstyle{wrap} = [inner sep=0pt, fit=#1]
% Gate
\tikzstyle{gate} = [draw, rectangle, dashed, fit=#1]

% Caption node
\tikzstyle{caption} = [font=\footnotesize, node distance=0] %
\tikzstyle{plate caption} = [caption, node distance=0, inner sep=0pt,
below left=5pt and 0pt of #1.south east] %
\tikzstyle{factor caption} = [caption] %
\tikzstyle{every label} += [caption] %

\tikzset{>={triangle 45}}

%\pgfdeclarelayer{b}
%\pgfdeclarelayer{f}
%\pgfsetlayers{b,main,f}

% \factoredge [options] {inputs} {factors} {outputs}
\newcommand{\factoredge}[4][]{ %
  % Connect all nodes #2 to all nodes #4 via all factors #3.
  \foreach \f in {#3} { %
    \foreach \x in {#2} { %
      \path (\x) edge[-,#1] (\f) ; %
      %\draw[-,#1] (\x) edge[-] (\f) ; %
    } ;
    \foreach \y in {#4} { %
      \path (\f) edge[->,#1] (\y) ; %
      %\draw[->,#1] (\f) -- (\y) ; %
    } ;
  } ;
}

% \edge [options] {inputs} {outputs}
\newcommand{\edge}[3][]{ %
  % Connect all nodes #2 to all nodes #3.
  \foreach \x in {#2} { %
    \foreach \y in {#3} { %
      \path (\x) edge [->,#1] (\y) ;%
      %\draw[->,#1] (\x) -- (\y) ;%
    } ;
  } ;
}

% \factor [options] {name} {caption} {inputs} {outputs}
\newcommand{\factor}[5][]{ %
  % Draw the factor node. Use alias to allow empty names.
  \node[factor, label={[name=#2-caption]#3}, name=#2, #1,
  alias=#2-alias] {} ; %
  % Connect all inputs to outputs via this factor
  \factoredge {#4} {#2-alias} {#5} ; %
}

% \plate [options] {name} {fitlist} {caption}
\newcommand{\plate}[4][]{ %
  \node[wrap=#3] (#2-wrap) {}; %
  \node[plate caption=#2-wrap] (#2-caption) {#4}; %
  \node[plate=(#2-wrap)(#2-caption), #1] (#2) {}; %
}

% \gate [options] {name} {fitlist} {inputs}
\newcommand{\gate}[4][]{ %
  \node[gate=#3, name=#2, #1, alias=#2-alias] {}; %
  \foreach \x in {#4} { %
    \draw [-*,thick] (\x) -- (#2-alias); %
  } ;%
}

% \vgate {name} {fitlist-left} {caption-left} {fitlist-right}
% {caption-right} {inputs}
\newcommand{\vgate}[6]{ %
  % Wrap the left and right parts
  \node[wrap=#2] (#1-left) {}; %
  \node[wrap=#4] (#1-right) {}; %
  % Draw the gate
  \node[gate=(#1-left)(#1-right)] (#1) {}; %
  % Add captions
  \node[caption, below left=of #1.north ] (#1-left-caption)
  {#3}; %
  \node[caption, below right=of #1.north ] (#1-right-caption)
  {#5}; %
  % Draw middle separation
  \draw [-, dashed] (#1.north) -- (#1.south); %
  % Draw inputs
  \foreach \x in {#6} { %
    \draw [-*,thick] (\x) -- (#1); %
  } ;%
}

% \hgate {name} {fitlist-top} {caption-top} {fitlist-bottom}
% {caption-bottom} {inputs}
\newcommand{\hgate}[6]{ %
  % Wrap the left and right parts
  \node[wrap=#2] (#1-top) {}; %
  \node[wrap=#4] (#1-bottom) {}; %
  % Draw the gate
  \node[gate=(#1-top)(#1-bottom)] (#1) {}; %
  % Add captions
  \node[caption, above right=of #1.west ] (#1-top-caption)
  {#3}; %
  \node[caption, below right=of #1.west ] (#1-bottom-caption)
  {#5}; %
  % Draw middle separation
  \draw [-, dashed] (#1.west) -- (#1.east); %
  % Draw inputs
  \foreach \x in {#6} { %
    \draw [-*,thick] (\x) -- (#1); %
  } ;%
}



\newcommand{\Arrow}[1]{%
\parbox{#1}{\tikz{\draw[->](0,0)--(#1,0);}}
}

\mode<presentation>
 {
 %   \usetheme{Madrid}      % or try Darmstadt, Madrid, Warsaw, ...
 %   \usecolortheme{default} % or try albatross, beaver, crane, ...
 %   \usefonttheme{serif}  % or try serif, structurebold, ...
  \usetheme{Antibes}
  \setbeamertemplate{navigation symbols}{}
 }
\estrue
\usepackage{todonotes}
\setbeameroption{show notes}
%
\newcommand{\gray}{\color{black!55}}
\usepackage{ulem} % sout
\usepackage{mdframed}
\usepackage{listings}
\lstset{
  aboveskip=3mm,
  belowskip=3mm,
  showstringspaces=true,
  columns=flexible,
  basicstyle={\ttfamily},
  breaklines=true,
  breakatwhitespace=true,
  tabsize=4,
  showlines=true
}


\begin{document}

\color{black!85}
\large


\begin{frame}[plain,noframenumbering]

\begin{textblock}{160}(0,0)
\includegraphics[width=1\textwidth]{../../auxiliar/static/peligro_predador}
\end{textblock}

\begin{textblock}{160}(127,67)
\LARGE \textcolor{black!5}{\fontsize{22}{0}\selectfont \textbf{Inferencia  \\[-0.1cm] \hspace{0.5cm} causal}}
\end{textblock}

\begin{textblock}{55}(2,3)
\begin{turn}{0}
\parbox{15cm}{\small
\textcolor{black!95}{Flujos de inferencia en modelos causales. Efecto de}\\
\textcolor{black!95}{las intervenciones en modelos causales. Conclusiones} \\
\textcolor{black!95}{causales a partir de datos observables. Identificación} \\
\textcolor{black!95}{de modelo causal.} \\
\normalsize\textcolor{black!95}{Unidad 2} \\
}
\end{turn}
\end{textblock}

\end{frame}


\begin{frame}[plain]
\begin{textblock}{160}(0,4)
 \centering \LARGE \phantom{Causales} Modelos \only<1-8>{Gráficos}\only<9->{\sout{Gráficos}} \onslide<9->{Causales}\\
 \large o redes bayesianas
\end{textblock}
 \centering
 \vspace{0.75cm}

\begin{textblock}{70}(5,18)
\raggedleft
 \tikz{
    \node[det] (a) {$A_{_{\onslide<6->{\phantom}{i}}}$} ; %
    \node[det, below=of a] (b) {$B_{_{\onslide<6->{\phantom}{i}}}$} ; %
    \node[const, left= of a, xshift=-0.3cm, yshift=0.1cm] (pa) { \small
    \begin{tabular}{|c|c|}
          $A=0$  &  $A=1$   \\ \hline
        $0.5$ & $0.5$   \\ \hline
    \end{tabular}
    }; %
    \node[const, above= of pa] (npa) {\small$P(A)$};


    \node[const, left=of b, xshift=-0.3cm, yshift=-0.1cm] (pb) { \small
    \begin{tabular}{c|c|c|}
        &  $B=0$  &  $B=1$   \\ \hline
       $A=0$ & $0.95$ & $0.05$   \\ \hline
       $A=1$ & $0.05$ & $0.95$   \\ \hline
    \end{tabular}
    };
    \node[const, above= of pb] (npb) {\small$P(B|A)$};

    \node[invisible, above=of a, yshift=1cm] (ia) {};

    \onslide<6->{\plate {datos} {(a)(b)} {\tiny$i$: Dato};}

    \edge {a} {b};
    }
\end{textblock}
\only<4->{
\begin{textblock}{70}(85,18)
\raggedright
    \tikz{
    \node[det] (a) {$A_{_{\onslide<6->{\phantom}{i}}}$} ; %
    \node[det, below=of a] (b) {$B_{_{\onslide<6->{\phantom}{i}}}$} ; %
    \node[const, right= of a, xshift=0.3cm, yshift=0.1cm] (pa) { \small
    \begin{tabular}{c|c|c|}
        &  $A=0$  &  $A=1$   \\ \hline
       $B=0$ & $0.95$ & $0.05$   \\ \hline
       $B=1$ & $0.05$ & $0.95$   \\ \hline
    \end{tabular}
    }; %
    \node[const, above= of pa] (npa) {\small$P(A|B)$};


    \node[const, right=of b, xshift=0.3cm, yshift=-0.1cm] (pb) { \small
    \begin{tabular}{|c|c|}
          $B=0$  &  $B=1$   \\ \hline
        $0.5$ & $0.5$   \\ \hline
    \end{tabular}
    };
    \node[const, above= of pb] (npb) {\small$P(B)$};

    \node[invisible, above=of a, yshift=1cm] (ia) {};

    \onslide<6->{\plate {datos} {(a)(b)} {\tiny$i$: Dato};}

    \edge {b} {a};
    }
\end{textblock}
}

\only<2-5>{
\begin{textblock}{140}(44,66)
\begin{flalign*}
& P(A, B) = P(A) P(B|A) \only<3->{= P(B) P(A|B)} &&
\end{flalign*}
\end{textblock}
}

\only<6-7>{
\begin{textblock}{160}(0,60)
\begin{equation*}
P(\text{Modelo} | \text{Datos} = \{(a_1, b_1), (a_2, b_2), \dots \}) = \onslide<6>{?} \onslide<7>{\frac{\overbrace{P(\text{Datos}|\text{Modelo})}^{\text{Predicción}} P(\text{Modelo})}{P(\text{Datos})}}
\end{equation*}
\end{textblock}
}

\only<8->{
\begin{textblock}{160}(0,66)
\begin{equation*}
P(\text{Datos}|\text{Modelo}_{A\rightarrow B}) =  P(\text{Datos}|\text{Modelo}_{B\rightarrow A})
\end{equation*}

\only<9>{Con mismas predicciones, no podemos distinguir el modelo causal!}

\end{textblock}
}


\end{frame}


\begin{frame}[plain]
\begin{textblock}{160}(0,4)
 \centering \LARGE Inferencia causal \\
 \large Los \textbf{niveles} del razonamiento causal
\end{textblock}

 \vspace{0.75cm}

\only<1-3>{
\begin{textblock}{140}(10,24)
1. \textbf{Asociacional}:  $P(y \, | \, x , \, \text{Modelo Causal})$ \ y \ $P(\text{Modelo Causal} \, | \, x )$  \\[0.05cm] \normalsize
\hspace{0.3cm} Permite evaluar el efecto y el modelo causal sólo si se cumplen ciertas condiciones \\

\vspace{1cm} \large


\only<2-3>{
2. \textbf{Intervencional}:  $P(y \, | \, \text{do}(x), \, \text{Modelo Causal})$ \ y \ $P(\text{Modelo Causal} \, | \, y, \, \text{do}(x) )$ \\[0.05cm] \normalsize
\hspace{0.3cm} Permite evaluar tanto el efecto causal y el modelo causal
}

\vspace{0.7cm} \large

\only<3->{
3. \textbf{Contrafactual}:  $P(\overbrace{y \, | \, \text{do}(x),}^{\text{Contrafactual}} \, \overbrace{y^{\prime}, \, \text{do}(x^{\prime}),}^{\text{Factual}}\, \text{Modelo Causal})$ \\[0.05cm] \normalsize
\hspace{0.3cm} Permite evaluar el efecto causal contrafactual (no permite evaluar el modelo causal)
}
\end{textblock}
}

\only<4>{
\begin{textblock}{140}(10,36) \centering \Large
Estos niveles surgen naturalmente \\

del proceso generativo de lo datos
\end{textblock}
}

\end{frame}

\begin{frame}[plain]
\begin{textblock}{160}(0,4)
 \centering \LARGE Monty Hall Causal \\
 \large \only<1>{Los \textbf{niveles} del razonamiento causal}\only<2>{Asociación}\only<3>{Intervención}\only<4-7>{Contrafáctico}\only<8->{Los \textbf{niveles} del razonamiento causal}
\end{textblock}
 \centering
 \vspace{0.75cm}


\only<2>{
\begin{textblock}{80}(5,22)
\raggedright
%\onslide<2->{Modelo gráfico} \\ \vspace{0.3cm}
 \tikz{
    \node[det, minimum size=1.4cm] (r) {\includegraphics[width=0.12\textwidth]{../../auxiliar/static/regalo.png}} ;
    \node[const,above=of r] (pr) {\normalsize $P(r)$} ;

    \node[det, below=of r,minimum size=1.4cm] (d) {\includegraphics[width=0.10\textwidth]{../../auxiliar/static/dedo.png}} ;
    \node[const,left=of d] (pd) {\normalsize $P(s|r)$} ;

\node[invisible,left=of d, xshift=-1.5cm] () {} ;


    \edge {r} {d};
}
\end{textblock}

\begin{textblock}{80}(65,24) \centering \Large
Asociación
\begin{equation*}
P(r, s)
\end{equation*}
\large
\begin{tabular}{c|c|c|c|}
         &  $r1$  &  $r2$ & $r3$   \\ \hline
       $s1$ & $0$ & $1/6$ & $1/6$ \\ \hline
       $s2$ & $1/6$ & $0$ & $1/6$ \\ \hline
       $s3$ & $1/6$ & $1/6$ & $0$ \\ \hline
    \end{tabular}
\end{textblock}
}



\only<3>{
\begin{textblock}{80}(5,22)
\raggedright
%\onslide<2->{Modelo gráfico} \\ \vspace{0.3cm}
 \tikz{

    \node[det, minimum size=1.4cm] (d) {\includegraphics[width=0.10\textwidth]{../../auxiliar/static/dedo.png}} ;
    \node[const,left=of d] (pd) {\normalsize $P(s|r,c)$} ;

    \node[det, above=of d, minimum size=1.4cm, xshift=1.2cm] (r) {\includegraphics[width=0.12\textwidth]{../../auxiliar/static/regalo.png}} ;
    \node[const,above=of r] (pr) {\normalsize $P(r)$} ;

    \node[det, minimum size=1.4cm, fill=black!40, above=of d, xshift=-1.2cm] (c) {\includegraphics[width=0.12\textwidth]{../../auxiliar/static/cerradura.png}} ;
    \node[const,above=of c] (pc) {\normalsize $\text{do}(c) = 1$} ;


    \node[invisible,left=of d, xshift=-1.5cm] () {} ;



    \edge {r,c} {d};
}
\end{textblock}

\begin{textblock}{80}(65,24) \centering \Large
Intervención
\begin{equation*}
P(r, s| \text{do}_c = 1)
\end{equation*}
\large
\begin{tabular}{c|c|c|c|}
         &  $r1$  &  $r2$ & $r3$   \\ \hline
       $s1$ & $0$ & $0$ & $0$ \\ \hline
       $s2$ & $1/6$ & $0$ & $1/3$ \\ \hline
       $s3$ & $1/6$ & $1/3$ & $0$ \\ \hline
    \end{tabular}
\end{textblock}
}


\only<4-7>{
\begin{textblock}{80}(5,22)
\raggedright
%\onslide<2->{Modelo gráfico} \\ \vspace{0.3cm}
 \tikz{

    \only<4-5>{
      \node[det, minimum size=1.4cm, fill=black!15] (d) {\includegraphics[width=0.10\textwidth]{../../auxiliar/static/dedo.png}} ;
      \node[const,left=of d] (pd) {\normalsize $s=2$} ;

      \node[det, minimum size=1.4cm, fill=black!40, above=of d, xshift=-1.2cm] (c) {\includegraphics[width=0.12\textwidth]{../../auxiliar/static/cerradura.png}} ;
      \node[const,above=of c] (pc) {\normalsize $\text{do}(c)=1$} ;
       \edge {r,c} {d};
    }
    \only<6->{
      {\color{black!15}
      \node[det, minimum size=1.4cm, fill=black!15, opacity=0.3] (d) {\includegraphics[width=0.10\textwidth]{../../auxiliar/static/dedo.png} } ;
      \node[const,left=of d] (pd) {\normalsize $s=2$} ;


      \node[det, minimum size=1.4cm, fill=black!40, above=of d, xshift=-1.2cm, opacity=0.3] (c) {\includegraphics[width=0.12\textwidth]{../../auxiliar/static/cerradura.png}} ;
      \node[const,above=of c] (pc) {\normalsize $\text{do}(c)=1$} ;
      \edge {r,c} {d};
      }
    }

      \node[det, above=of d, minimum size=1.4cm, xshift=1.2cm] (r) {\includegraphics[width=0.12\textwidth]{../../auxiliar/static/regalo.png}} ;
      \node[const,above=of r] (pr) {\normalsize $P(r)$} ;

      \only<6->{
        \node[det,  below=of r, xshift=1.2cm, minimum size=1.4cm] (d_prima) {\includegraphics[width=0.10\textwidth]{../../auxiliar/static/dedo.png} } ;
        \node[const,right=of d_prima] (pd_prima) {\normalsize $P(s^{\prime}|r,c^{\prime})$} ;

        \node[det, minimum size=1.4cm, fill=black!40, above=of d_prima, xshift=1.2cm] (c_prima) {\includegraphics[width=0.12\textwidth]{../../auxiliar/static/cerradura.png}} ;
        \node[const,above=of c_prima] (pc_prima) {\normalsize $\text{do}(c^{\prime})=2$} ;
        \edge {r,c_prima} {d_prima};
      }

    \node[invisible,left=of d, xshift=-1.5cm] () {} ;



}
\end{textblock}
}

\only<4-5>{
\begin{textblock}{80}(65,24) \centering \Large
Factual
\begin{equation*}
P(r| \text{do}_{c} = 1, s = 2 )
\end{equation*}
\large
\begin{tabular}{|c|c|c|}
           $r1$  &  $r2$ & $r3$   \\ \hline
        $1/3$ & $0$ & $2/3$ \\ \hline
    \end{tabular}
\end{textblock}
}

\only<6-7>{
\begin{textblock}{80}(75,24) \centering \Large
Contra factual
\begin{equation*}
P(s^{\prime}, r | \text{do}_{c} = 1, s = 2 , \text{do}_{c^{\prime}}= 2)
\end{equation*}
\large
\only<7>{
\begin{tabular}{c|c|c|c|}
         &  $r1$  &  $r2$ & $r3$   \\ \hline
       $s^{\prime}1$ & $0$ & $0$ & $2/3$ \\ \hline
       $s^{\prime}2$ & $0$ & $0$ & $0$ \\ \hline
       $s^{\prime}3$ & $1/3$ & $0$ & $0$ \\ \hline
    \end{tabular}
}
\end{textblock}
}


\only<5-7>{
\begin{textblock}{140}(10,73) \centering \Large
¿Que caja hubiera señalado si hubieramos elegido la caja 2, \\
dado que elegimos la caja 1 y señaló la caja 2?
\end{textblock}
}

\only<8->{
\begin{textblock}{50}(4,28) \centering
\textbf{Asociación}
\begin{equation*}
P(r, s)
\end{equation*}
\large
\begin{tabular}{c|c|c|c|}
         &  $r1$  &  $r2$ & $r3$   \\ \hline
       $s1$ & $0$ & $1/6$ & $1/6$ \\ \hline
       $s2$ & $1/6$ & $0$ & $1/6$ \\ \hline
       $s3$ & $1/6$ & $1/6$ & $0$ \\ \hline
    \end{tabular}
\end{textblock}
}

\only<9->{
\begin{textblock}{50}(54,28) \centering
\textbf{Intervención}
\begin{equation*}
P(r, s| \text{do}_c = 1)
\end{equation*}
\large
\begin{tabular}{c|c|c|c|}
         &  $r1$  &  $r2$ & $r3$   \\ \hline
       $s1$ & $0$ & $0$ & $0$ \\ \hline
       $s2$ & $1/6$ & $0$ & $1/3$ \\ \hline
       $s3$ & $1/6$ & $1/3$ & $0$ \\ \hline
    \end{tabular}
\end{textblock}
}

\only<10->{
\begin{textblock}{50}(104,28) \centering
\textbf{Contra factual}
\begin{equation*}
P(s^{\prime}, r | \text{do}_{c^{\prime}}= 2, \text{do}_{c} = 1, s = 2) \ \
\end{equation*}
\large
\begin{tabular}{c|c|c|c|}
         &  $r1$  &  $r2$ & $r3$   \\ \hline
       $s^{\prime}1$ & $0$ & $0$ & $2/3$ \\ \hline
       $s^{\prime}2$ & $0$ & $0$ & $0$ \\ \hline
       $s^{\prime}3$ & $1/3$ & $0$ & $0$ \\ \hline
    \end{tabular}
\end{textblock}
}

\only<11>{
\begin{textblock}{160}(0,72) \centering
\textbf{Efecto causal}\\[-0.8cm]
\begin{align*}
\underbrace{P(r,s|\text{do}_c = 1, \text{Modelo Causal})}_{\text{Intervención 1}} - \underbrace{P(r,s|\text{do}_c = 2, \text{Modelo Causal})}_{\text{Intervención 2}}
\end{align*}
\end{textblock}
}

\only<12>{
\begin{textblock}{160}(0,72) \centering
\textbf{Efecto causal}\\ \Large
Suponemos que el modelo causal es correcto!

\large(está en el condicional)
\end{textblock}
}


\end{frame}


\begin{frame}[plain]
\begin{textblock}{160}(0,4)
 \centering \LARGE Identificación de modelo causal \\
\end{textblock}


\begin{textblock}{140}(10,12) \centering \Large
\begin{equation*}
P(\text{Modelo causal}\,|\, \text{Datos},\, \text{Intervenciones}) = \,  ?
\end{equation*}
\end{textblock}

\begin{textblock}{80}(5,34) \centering
 \tikz{
    \node[det] (a) {$A_{_{i}}$} ; %
    \node[det, below=of a] (b) {$B_{_{i}}$} ; %
    \node[const, left= of a, xshift=-0.3cm, yshift=0.1cm] (pa) { \small
    \begin{tabular}{|c|c|}
          $A=0$  &  $A=1$   \\ \hline
        $0.5$ & $0.5$   \\ \hline
    \end{tabular}
    }; %
    \node[const, above= of pa] (npa) {\small$P(A)$};


    \node[const, left=of b, xshift=-0.3cm, yshift=-0.1cm] (pb) { \small
    \begin{tabular}{c|c|c|}
        &  $B=0$  &  $B=1$   \\ \hline
       $A=0$ & $0.95$ & $0.05$   \\ \hline
       $A=1$ & $0.05$ & $0.95$   \\ \hline
    \end{tabular}
    };
    \node[const, above= of pb] (npb) {\small$P(B|A)$};

    \node[invisible, above=of a, yshift=1cm] (ia) {};

    \plate {datos} {(a)(b)} {\tiny$i$: Dato};

    \edge {a} {b};
    }
\end{textblock}
\begin{textblock}{80}(75,34) \centering
    \tikz{
    \node[det] (a) {$A_{_{i}}$} ; %
    \node[det, below=of a] (b) {$B_{_{i}}$} ; %
    \node[const, right= of a, xshift=0.3cm, yshift=0.1cm] (pa) { \small
    \begin{tabular}{c|c|c|}
        &  $A=0$  &  $A=1$   \\ \hline
       $B=0$ & $0.95$ & $0.05$   \\ \hline
       $B=1$ & $0.05$ & $0.95$   \\ \hline
    \end{tabular}
    }; %
    \node[const, above= of pa] (npa) {\small$P(A|B)$};


    \node[const, right=of b, xshift=0.3cm, yshift=-0.1cm] (pb) { \small
    \begin{tabular}{|c|c|}
          $B=0$  &  $B=1$   \\ \hline
        $0.5$ & $0.5$   \\ \hline
    \end{tabular}
    };
    \node[const, above= of pb] (npb) {\small$P(B)$};

    \node[invisible, above=of a, yshift=1cm] (ia) {};

    \plate {datos} {(a)(b)} {\tiny$i$: Dato};

    \edge {b} {a};
    }
\end{textblock}




\end{frame}


\begin{frame}[plain]
\only<1->{
\begin{textblock}{160}(0,4)
 \centering \LARGE Factor Graph \\
 \large \only<1-6>{Notación extendida para de los modelos gráficos}\only<7->{\textbf{Modelos dinámicos: independencias específicas al contexto}}
\end{textblock}
}


\only<1>{
\begin{textblock}{80}(5,34)
\raggedleft
 \tikz{
    \node[det] (a) {$A$} ; %
    \node[det, below=of a] (b) {$B$} ; %
    \node[const, left= of a, xshift=-0.3cm, yshift=0.1cm] (pa) { \small
    \begin{tabular}{|c|c|}
          $A=0$  &  $A=1$   \\ \hline
        $0.5$ & $0.5$   \\ \hline
    \end{tabular}
    }; %
    \node[const, above= of pa] (npa) {\small$P(A)$};


    \node[const, left=of b, xshift=-0.3cm, yshift=-0.1cm] (pb) { \small
    \begin{tabular}{c|c|c|}
        &  $B=0$  &  $B=1$   \\ \hline
       $A=0$ & $0.95$ & $0.05$   \\ \hline
       $A=1$ & $0.05$ & $0.95$   \\ \hline
    \end{tabular}
    };
    \node[const, above= of pb] (npb) {\small$P(B|A)$};

    \node[invisible, right=of a, xshift=1.5cm] (ia) {};

    \edge {a} {b};
    }
\end{textblock}
}

\only<2-4>{
\begin{textblock}{80}(5,18)
\raggedleft
 \tikz{

    \node[factor] (fa) {} ; %
    \node[det, below=of fa, yshift=0.3cm] (a) {$A$} ; %
    \node[factor, below=of a, yshift=0.3cm] (fb) {} ; %
    \node[det, below=of fb, yshift=0.3cm] (b) {$B$} ; %


    \node[const, right= of fa, xshift=0cm] (npa) {\small\only<2-3>{$P(A)$}\only<4->{$P(A|\text{do}_A)$}};
    \node[const, left= of fa, xshift=0cm] (pa) { \small
      \only<2-3>{
      \begin{tabular}{|c|c|}
            $A=0$  &  $A=1$   \\ \hline
          $0.5$ & $0.5$   \\ \hline
      \end{tabular}
      }
      \only<4->{
      \begin{tabular}{c|c|c|}
            & $A=0$  &  $A=1$   \\ \hline
         {\scriptsize \text{do}$_A = 0$ } & $0.5$ & $0.5$   \\ \hline
         {\scriptsize \text{do}$_A = 1$ } & $1-\alpha$ & $\alpha$   \\ \hline
      \end{tabular}
      }
    }; %

    \onslide<4->{

      %\node[factor, right=of fa, xshift=-0.7cm] (f2a) {} ;
      %\gate {if} {(fa)(f2a)} {};
      \node[det, above=of fa, yshift=-0.3cm] (doA) {do$_A$} ;
      %\gate {if_Trata} {(f2a)} {};
      %\gate {if_noTrata} {(f2a)} {};

      \node[factor, above=of doA, yshift=-0.3cm] (fdoA) {} ;

      \node[const, right= of fdoA] (npdoA) {\small$P(\text{do}_A)$};
      \node[const, left= of fdoA, xshift=0cm] (pdoA) { \small
        \begin{tabular}{|c|c|}
            do$_A=0$  &  do$_A=1$   \\ \hline
            $1-\delta_{A}$ & $\delta_A$   \\ \hline
        \end{tabular}
    }; %

    }


    \node[const, right= of fb] (npb) {\small$P(B|A)$};
    \node[const, left=of fb, xshift=-0.3cm] (pb) { \small
    \begin{tabular}{c|c|c|}
        &  $B=0$  &  $B=1$   \\ \hline
       $A=0$ & $0.95$ & $0.05$   \\ \hline
       $A=1$ & $0.05$ & $0.95$   \\ \hline
    \end{tabular}
    };

    \node[invisible, right=of a, xshift=1.5cm] (ia) {};


    \edge {fa} {a};
    \onslide<4->{
      \edge {fdoA} {doA};
      \edge {doA} {fa};
    }
    \edge[-] {a} {fb};
    \edge {fb} {b};

    }
\end{textblock}
}

\only<1-2>{
\begin{textblock}{70}(80,42) \centering \Large
\begin{equation*}
 P(A, B | \text{Modelo}_{A \rightarrow B})
\end{equation*}
\end{textblock}
}
\only<3>{
\begin{textblock}{70}(80,42) \centering
\textbf{Nodos}: Variables y Funciones \\[0.6cm]

\textbf{Ejes}: Variable $v$ es parámetro

de la función $f$
\end{textblock}
}
\only<4>{
\begin{textblock}{70}(80,42) \centering \Large
\begin{equation*}
 P(A, B, \text{do}_A | \text{Modelo}_{A \rightarrow B })
\end{equation*}
\end{textblock}
}

\only<5>{
\begin{textblock}{70}(0,42) \centering \Large
\begin{equation*}
 P(A, B, \text{do}_A | \text{Modelo}_{B \rightarrow A })
\end{equation*}
\end{textblock}
}

\begin{textblock}{160}(15,74)
\begin{flalign*}
 \only<8>{& P(A, B, \text{do}_A | \text{Modelo}_{B \rightarrow A}) }
 \only<8>{= P(B) \, P_0(A|B)^{1-\text{do}_A} \, P_1(A)^{\text{do}_A} \, P(\text{do}_A)   \\}
 \only<9>{& P(A, B | \underbrace{\text{do}_A = 1, \text{Modelo}_{B \rightarrow A}}_{\text{Intervención}})= P(B) \, P_1(A)    \\}
 \only<10>{& P(A, B | \underbrace{\text{do}_A = 0, \text{Modelo}_{B \rightarrow A}}_{\text{Sin intervención}})= P(B) \, P_0(A|B)    \\}
 &&
\end{flalign*}
\end{textblock}





\only<5->{
\begin{textblock}{85}(25,18)
\raggedright
 \tikz{

    \node[det] (a) {$A$} ; %
    \only<-8,10->{\node[factor, below=of a] (fa) {} ; }
    \only<9>{\node[factor, below=of a, fill=black!7, draw=black!7] (fa) {} ; }
    \node[det, below=of fa] (b) {$B$} ; %
    \node[factor, below=of b] (fb) {} ; %

    \onslide<6->{
      \only<-8>{\node[det, left=of b] (doA) {do$_A$} ;}
      \only<9->{\node[det, left=of b, fill=black!20] (doA) {do$_A$} ;}
      \only<9>{\node[const, left=of doA] (odoA) {\small do$_A$=1} ;}
      \only<10>{\node[const, left=of doA] (odoA) {\small do$_A$=0} ;}
      \node[factor, below=of doA] (fdoA) {} ; %
      \node[const, left= of fdoA] (ndoA) {\small$P(\text{do}_A)$};
      \node[const, left=of ndoA] (pdoA) { \small
          \begin{tabular}{|c|c|}
                do$_A=0$  &  do$_A=1$   \\ \hline
              $1-\delta_A$ & $\delta_A$   \\ \hline
          \end{tabular}
        };
    }
    \onslide<7->{
      \only<-9>{\node[factor, left=of fa, xshift=0.725cm] (f2a) {} ;}
      \only<10->{\node[factor, left=of fa, draw=black!7, fill=black!7, xshift=0.725cm] (f2a) {} ;}
      \gate {ifA} {(fa)(f2a)} {};
      {\only<10>{\color{black!7}} \gate {ifA1} {(f2a)} {};}
      {\only<9>{\color{black!7}} \gate {ifA0} {(fa)} {};}
    }

    \only<-6>{\node[const, left=of fa] (npa) {\small \only<5>{$P(A|B)$} \only<6>{$P(A|B, \text{do}_A)$}};}
    \node[const, right= of fa, xshift=0.3cm] (pa) { \small
      \only<5>{
        \begin{tabular}{c|c|c|}
            &  $A=0$  &  $A=1$   \\ \hline
          $B=0$ & $0.95$ & $0.05$   \\ \hline
          $B=1$ & $0.05$ & $0.95$   \\ \hline
        \end{tabular}
      }
      \only<6>{
        \begin{tabular}{c|c|c|}
            &  $A=0$  &  $A=1$   \\ \hline
          {\scriptsize $B=0$ do$_A = 0$} & $0.95$ & $0.05$   \\ \hline
          {\scriptsize $B=1$ do$_A = 0$} & $0.05$ & $0.95$   \\ \hline
          {\scriptsize do$_A = 1$} & $1-\alpha$ & $\alpha$   \\ \hline
        \end{tabular}
      }
    }; %
    \only<7->{
      {\only<9>{\color{black!7}}\node[const, right=of ifA] (npa) {\small $P_0(A|B)$};}
      {\only<10>{\color{black!7}} \node[const, left=of ifA] (np2a) {\small $P_1(A)$}; }
      {\only<9>{\color{black!7}}\node[const, right= of npa] (pa) { \small
        \begin{tabular}{c|c|c|}
            &  $A=0$  &  $A=1$   \\ \hline
          $B=0$ & $0.95$ & $0.05$   \\ \hline
          $B=1$ & $0.05$ & $0.95$   \\ \hline
        \end{tabular}
      };}
      {\only<10>{\color{black!7}} \node[const, left= of np2a] (p2a) { \small
        \begin{tabular}{|c|c|}
              $A=0$  &  $A=1$   \\ \hline
           $1-\alpha$ & $\alpha$   \\ \hline
        \end{tabular}
      };}
    }


    \node[const, right= of fb] (npb) {\small$P(B)$};
    \node[const, right=of npb, xshift=0.3cm] (pb) { \small
    \begin{tabular}{|c|c|}
          $B=0$  &  $B=1$   \\ \hline
        $0.5$ & $0.5$   \\ \hline
    \end{tabular}
    };

%     \node[invisible, above=of a, yshift=0.2cm] (ia) {};
%     \node[invisible, left=of a, xshift=0.2cm] (ia) {};


    {\only<9>{\color{black!7}} \edge {fa} {a};}
    \edge {fb} {b};
    {\only<9>{\color{black!7}} \edge[-] {b} {fa}; }
    \onslide<6->{
      \edge {fdoA} {doA};
    }
    \onslide<6>{
      \edge[-] {doA} {fa};
    }
    \onslide<7->{
      \edge[-,dashed] {doA} {ifA0};
      {\only<10>{\color{black!7}} \edge {f2a} {a}; }
    }
    }
\end{textblock}
}

\end{frame}

\begin{frame}[plain]
\begin{textblock}{160}(0,4)
 \centering \LARGE Identificación de modelo causal\\
 \large A través de intervenciones do$(\cdot)$
 \end{textblock}
 \vspace{0.75cm}

\begin{textblock}{140}(3,24)
Datos:

\vspace{0.3cm}
\normalsize
\begin{tabular}{c|c|c|c|}
    $i$ & do$_{Ai}$ &  $A_i$  &  $B_i$   \\ \hline
    \onslide<1-7>{1 & $0$ & $1$ & $1$  \\ \hline
    {\tiny$\dots$} & $0$ & {\tiny$\dots$} & {\tiny$\dots$}   \\ \hline
    10 & $0$ & $0$ & $0$   \\ \hline \hline}11 & $1$ & $0$ & $1$   \\ \hline
    12 & $1$ & $1$ & $0$   \\ \hline
    {\tiny$\dots$} & $1$ & {\tiny$\dots$} & {\tiny$\dots$}  \\ \hline
\end{tabular}
\end{textblock}

\only<2->{
\begin{textblock}{110}(46,15) \normalsize
\begin{flalign*}
&
\only<2>{P(\text{Modelo}_{B\rightarrow A}|\text{Datos})}
\only<3->{\frac{P(\text{Modelo}_{B\rightarrow A}|\text{Datos})}{P(\text{Modelo}_{A\rightarrow B}|\text{Datos})}}
=
\only<2>{\frac{P(\text{Datos}|\text{M}_{B\rightarrow A}) \, P(\text{M}_{B\rightarrow A}) }{P(\text{Datos})}}
\only<3>{\frac{P(\text{Datos}|\text{M}_{B\rightarrow A}) \, P(\text{M}_{B\rightarrow A}) }{P(\text{Datos}|\text{M}_{A\rightarrow B}) \, P(\text{M}_{A\rightarrow B})}}
\only<4->{\frac{P(\text{Datos}|\text{M}_{B\rightarrow A})}{P(\text{Datos}|\text{M}_{A\rightarrow B}) }}
\\[0.5cm] &
\only<5>{ = \frac{\prod_i^n P(B_i, A_i, \text{do}_{A_i}|\text{M}_{BA}) }{\prod_i^n P(B_i, A_i, \text{do}_{A_i}|\text{M}_{AB}) }  }
\only<6>{ = \frac{\prod_i^n P(B_i|\text{M}_{_{BA}}) P_0(A_i|B_i,\text{M}_{_{BA}})^{1-\text{do}_A} P_1(A_i|\text{M}_{_{BA}})^{\text{do}_A}  P(\text{do}_A|\text{M}_{BA}) }{\prod_i^n P(A_i|\text{do}_A,\text{M}_{_{AB}}) P(B_i|A_i,\text{M}_{_{AB}}) P(\text{do}_A|\text{M}_{AB}) }}
\only<7-8>{ = \frac{\prod_i^n  P(B_i|\text{M}_{_{BA}}) P_0(A_i|B_i,\text{M}_{_{BA}})^{1-\text{do}_A} P_1(A_i|\text{M}_{_{BA}})^{\text{do}_A}  \phantom{P(\text{do}_A|\text{M}_{BA})} }{\prod_i^n  P(A_i|\text{do}_A,\text{M}_{_{AB}}) P(B_i|A_i,\text{M}_{_{AB}}) \phantom{P(\text{do}_A|\text{M}_{AB})} }}
\only<9>{ = \prod_{i=11}^n  \frac{P(B_i|\text{M}_{_{BA}}) \alpha^{A_i} \, (1 - \alpha)^{1-A_i}}{P(B_i|A_i,\text{M}_{_{AB}}) \alpha^{A_i} \, (1 - \alpha)^{1-A_i}  }  }
\only<10->{ = \prod_{i=11}^n \frac{P(B_i|\text{M}_{_{BA}})}{P(B_i|A_i,\text{M}_{_{AB}})  }  }
&&
\end{flalign*}

\centering
\only<11>{
\includegraphics[width=0.7\textwidth]{figuras/identificacion.pdf}
}
\end{textblock}
}

\end{frame}


\begin{frame}[plain]
\begin{textblock}{160}(0,4)
 \centering \LARGE Identificación de modelo causal\\
 \large El conocimiento experto
 \end{textblock}
 \vspace{0.75cm}


 \begin{textblock}{160}(0,42) \Large \centering
 La principal fuente de información para la identificación

 modelos causales alternativos es el conocimiento experto.
 \end{textblock}




\end{frame}


\begin{frame}[plain]
\begin{textblock}{160}(0,4)
 \centering \LARGE Identificación de efecto causal\\
 \large Con observación, sin intervenciones.
 \end{textblock}
 \vspace{0.75cm}


\only<1-2>{
\begin{textblock}{160}(0,24)
\begin{equation*}
P(y|\text{do}(x) = a, \text{Modelo Causal}) - P(y|\text{do}(x) = b, \text{Modelo Causal})
\end{equation*}
\end{textblock}
}

\only<2>{
\begin{textblock}{160}(0,44) \centering
\Large ¿Es posible evaluar el efecto causal sin hacer intervenciones?
\large
\begin{equation*}
P(y|\text{do}(x) = a, \text{Modelo Causal}) \overset{?}{=} P(y| x = a, Z, \text{Modelo Causal})
\end{equation*}
\end{textblock}
}


\only<3>{
\begin{textblock}{140}(10,27) \centering
\includegraphics[width=0.70\textwidth, page=1]{figuras/simpson.pdf}
\end{textblock}
}
\only<4>{
\begin{textblock}{140}(10,27) \centering
\includegraphics[width=0.70\textwidth, page=2]{figuras/simpson.pdf}
\end{textblock}
}


\only<3->{
\begin{textblock}{150}(5,14) \centering
\begin{flalign*}
& P(\text{Derrumbes}|do(\text{Intensidad del sismo})) = \only<3>{?}\only<4>{P(\text{Derrumbes}|\text{Intensidad del sismo}, \text{Ciudad})}
&&
\end{flalign*}
\end{textblock}
}


\end{frame}



\begin{frame}[plain]
\begin{textblock}{160}(0,4)
 \centering \LARGE Flujo de inferencia \\
 \end{textblock}
 \vspace{0.75cm}

% \only<1>{
%  \begin{textblock}{160}(0,23) \centering
%  \tikz{
%     \node[det] (u) {U} ; %
%     \node[const, above=of u] (nu) {Genotipo} ; %
%     \node[det, below=of u, xshift=-1.2cm] (f) {F} ; %
%     \node[const, left=of f] (nf) {Fuma} ; %
%     \node[det, below=of u, xshift=1.2cm] (c) {C};
%     \node[const, right=of c] (nc) {Cancer} ; %
%
%     \edge {u} {f,c};
%     \edge {f} {c};
%     }
%
% \vspace{1cm}
%  $P( \text{Cancer}|do(\text{Fuma}) ) \overset{?}{=}  P(\text{Cancer}|\text{Fuma})$
% \end{textblock}
% }


\only<1-2>{
\begin{textblock}{160}(0,14) \centering
 \tikz{
    \node[det] (l) {$l$} ; %
    \node[det, above=of l] (a) {$a$} ; %
    \node[det, above=of a,xshift=1.5cm] (t) {$t$} ; %
    \node[det, above=of a,xshift=-1.5cm] (e) {$e$} ; %
    \node[det, below=of t,xshift=1.5cm] (r) {$r$} ; %
    \node[det, above=of t,xshift=-1.5cm] (c) {$c$} ;


    \edge {a} {l};
    \edge {t,e} {a};
    \edge {t} {r};
    \edge {c} {e,t};

    \node[const, left= of e, xshift=-0.1cm] (cpd_e) {Entradera:};
    \node[const, right= of t, xshift=0.1cm] (cpd_t) {:Terremoto};
    \node[const, right= of c, xshift=0.1cm] (cpd_c) {:Ciudad};
    \node[const, left= of a, xshift=-0.1cm] (cpd_a) {Alarma:};
    \node[const, right= of r, xshift=0.1cm] (cpd_r) {:Redes};
    \node[const, left= of l, xshift=-0.1cm] (cpd_l) {Llamada:};

    }
\end{textblock}
}

\only<3>{
\begin{textblock}{160}(0,14) \centering
 \tikz{
    \node[det] (l) {$l$} ; %
    \node[det, above=of l] (a) {$a$} ; %
    \node[det, above=of a,xshift=1.5cm] (t) {$t$} ; %
    \node[det, double, double distance=0.5mm, above=of a,xshift=-1.5cm] (e) {$e$} ;
    \node[det, below=of t,xshift=1.5cm] (r) {$r$} ; %
    \node[det, above=of t,xshift=-1.5cm] (c) {$c$} ;


    \edge {a} {l};
    \edge {t,e} {a};
    \edge {t} {r};
    \edge {c} {t};

     \node[const, left= of e, xshift=-0.1cm] (cpd_e) {Entradera:};
     \node[const, right= of t, xshift=0.1cm] (cpd_t) {:Terremoto};
     \node[const, right= of c, xshift=0.1cm] (cpd_c) {:Ciudad};
     \node[const, left= of a, xshift=-0.1cm] (cpd_a) {Alarma:};
     \node[const, right= of r, xshift=0.1cm] (cpd_r) {:Redes};
     \node[const, left= of l, xshift=-0.1cm] (cpd_l) {Llamada:};

    }
\end{textblock}
}


\only<2->{
\begin{textblock}{160}(0,76) \centering
\begin{equation*}
P( \text{Llamada} | do(\text{Entradera}) ) \overset{?}{=}  P(\text{Llamada}|\text{Entradera})
\end{equation*}
\end{textblock}
}

\end{frame}


\begin{frame}[plain]
\begin{textblock}{160}(0,4)
 \centering \LARGE Flujo de inferencia \\
 \large Sum-product algorithm
\end{textblock}

\only<2-8>{
\begin{textblock}{80}(45,34)
\begin{description}
 \onslide<3->{\item[$v(n)$ :] Vecinos del nodo $n$}
 \item[$m_{x \rightarrow f}(x)$ :] Mensaje de variable $x$ a factor $f$
 \item[$m_{f \rightarrow x}(x)$ :] Mensaje de factor $f$ a variable $x$
\end{description}
\end{textblock}
}


\only<1-2>{
\begin{textblock}{80}(45,20)
\centering
Algoritmo para calcular cualquier marginal \\ mediante pasaje de mensajes entre nodos
\end{textblock}
}


\only<3-9>{
\begin{textblock}{80}(45,20)
\begin{equation*}
P(x) = \prod_{f \in v(x)} m_{f \rightarrow x}(x)
\end{equation*}
\end{textblock}
}

\only<4-9>{
\begin{textblock}{80}(45,56)
\begin{equation*}\label{eq:m_v_f}
m_{x \rightarrow f}(x) = \prod_{g \in \only<5>{\textcolor{red}}{v(x) \setminus \{f\}} } m_{g \rightarrow x}(x)
\end{equation*}
\end{textblock}
}

\only<6-9>{
\begin{textblock}{80}(45,68)
\begin{equation*}\label{eq:m_f_v}
 m_{f \rightarrow x}(x) = \onslide<8->{\sum_{\bm{y}} \Big(} \onslide<7->{f(\bm{y},x)} \prod_{y \in v(f) \setminus \{x\} } m_{y \rightarrow f}(y) \onslide<8->{\Big)}
\end{equation*}
\end{textblock}
}

\only<1->{
\begin{textblock}{40}(0,14)
\centering
\begin{figure}[H]
\centering
  \scalebox{.8}{
\tikz{ %
        \node[factor] (fa) {} ;
        \node[det, below=of fa, minimum size=0.55cm, yshift=0.4cm] (a) {$a$} ; %

        \node[factor, below=of a, yshift=0.4cm] (fl) {} ;
        \node[det, below=of fl, minimum size=0.55cm,yshift=0.4cm] (l) {$l$} ; %

        \node[det, above=of fa, minimum size=0.55cm, xshift=-0.8cm, yshift=-0.4cm] (e) {$e$} ; %
        \node[factor, above=of e, yshift=-0.4cm] (fe) {} ;

        \node[det,minimum size=0.55cm, above=of fa, xshift=0.8cm, yshift=-0.4cm] (t) {$t$} ; %
        \node[factor, above=of t, yshift=-0.4cm] (ft) {} ;
        \node[factor, below=of t, xshift=0.8cm, yshift=0.4cm] (fr) {} ;
        \node[det,minimum size=0.55cm, below=of fr, yshift=0.4cm] (r) {$r$} ; %

        \node[det, above=of fe, minimum size=0.55cm, xshift=0.8cm, yshift=-0.4cm] (c) {$c$} ; %
        \node[factor, above=of c, yshift=-0.4cm] (fc) {} ;


        \edge {fc} {c};
        \edge[-] {c} {fe,ft};
        \edge[-] {fa} {e,t};
        \edge {fa} {a};
        \edge {fe} {e};
        \edge {ft} {t};
        \edge[-] {fr} {t};
        \edge {fr} {r};
        \edge[-] {fl} {a};
        \edge {fl} {l};
        }
}
\end{figure}
\end{textblock}
}


\only<9>{
\begin{textblock}{110}(35,40) \Large
\centering El método de inferencia más eficiente
\end{textblock}
}

\end{frame}


 \begin{frame}[plain]
\begin{textblock}{160}(0,4)
 \centering \LARGE Flujo de inferencia \\
 \large Independencia condicional
 \end{textblock}
 \centering
 \vspace{0.75cm}

 \begin{textblock}{68}(0,25)
 \tikz{
    \node[det] (l) {$l$} ; %
    \node[det, above=of l] (a) {$a$} ; %
    \node[det, above=of a,xshift=1.5cm] (t) {$t$} ; %
    \node[det, above=of a,xshift=-1.5cm] (e) {$e$} ; %
    \node[det, below=of t,xshift=1.5cm] (r) {$r$} ; %

    \edge {a} {l};
    \edge {t,e} {a};
    \edge {t} {r};

    \node[const, above= of e, yshift=0.1cm] (cpd_e) {Entradera:};
    \node[const, above= of t, yshift=0.1cm] (cpd_t) {Terremoto:};
    \node[const, left= of a, xshift=-0.1cm] (cpd_a) {Alarma:};
    \node[const, left= of r, xshift=-0.1cm] (cpd_r) {Redes:};
    \node[const, left= of l, xshift=-0.1cm] (cpd_l) {Llamada:};

    }
\end{textblock}


\only<1->{
\begin{textblock}{90}(68,24)
\centering
 \begin{tabular}{c c c}
 & \onslide<2->{$\hfrac{\text{Intermedio}}{\text{no observable}}$} &   \onslide<3->{$\hfrac{\text{Intermedio}}{\text{observable}}}$ \\
 & & \\
 $ e \rightarrow a \rightarrow l $    & \onslide<2->{$P(l) \overset{?}{=} P(l|e)$} & \onslide<3->{$P(l|a) \overset{?}{=} P(l|e,a)$} \\
 %$ l \leftarrow a \leftarrow t $      &  \onslide<4->{$P(t) \overset{?}{=} P(t|l)$}  & \onslide<5->{$P(t|a) \overset{?}{=} P(t|a,l)$} \\
 $ a \leftarrow t \rightarrow r $     & \onslide<4->{$P(r) \overset{?}{=} P(r|a)$} & \onslide<5->{$P(r|t) \overset{?}{=} P(r|t,a)$} \\
 $ e \rightarrow a \leftarrow t $     & \onslide<6->{$P(t) \overset{?}{=} P(t|e)$} & \onslide<7->{$P(t|a) \overset{?}{=} P(t|e,a)$} \\
            $\overset{\downarrow}{l}$  &  & \onslide<8->{$P(t|l) \overset{?}{=} P(t|e,l)$}
 \end{tabular}
 \end{textblock}
 }


 \end{frame}

%  \begin{frame}[plain]
% \begin{textblock}{160}(0,4)
%  \centering \Large
%  Modelo Causal \\
%  \large Prior redes sociales
%  \end{textblock}
%  \vspace{0.75cm}
%
%  \centering
%
%
%  \begin{textblock}{160}(0,25)
%   $P(r|t)$ \\[0.1cm]
%     \begin{tabular}{|c|c|c|}
%         \hline
%         & \, $r^0$ \, & \, $r^1$ \,  \\ \hline
%        $(t^0)$ & \only<3-4>{$1$}\only<5>{$0.99$} & \only<3-4>{$0$}\only<5>{$0.01$}   \\ \hline
%        $(t^1)$ & \only<3-4>{$0$}\only<5>{$0.01$} & \only<3-4>{$1$}\only<5>{$0.99$}   \\ \hline
%     \end{tabular}
% \end{textblock}
%
%
%  \begin{textblock}{140}(10,54)
% \onslide<2->{
%  \begin{itemize}
%   \item[$\bullet$] Siempre que hay un terremoto, en alguna de mis redes sociales (whatsapp, facebook, twitter, instagram), se habla del tema.
%   \only<4->{\item[$\bullet$] También puedo mirar mal, o por algún momento no haya nada, o que por algún otro motivo nadie pueda comunicarse}
%  \end{itemize}
% }
% \end{textblock}
%
%
%  \end{frame}
%
%
%  \begin{frame}[plain]
% \begin{textblock}{160}(0,4)
%  \centering \Large
%  Modelo Causal \\
%  \large Prior llamada
%  \end{textblock}
%  \vspace{0.75cm}
%
%  \centering
%
%
%  \begin{textblock}{160}(0,25)
%   $P(l|a)$ \\[0.1cm]
%     \begin{tabular}{|c|c|c|}
%         \hline
%         & \, $l^0$ \, & \, $l^1$ \,  \\ \hline
%        $(a^0)$ & \onslide<3>{$0.99$} & \onslide<3>{$0.01$}   \\ \hline
%        $(a^1)$ & \onslide<3>{$0.01$} & \onslide<3>{$0.99$}   \\ \hline
%     \end{tabular}
% \end{textblock}
%
%
%  \begin{textblock}{140}(10,54)
% \onslide<2->{
%  \begin{itemize}
%   \item[$\bullet$] Siempre que se activa la alarma, me llaman desde el call center de la empresa (si no pasa nada raro)
%  \end{itemize}
% }
% \end{textblock}
%
%  \end{frame}

%
%  \begin{frame}[plain]
% \begin{textblock}{160}(0,4)
%  \centering \Large
%  Modelo Causal \\
%  \large Prior alarma
%  \end{textblock}
%  \vspace{0.75cm}
%
%  \centering
%
%
%  \begin{textblock}{160}(0,18)
%   $P(l|a)$ \\[0.1cm]
%     \begin{tabular}{|c|c|c|}
%         \hline
%         & \hspace{1cm} $a^0$ \hspace{1cm} & \hspace{1cm} $a^1$ \hspace{1cm} \\ \hline
%        $(e^0, t^0)$ & \only<10->{$0.99$}\only<4-9>{$P(\overline{\alpha})$} & \only<11->{$0.01$}\only<4-10>{$P(\alpha)$} \\ \hline
%        $(e^1, t^0)$ & \only<13->{$\approx 0.01$}\only<12>{$0.99 \cdot 0.01$}\only<7-11>{$P(\overline{\alpha} \cap \overline{\varepsilon})$}\only<6>{$P(\overline{\alpha \cup \varepsilon})$} & \only<14->{$\approx 0.99$} \only<5-13>{$P(\alpha \cup \varepsilon)$}  \\ \hline
%        $(e^0, t^1)$ & \only<15->{$\approx 0.01$}\only<8-14>{$P(\overline{\alpha} \cap \overline{\tau})$} &\only<15->{$\approx0.99$}\only<8-14>{$P(\alpha \cup \tau)$}  \\ \hline
%        $(e^1, t^1)$ & \only<16>{$\approx 0.0001$} \only<9-15>{$P(\overline{\alpha} \cap \overline{\varepsilon}  \cap \overline{\tau})$} & \only<16>{$\approx 0.9999$} \only<9-15>{$P(\alpha \cup \varepsilon \cup \tau)$}  \\ \hline
%     \end{tabular}
% \end{textblock}
%
%
%  \begin{textblock}{140}(10,52)
% \onslide<2->{
%  \begin{itemize}
%   \item[$\bullet$] Siempre que entra alguien a la casa o que hay un terremoto se activa la alarma (si no pasa nada raro):
%   \onslide<3->{
%   \begin{description}
%    \item[$\alpha$] Se activa sola: $P(\alpha) = 0.01$
%    \item[$\overline{\varepsilon}$] No se activa a pesar de entradera: $P(\overline{\varepsilon}) = 0.01$
%    \item[$\overline{\tau}$] No se activa a pesar de terremoto: $P(\overline{\tau}) = 0.01$
%   \end{description}
% }
%  \end{itemize}
% }
% \end{textblock}
%
%  \end{frame}
%

 \begin{frame}[plain]
\begin{textblock}{160}(0,4)
 \centering \Large
 Modelo Causal
 \end{textblock}

\begin{textblock}{160}(0,12)
 \centering

 \tikz{
    \node[det] (l) {$l$} ; %
    \node[det, above=of l] (a) {$a$} ; %
    \node[det, above=of a,xshift=1.5cm] (t) {$t$} ; %
    \node[det, above=of a,xshift=-1.5cm] (e) {$e$} ; %
    \node[det, below=of t,xshift=1.5cm] (r) {$r$} ; %

    \edge {a} {l};
    \edge {t,e} {a};
    \edge {t} {r};

    \onslide<2->{
    \node[const, left= of e, xshift=-0.1cm] (cpd_e) {
    \begin{tabular}{|c|c|}
        \hline
        $e^0$ & $e^1$ \\ \hline
        $0.999$ & $0.001$  \\ \hline
    \end{tabular}
    };
    \node[const, above= of cpd_e] (n_e) {Entradera:};
        }

    \onslide<3->{
    \node[const, right= of t, xshift=0.1cm] (cpd_t) {
    \begin{tabular}{|c|c|}
        \hline
        $t^0$ & $t^1$ \\ \hline
        $0.992$ & $0.008$  \\ \hline
    \end{tabular}
    };
    \node[const, above= of cpd_t] (n_t) {Terremoto:};
    }

    \onslide<4->{
    \node[const, left= of a, yshift=-0.7cm, xshift=-0.3cm] (cpd_a) {
    \begin{tabular}{|c|c|c|}
        \hline
        & $a^0$ & $a^1$ \\ \hline
       ($e^0, t^0$) & $0.99$ & $0.01$  \\ \hline
       ($e^1, t^0$) & $0.01$ & $0.99$  \\ \hline
       ($e^0, t^1$) & $0.01$ & $0.99$  \\ \hline
       ($e^1, t^1$) & $0.0001$ & $0.9999$  \\ \hline
    \end{tabular}
    };
    \node[const, above= of cpd_a] (n_a) {Alarma:};
    }

 \onslide<5->{
    \node[const, right= of r, xshift=0.1cm] (cpd_r) {
    \begin{tabular}{|c|c|c|}
        \hline
        & \, $r^0$ \, & \, $r^1$ \,  \\ \hline
       $(t^0)$ & $0.99$ & $0.01$   \\ \hline
       $(t^1)$ & $0.01$ & $0.99$   \\ \hline
    \end{tabular}
    };
    \node[const, above= of cpd_r] (n_r) {Redes:};
    }


 \onslide<6->{
    \node[const, right= of l, xshift=0.1cm] (cpd_l) {
    \begin{tabular}{|c|c|c|}
        \hline
        & \, $l^0$ \, & \, $l^1$ \,  \\ \hline
       $(a^0)$ & $0.99$ & $0.01$   \\ \hline
       $(a^1)$ & $0.01$ & $0.99$   \\ \hline
    \end{tabular}
    };
    \node[const, above= of cpd_l] (n_l) {Llamada:};
    }


 }
\end{textblock}

\only<2>{
\begin{textblock}{160}(10,66)
 \begin{itemize}
  \item[$\bullet$] Una vez cada 3 años
  \item[$\bullet$] Una casa cada 1000 por día
 \end{itemize}
\end{textblock}
}

\only<3>{
 \begin{textblock}{160}(10,66)
 \begin{itemize}
  \item[$\bullet$] Hay 3 terremotos por años
 \end{itemize}
\end{textblock}
}

\only<4>{
 \begin{textblock}{160}(10,66)
 \begin{itemize}
   \item[$\bullet$] Se activa sola: $P(\alpha) = 0.01$
   \item[$\bullet$] No se activa a pesar de entradera: $P(\overline{\varepsilon}) = 0.01$
   \item[$\bullet$] No se activa a pesar de terremoto: $P(\overline{\tau}) = 0.01$
 \end{itemize}
\end{textblock}
}

\only<5>{
 \begin{textblock}{160}(10,66)
 \begin{itemize}
   \item[$\bullet$] Siempre se habla de los terremotos en las redes (si no pasa nada raro)
 \end{itemize}
\end{textblock}
}

\only<6>{
 \begin{textblock}{160}(10,66)
 \begin{itemize}
   \item[$\bullet$] Llaman siempre que se activa la alarma (si no pasa nada raro)
 \end{itemize}
\end{textblock}
}



\end{frame}



\begin{frame}[plain]
\begin{textblock}{160}(0,4)
 \centering \Large
 Modelo Causal
 \end{textblock}

\begin{textblock}{160}(0,12)
 \centering

 \tikz{
    \node[det] (l) {$l$} ; %
    \node[det, above=of l] (a) {$a$} ; %
    \node[det, above=of a,xshift=1.5cm] (t) {$t$} ; %
    \node[det, above=of a,xshift=-1.5cm] (e) {$e$} ; %
    \node[det, below=of t,xshift=1.5cm] (r) {$r$} ; %

    \edge {a} {l};
    \edge {t,e} {a};
    \edge {t} {r};

    \node[const, left= of e, xshift=-0.1cm] (cpd_e) {
    \begin{tabular}{|c|c|}
        \hline
        $e^0$ & $e^1$ \\ \hline
        $0.999$ & $0.001$  \\ \hline
    \end{tabular}
    };
    \node[const, above= of cpd_e] (n_e) {Entradera:};

    \node[const, right= of t, xshift=0.1cm] (cpd_t) {
    \begin{tabular}{|c|c|}
        \hline
        $t^0$ & $t^1$ \\ \hline
        $0.992$ & $0.008$  \\ \hline
    \end{tabular}
    };
    \node[const, above= of cpd_t] (n_t) {Terremoto:};

    \node[const, left= of a, yshift=-0.7cm, xshift=-0.3cm] (cpd_a) {
    \begin{tabular}{|c|c|c|}
        \hline
        & $a^0$ & $a^1$ \\ \hline
       ($e^0, t^0$) & $0.99$ & $0.01$  \\ \hline
       ($e^1, t^0$) & $0.01$ & $0.99$  \\ \hline
       ($e^0, t^1$) & $0.01$ & $0.99$  \\ \hline
       ($e^1, t^1$) & $0.0001$ & $0.9999$  \\ \hline
    \end{tabular}
    };
    \node[const, above= of cpd_a] (n_a) {Alarma:};

    \node[const, right= of r, xshift=0.1cm] (cpd_r) {
    \begin{tabular}{|c|c|c|}
        \hline
        & \, $r^0$ \, & \, $r^1$ \,  \\ \hline
       $(t^0)$ & $0.99$ & $0.01$   \\ \hline
       $(t^1)$ & $0.01$ & $0.99$   \\ \hline
    \end{tabular}
    };
    \node[const, above= of cpd_r] (n_r) {Redes:};

    \node[const, right= of l, xshift=0.1cm] (cpd_l) {
    \begin{tabular}{|c|c|c|}
        \hline
        & \, $l^0$ \, & \, $l^1$ \,  \\ \hline
       $(a^0)$ & $0.99$ & $0.01$   \\ \hline
       $(a^1)$ & $0.01$ & $0.99$   \\ \hline
    \end{tabular}
    };
    \node[const, above= of cpd_l] (n_l) {Llamada:};

 }
\end{textblock}

\only<1>{
\begin{textblock}{130}(15,63)
\begin{flalign*}
 & P(e^0,t^0,a^0,r^0,l^0) = P(e^0)P(t^0)P(a^0|t^0,e^0)P(r^0|t^0)P(l^0|a^0) &&
\end{flalign*}
\end{textblock}
}

\only<2>{
\begin{textblock}{130}(15,63)
\begin{flalign*}
 & P(e^0,t^0,a^0,r^0,l^0) = 0.999 \cdot P(t^0)P(a^0|t^0,e^0)P(r^0|t^0)P(l^0|a^0) &&
\end{flalign*}
\end{textblock}
}

\only<3>{
\begin{textblock}{130}(15,63)
\begin{flalign*}
 & P(e^0,t^0,a^0,r^0,l^0) = 0.999 \cdot 0.992 \cdot 0.99 \cdot 0.99 \cdot 0.99 &&
\end{flalign*}
\end{textblock}
}

\only<4>{
\begin{textblock}{130}(15,63)
\begin{flalign*}
 & P(e^0,t^0,a^0,r^0,l^0) = 0.999 \cdot 0.992 \cdot 0.99 \cdot 0.99 \cdot 0.99 \approx 0.96 &&
\end{flalign*}
\end{textblock}
}

\only<5>{
 \begin{textblock}{130}(15,63)
 \begin{flalign*}
  & P(a^1) =  \phantom{ \sum_{e,t,r,l}} &&
 \end{flalign*}
 \end{textblock}
}

\only<6>{
\begin{textblock}{130}(15,63)
\begin{flalign*}
 & P(a^1) = \sum_e\sum_t\sum_r\sum_l P(e,t,a^1,r,l) &&
\end{flalign*}
\end{textblock}
}

\only<7>{
\begin{textblock}{130}(15,63)
\begin{flalign*}
 & P(a^1) = \sum_{e,t,r,l} P(e,t,a^1,r,l) &&
\end{flalign*}
\end{textblock}
}

\only<8>{
\begin{textblock}{130}(15,63)
\begin{flalign*}
 & P(a^1) = \sum_{e,t,r,l} P(e)P(t)P(a^1|t,e)P(r|t)P(l|a^1) &&
\end{flalign*}
\end{textblock}
}

\only<9>{
\begin{textblock}{130}(15,63)
\begin{flalign*}
 & P(a^1) = \sum_{e,t,r,\textcolor{red}{l}} P(e)P(t)P(a^1|t,e)P(r|t)\textcolor{red}{\bm{P(l|a^1)}} &&
\end{flalign*}
\end{textblock}
}

\only<10>{
\begin{textblock}{130}(15,63)
\begin{flalign*}
  P(a^1) & = P(l^0|a^1) \sum_{e,t,r} P(e)P(t)P(a^1|t,e)P(r|t) \\
 & + P(l^1|a^1) \sum_{e,t,r} P(e)P(t)P(a^1|t,e)P(r|t) &&
\end{flalign*}
\end{textblock}
}

\only<11>{
\begin{textblock}{130}(15,63)
\begin{flalign*}
 & P(a^1) = \Big(\sum_l P(l|a^1) \Big) \Big(\sum_{e,t,r} P(e)P(t)P(a^1|t,e)P(r|t) \Big) &&
\end{flalign*}
\end{textblock}
}

\only<12>{
\begin{textblock}{130}(15,63)
\begin{flalign*}
 & P(a^1) = \Big(\sum_l P(l|a^1) \Big) \Big(\sum_{e,\textcolor{red}{\bm{t}},r} P(e)P(t)P(a^1|t,e)P(r|\textcolor{red}{\bm{t}}) \Big) &&
\end{flalign*}
\end{textblock}
}


\only<13>{
\begin{textblock}{130}(15,63)
\begin{flalign*}
 & P(a^1) = \Big(\sum_l P(l|a^1) \Big) \Big(\sum_{e,t} P(e)P(t)P(a^1|t,e)(\sum_r P(r|t)) \Big) &&
\end{flalign*}
\end{textblock}
}


\only<14>{
\begin{textblock}{130}(15,63)
\begin{flalign*}
 & P(a^1) = \Big(\cancel{\sum_l P(l|a^1)} \Big) \Big(\sum_{e,t} P(e)P(t)P(a^1|t,e)(\sum_r P(r|t)) \Big) &&
\end{flalign*}
\end{textblock}
}


\only<15>{
\begin{textblock}{130}(15,63)
\begin{flalign*}
 & P(a^1) = \Big(\cancel{\sum_l P(l|a^1)} \Big) \Big(\sum_{e,t} P(e)P(t)P(a^1|t,e)(\cancel{\sum_r P(r|t)}) \Big) &&
\end{flalign*}
\end{textblock}
}

\only<16>{
\begin{textblock}{130}(15,63)
\begin{flalign*}
 & P(a^1) = \sum_{e,t} P(e)P(t)P(a^1|t,e) \phantom{\cancel{\sum_h}}&&
\end{flalign*}
\end{textblock}
}

\only<17>{
\begin{textblock}{130}(15,63)
\begin{flalign*}
 & P(a^1) = \sum_{e,t} P(e)P(t)P(a^1|t,e) = ?  &&
\end{flalign*}
\end{textblock}
}

\only<18>{
\begin{textblock}{130}(15,63)
\begin{flalign*}
 & P(a^1) = \sum_{e,t} P(e)P(t)P(a^1|t,e) \approx 0.019  &&
\end{flalign*}
\end{textblock}
}

\end{frame}

%
% \begin{frame}[plain]
% \begin{textblock}{160}(0,4)
%  \centering \Large
%  Grafo de factorización
% \end{textblock}
%
% \begin{textblock}{40}(5,12)
% \centering
% \begin{figure}[H]
% \centering
%   \scalebox{0.8}{
% \tikz{ %
%
%
%         \node[factor] (fa) {} ;
%         \node[const,right=of fa] (pa) {$P(a|e,t)$} ;
%         \node[det, below=of fa, yshift=0.1cm] (a) {$a$} ; %
%
%         \node[factor, below=of a] (fl) {} ;
%         \node[const,right=of fl] (pl) {$P(l|a)$} ;
%         \node[det, below=of fl,yshift=0.1cm] (l) {$l$} ; %
%
%         \node[det, above=of fa, xshift=-1.6cm,yshift=-0.1cm] (e) {$e$} ; %
%         \node[factor, above=of e,yshift=-0.1cm] (fe) {} ;
%         \node[const,right=of fe] (pe) {$P(e)$} ;
%
%         \node[det, above=of fa, xshift=1.6cm,yshift=-0.1cm] (t) {$t$} ; %
%         \node[factor, above=of t,yshift=-0.1cm] (ft) {} ;
%         \node[const,right=of ft] (pt) {$P(t)$} ;
%         \node[factor, below=of t, xshift=1.6cm,yshift=0.1cm] (fr) {} ;
%         \node[det,below=of fr,yshift=0.1cm] (r) {$r$} ; %
%         \node[const,right=of fr] (pr) {$P(r|t)$} ;
%
%         \edge[-] {fa} {a,e,t};
%         \edge[-] {fe} {e};
%         \edge[-] {ft} {t};
%         \edge[-] {fr} {r,t};
%         \edge[-] {fl} {l,a};
%         }
% }
% \end{figure}
% \end{textblock}
%
% \begin{textblock}{115}(45,30)
%
%  \centering Nodos: \\ variables y funciones
%
%  \vspace{1cm}
%
%  \centering Ejes: \\
%   ``la variable $v$ es argumento de la funci\'on $f$''
%
% \end{textblock}
% \end{frame}


\begin{frame}[plain]
\begin{textblock}{160}(0,4)
 \centering \Large El truco de la factorización
\end{textblock}
\vspace{1.5cm} \centering

\begin{equation*}
\sum \prod = \prod \sum
\end{equation*}

$a_1 \, b_1 + a_1 \, b_2 + a_2 \, b_1 + a_2 \, b_2  = (a1 + a_2) \, (b_1 + b_2)$

\pause

\begin{equation*}
   P(a^1) = \Big(\sum_l P(l|a^1) \Big) \Big(\sum_{e,t} P(e)P(t)P(a^1|t,e)(\sum_r P(r|t)) \Big)
\end{equation*}

  \includegraphics[width=0.35\textwidth]{../../auxiliar/static/sum_product.png}

\end{frame}


\begin{frame}[plain]
\begin{textblock}{160}(0,4)
 \centering \Large
 Modelo sin observables
\end{textblock}

\begin{textblock}{40}(5,12)
\centering
\begin{figure}[H]
\centering
  \scalebox{0.8}{
\tikz{ %


        \node[factor] (fa) {} ;
        %\node[const,left=of fa] (pa) {$P(a|e,t)$} ;
        \node[det, below=of fa, yshift=0.1cm] (a) {$a$} ; %

        \node[factor, below=of a] (fl) {} ;
        %\node[const,right=of fl] (pl) {$P(l|a)$} ;
        \node[det, below=of fl,yshift=0.1cm] (l) {$l$} ; %

        \node[det, above=of fa, xshift=-1.6cm,yshift=-0.1cm] (e) {$e$} ; %
        \node[factor, above=of e,yshift=-0.1cm] (fe) {} ;
        %\node[const,right=of fe] (pe) {$P(e)$} ;

        \node[det, above=of fa, xshift=1.6cm,yshift=-0.1cm] (t) {$t$} ; %
        \node[factor, above=of t,yshift=-0.1cm] (ft) {} ;
        %\node[const,right=of ft] (pt) {$P(t)$} ;
        \node[factor, below=of t, xshift=1.6cm,yshift=0.1cm] (fr) {} ;
        \node[det,below=of fr,yshift=0.1cm] (r) {$r$} ; %
        %\node[const,left=of fr] (pr) {$P(r|t)$} ;


        \edge[-] {fa} {a,e,t};
        \edge[-] {fe} {e};
        \edge[-] {ft} {t};
        \edge[-] {fr} {r,t};
        \edge[-] {fl} {l,a};

        \onslide<1->{\path[draw, ->, fill=black!50] (fe) edge[bend left,draw=black!50] node[right,color=black!75] {\only<2->{$P(e)$}} (e);}
        \onslide<3>{\path[draw, ->, fill=black!50] (e) edge[bend left,draw=black!50] node[above,color=black!75] {$P(e)$} (fa);}
        \onslide<4->{\path[draw, ->, fill=black!50] (ft) edge[bend left,draw=black!50] node[right,color=black!75] {$P(t)$} (t);}
        \onslide<5>{\path[draw, ->, fill=black!50] (r) edge[bend left,draw=black!50] node[left,color=black!75] {$1$} (fr);}
        \onslide<6->{\path[draw, ->, fill=black!50] (fr) edge[bend left,draw=black!50] node[left,color=black!75] {\only<7->{$1$}} (t);}
        \onslide<8->{\path[draw, ->, fill=black!50] (fl) edge[bend left,draw=black!50] node[left,color=black!75] {$1$} (a);}
        \onslide<9->{\path[draw, ->, fill=black!50] (fa) edge[bend left,draw=black!50] node[left,color=black!75] {\only<12->{$1$}} (t);}
        \onslide<13->{\path[draw, ->, fill=black!50] (fa) edge[bend left,draw=black!50] node[left,color=black!75] {$1$} (e);}
        \onslide<15->{\path[draw, ->, fill=black!50] (fr) edge[bend left,draw=black!50] node[right,color=black!75] {\only<16->{$P(r)$}} (r);}
        \onslide<17->{\path[draw, ->, fill=black!50] (fa) edge[bend left,draw=black!50] node[right,color=black!75] {\only<18->{$P(a)$}} (a);}
        \onslide<19->{\path[draw, ->, fill=black!50] (fl) edge[bend left,draw=black!50] node[right,color=black!75] {$P(l)$} (l);}
        }
}
\end{figure}
\end{textblock}

\begin{textblock}{100}(62,16)
 $m_{f_e\rightarrow e}(e) = \onslide<2->{P(e)} \onslide<3>{= m_{e\rightarrow f_a}(e)}$  \\
 \onslide<4->{$m_{f_t\rightarrow t}(t) = P(t)$ } \\
 \only<5>{$m_{r \rightarrow f_r}(r) = 1$ \\}
 $\onslide<6->{m_{f_r\rightarrow t}(t) = \sum_r \, P(r|t)} \onslide<7->{= 1}$  \\
 $\onslide<8->{m_{f_l\rightarrow a}(a) = \sum_l \, P(l|a) = 1} $  \\
 \onslide<9->{$m_{f_a\rightarrow t}(t) = \sum_{ea} \, P(e) P(a|e,t) \only<10>{= \sum_{e} \, P(e) \sum_{a} P(a|e,t) } \only<11>{= \sum_{e} \, P(e)} \only<12->{= 1 }$  }\\
 \onslide<13->{$m_{f_a\rightarrow e}(e) = \sum_{ta} \, P(t) P(a|e,t) = 1$ } \\
 \onslide<15->{$m_{f_r\rightarrow r}(r) = \sum_{t} \, P(t) P(r|t) \onslide<16->{= P(r)}$ } \\
 \onslide<17->{$m_{f_a\rightarrow a}(a) = \sum_{et} \, P(e)P(t)P(a|e,t)  \onslide<18->{= P(a)}$ } \\
 \onslide<19->{$m_{f_l\rightarrow l}(l) = \sum_{a} \, P(a) P(l|a) = P(l)$ }
 \end{textblock}

 \only<14>{
\begin{textblock}{80}(60,60)
 \centering
 Todos los mensajes \\ que suben son $1$
\end{textblock}
}

 \only<20>{
\begin{textblock}{80}(60,70)
 \centering
 Todos los mensajes que \\ bajan son marginales
\end{textblock}
}

\end{frame}


\begin{frame}[plain]
\begin{textblock}{160}(0,4)
 \centering \Large
Modelo con observables
\end{textblock}

\begin{textblock}{40}(5,12)
\centering
\begin{figure}[H]
\centering
  \scalebox{0.8}{
\tikz{ %


        \node[factor] (fa) {} ;
        %\node[const,left=of fa] (pa) {$P(a|e,t)$} ;
        \node[det, below=of fa, yshift=0.1cm] (a) {$a$} ; %

        \node[factor, below=of a] (fl) {} ;
        %\node[const,right=of fl] (pl) {$P(l|a)$} ;
        \node[det, below=of fl,yshift=0.1cm] (l) {$l$} ; %

        \node[det, fill=black!10,above=of fa, xshift=-1.6cm,yshift=-0.1cm] (e) {$e^*$} ; %
        \node[factor, above=of e,yshift=-0.1cm] (fe) {} ;
        %\node[const,right=of fe] (pe) {$P(e)$} ;

        \node[det, above=of fa, xshift=1.6cm,yshift=-0.1cm] (t) {$t$} ; %
        \node[factor, above=of t,yshift=-0.1cm] (ft) {} ;
        %\node[const,right=of ft] (pt) {$P(t)$} ;
        \node[factor, below=of t, xshift=1.6cm,yshift=0.1cm] (fr) {} ;
        \node[det,below=of fr,yshift=0.1cm] (r) {$r$} ; %
        %\node[const,left=of fr] (pr) {$P(r|t)$} ;


        \edge[-] {fa} {a,e,t};
        \edge[-] {fe} {e};
        \edge[-] {ft} {t};
        \edge[-] {fr} {r,t};
        \edge[-] {fl} {l,a};

        \path[draw, ->, fill=black!50] (fe) edge[bend left,draw=black!50] node[right,color=black!75] {$P(e^*)$} (e);
        \path[draw, ->, fill=black!50] (ft) edge[bend left,draw=black!50] node[right,color=black!75] {$P(t)$} (t);
        \path[draw, ->, fill=black!50] (fr) edge[bend left,draw=black!50] node[left,color=black!75] {\only<1->{$1$}} (t);
        \path[draw, ->, fill=black!50] (fl) edge[bend left,draw=black!50] node[left,color=black!75] {$1$} (a);
        \onslide<2->{\path[draw, ->, fill=black!50] (fa) edge[bend left,draw=black!50] node[left,color=black!75] {\only<3->{$P(e^*)$}} (t);}
        \path[draw, ->, fill=black!50] (fa) edge[bend left,draw=black!50] node[left,color=black!75] {$1$} (e);
        \onslide<7->{\path[draw, ->, fill=black!50] (fr) edge[bend left,draw=black!50] node[right,color=black!75] {\only<8->{$P(r,e^*)$}} (r);}
        \onslide<9->{\path[draw, ->, fill=black!50] (fa) edge[bend left,draw=black!50] node[right,color=black!75] {\only<10->{$P(a,e^*)$}} (a);}
        \onslide<11->{\path[draw, ->, fill=black!50] (fl) edge[bend left,draw=black!50] node[right,color=black!75] {\only<12->{$P(l,e^*)$}} (l);}
        }
}
\end{figure}
\end{textblock}

\begin{textblock}{100}(62,16)
 $m_{f_e\rightarrow e}\phantom{(e)} = P(e^*) $  \\
 $m_{f_t\rightarrow t}(t) = P(t)$  \\
 $m_{f_r\rightarrow t}(t) = \sum_r \, P(r|t) = 1$  \\
 $m_{f_l\rightarrow a}(a) = \sum_l \, P(l|a) = 1 $  \\
 $m_{f_a\rightarrow t}(t) = \onslide<2->{\sum_{a} \, P(e^*) P(a|e^*,t)} \onslide<3->{= P(e^*)}$  \\
 $m_{f_a\rightarrow e}\phantom{(e)} = \sum_{ta} \, P(t) P(a|e^*,t) = 1$  \\
 $m_{f_r\rightarrow r}(r) = \onslide<7->{\sum_{t} \, P(t) P(r|t)} \onslide<8->{= P(r,e^*)}$ \\
 $m_{f_a\rightarrow a}(a) = \onslide<9->{\sum_{t} \, P(e^*)P(t)P(a|e^*,t)}\onslide<10->{  = P(a,e^*)}$  \\
 $m_{f_l\rightarrow l}(l) = \onslide<11->{\sum_{a} \, P(a,e^*) P(l|a)} \onslide<12->{ = P(l,e^*)}$
 \end{textblock}


\only<4-6>{
\begin{textblock}{80}(70,60)
\begin{flalign*}
  P(t,e^*) & = P(e^*) P(t) \\
  \onslide<5->{P(t|e^*) &= \frac{P(e^*) P(t) }{P(e^*)}} \onslide<6->{= P(t)}&&
\end{flalign*}
\end{textblock}
}

\only<13-14>{
\begin{textblock}{80}(70,60)
\begin{flalign*}
  P(l|e^*) & \, = P(l,e^*)/P(e^*) \only<14>{\neq \, P(l)} &&
\end{flalign*}
\end{textblock}
}

\only<15>{
\begin{textblock}{80}(69.15,60)
\begin{flalign*}
  \underbrace{P(l|e^*)}_{\hfrac{\text{que llamen}}{\text{cuando entran}}} & = P(l,e^*)/P(e^*) \neq  \underbrace{P(l)}_{\hfrac{\text{que llamen}}{\text{}}} &&
\end{flalign*}
\end{textblock}
}

\end{frame}




 \begin{frame}[plain]
\begin{textblock}{160}(0,4)
 \centering \LARGE Flujo de inferencia \\
 \large Independencia condicional
 \end{textblock}
 \centering
 \vspace{0.75cm}

 \only<1-3>{
 \begin{textblock}{68}(0,25)
 \tikz{
    \node[det] (l) {$l$} ; %
    \node[det, above=of l] (a) {$a$} ; %
    \node[det, above=of a,xshift=1.5cm] (t) {$t$} ; %
    \node[det, above=of a,xshift=-1.5cm] (e) {$e$} ; %
    \node[det, below=of t,xshift=1.5cm] (r) {$r$} ; %

    \edge {a} {l};
    \edge {t,e} {a};
    \edge {t} {r};

    \node[const, above= of e, yshift=0.1cm] (cpd_e) {Entradera:};
    \node[const, above= of t, yshift=0.1cm] (cpd_t) {Terremoto:};
    \node[const, left= of a, xshift=-0.1cm] (cpd_a) {Alarma:};
    \node[const, left= of r, xshift=-0.1cm] (cpd_r) {Redes:};
    \node[const, left= of l, xshift=-0.1cm] (cpd_l) {Llamada:};

    }
\end{textblock}
}

\only<1-3>{
\begin{textblock}{90}(68,24)
\centering
 \begin{tabular}{c c c}
 & \onslide<1->{$\hfrac{\text{Intermedio}}{\text{no observable}}$} &   \onslide<1->{$\hfrac{\text{Intermedio}}{\text{observable}}}$ \\
 & & \\
 $ e \rightarrow a \rightarrow l $    & \only<1>{$P(l) \overset{?}{=} P(l|e)$}\only<2->{$P(l) \neq P(l|e)$} & \only<1-2>{$P(l|a) \overset{?}{=} P(l|e,a)$}\only<3->{$P(l|a) \overset{\phantom{?}}{=} P(l|e,a)$} \\
 $ a \leftarrow t \rightarrow r $     & \only<1-2>{$P(r) \overset{?}{=} P(r|a)$}\only<3->{$P(r) \neq P(r|a)$} & \only<1-2>{$P(r|t) \overset{?}{=} P(r|t,a)$}\only<3->{$P(r|t) \overset{\phantom{?}}{=} P(r|t,a)$} \\
 $ e \rightarrow a \leftarrow t $     & \only<1>{$P(t) \overset{?}{=} P(t|e)$}\only<2->{$P(t) \overset{\phantom{?}}{=} P(t|e)$} & \only<1-2>{$P(t|a) \overset{?}{=} P(t|e,a)$}\only<3->{$P(t|a) \neq P(t|e,a)$} \\
            $\overset{\downarrow}{l}$  &  & \only<1-2>{$P(t|l) \overset{?}{=} P(t|e,l)$}\only<3->{$P(t|l) \neq P(t|e,l)$}
 \end{tabular}
\end{textblock}
 }


\only<4>{
\begin{textblock}{140}(10,32) \Large
 Hay flujo de inferencia entre los extremos de una cadena si:\\[0.3cm] \large
 \begin{itemize}
  \item[$\bullet$] Todas las consecuencias comunes (o sus descendientes) son observables
  \item[$\bullet$] Ning\'una otra variable es observable
 \end{itemize}
\end{textblock}
}

 \end{frame}


 \begin{frame}[plain]
\begin{textblock}{160}(0,4)
 \centering \LARGE Flujo de inferencia causal \\
 \end{textblock}
 \vspace{0.75cm}


\only<2->{
\begin{textblock}{160}(0,14) \centering
 \tikz{
    \node[det] (l) {$l$} ; %
    \node[det, above=of l] (a) {$a$} ; %
    \node[det, above=of a,xshift=1.5cm] (t) {$t$} ; %
    \node[det, fill=black!15, above=of a,xshift=-1.5cm] (e) {$e$} ; %
    \node[det, below=of t,xshift=1.5cm] (r) {$r$} ; %
    \node[det, fill=black!15,above=of t,xshift=-1.5cm] (c) {$c$} ;


    \edge {a} {l};
    \edge {t,e} {a};
    \edge {t} {r};
    \edge {c} {e,t};

    \node[const, left= of e, xshift=-0.1cm] (cpd_e) {Entradera:};
    \node[const, right= of t, xshift=0.1cm] (cpd_t) {:Terremoto};
    \node[const, right= of c, xshift=0.1cm] (cpd_c) {:Ciudad};
    \node[const, left= of a, xshift=-0.1cm] (cpd_a) {Alarma:};
    \node[const, right= of r, xshift=0.1cm] (cpd_r) {:Redes};
    \node[const, left= of l, xshift=-0.1cm] (cpd_l) {Llamada:};

    }
\end{textblock}
}

\only<1>{
\begin{textblock}{160}(0,14) \centering
 \tikz{
    \node[det] (l) {$l$} ; %
    \node[det, above=of l] (a) {$a$} ; %
    \node[det, above=of a,xshift=1.5cm] (t) {$t$} ; %
    \node[det, double, double distance=0.5mm, above=of a,xshift=-1.5cm] (e) {$e$} ;
    \node[det, below=of t,xshift=1.5cm] (r) {$r$} ; %
    \node[det, above=of t,xshift=-1.5cm] (c) {$c$} ;


    \edge {a} {l};
    \edge {t,e} {a};
    \edge {t} {r};
    \edge {c} {t};

     \node[const, left= of e, xshift=-0.1cm] (cpd_e) {Entradera:};
     \node[const, right= of t, xshift=0.1cm] (cpd_t) {:Terremoto};
     \node[const, right= of c, xshift=0.1cm] (cpd_c) {:Ciudad};
     \node[const, left= of a, xshift=-0.1cm] (cpd_a) {Alarma:};
     \node[const, right= of r, xshift=0.1cm] (cpd_r) {:Redes};
     \node[const, left= of l, xshift=-0.1cm] (cpd_l) {Llamada:};

    }
\end{textblock}
}


\only<1-2>{
\begin{textblock}{160}(0,76) \centering
\begin{equation*}
P( \text{Llamada} | do(\text{Entradera}) ) = \onslide<2>{ P(\text{Llamada}|\text{Entradera}, \text{Ciudad}) }
\end{equation*}
\end{textblock}
}


\only<3>{
\begin{textblock}{150}(5,75)
Backdoor criterion

\Large
 Hay inferencia casual si todos los flujos ``traseros'' permanecen cerrados.
\end{textblock}
}

\end{frame}



 \begin{frame}[plain]
\begin{textblock}{160}(0,4)
 \centering \LARGE  Flujo de inferencia causal
 \end{textblock}

\begin{textblock}{160}(0,12)
 \centering

 \tikz{
    \node[det] (l) {$l$} ; %
    \node[det, above=of l] (a) {$a$} ; %
    \node[det, above=of a,xshift=1.5cm] (t) {$t$} ; %
    \only<1-2>{\node[det, above=of a,xshift=-1.5cm] (e) {$e$} ; }
    \only<3->{\node[det, above=of a,fill=black!15,xshift=-1.5cm] (e) {$e$} ; }
    \node[det, below=of t,xshift=1.5cm] (r) {$r$} ; %
    \only<1-3>{\node[det, above=of e,xshift=1.5cm] (c) {$c$} ; }
    \only<4->{\node[det, above=of e, fill=black!15, xshift=1.5cm] (c) {$c$} ; }

    \edge {c} {e,t};
    \edge {a} {l};
    \edge {t,e} {a};
    \edge {t} {r};


    \only<1-2>{
    \node[const, left= of c, xshift=-0.1cm] (cpd_c) {
    \begin{tabular}{|c|c|}
        \hline
        $c^0$ & $c^1$ \\ \hline
        $0.200$ & $0.800$  \\ \hline
    \end{tabular}
    };
    \node[const, above= of cpd_c] (n_c) {$P(\text{Ciudad})$};
    }
    \only<3>{
    \node[const, left= of c, xshift=-0.1cm] (cpd_c) {
    \begin{tabular}{|c|c|}
        \hline
        $c^0$ & $c^1$ \\ \hline
        $0.208$ & $0.792$  \\ \hline
    \end{tabular}
    };
    \node[const, above= of cpd_c] (n_c) {$P(\text{Ciudad}|\text{Entradera}=0)$};
    }
    \only<4>{
    \node[const, left= of c, xshift=-0.1cm] (cpd_c) {
    \begin{tabular}{|c|c|}
        \hline
        $c^0$ & $c^1$ \\ \hline
        $1.000$ & $0.000$  \\ \hline
    \end{tabular}
    };
    \node[const, above= of cpd_c] (n_c) {$\phantom{P|}\text{Ciudad}=0\phantom{P}$};
    }

    \only<1>{
    \node[const,  left= of e, yshift=-0.3cm, xshift=-0.1cm] (cpd_e) {
    \begin{tabular}{|c|c|c|}
        \hline
       & $e^0$ & $e^1$ \\ \hline
      $c^0$ & $0.999$ & $0.001$  \\ \hline
      $c^1$ & $0.95$ & $0.05$  \\ \hline
    \end{tabular}
    };
    \node[const, above= of cpd_e] (n_e) {$P(\text{Entradera}|\text{Ciudad})$};
    }
    \only<2>{
    \node[const,  left= of e, yshift=-0.3cm, xshift=-0.1cm] (cpd_e) {
    \begin{tabular}{|c|c|}
        \hline
        $e^0$ & $e^1$ \\ \hline
       $0.960$ & $0.040$  \\ \hline
    \end{tabular}
    };
    \node[const, above= of cpd_e] (n_e) {$P(\text{Entradera})$};
    }
    \only<3-4>{
    \node[const,  left= of e, yshift=-0.3cm, xshift=-0.1cm] (cpd_e) {
    \begin{tabular}{|c|c|}
        \hline
        $e^0$ & $e^1$ \\ \hline
       $1.000$ & $0.000$  \\ \hline
    \end{tabular}
    };
    \node[const, above= of cpd_e] (n_e) {$\text{Entradera}=0$};
    }

    \only<1>{
    \node[const, right= of t, yshift=0.9cm, xshift=0.1cm] (cpd_t) {
    \begin{tabular}{|c|c|c|}
        \hline
       & $t^0$ & $t^1$ \\ \hline
      $c^0$ & $0.99$ & $0.01$  \\ \hline
     $c^1$  & $0.95$ & $0.05$  \\ \hline
    \end{tabular}
    };
    \node[const, above= of cpd_t] (n_t) {$P(\text{Terremoto}|\text{Ciudad})$};
    }
    \only<2>{
    \node[const, right= of t, yshift=0.6cm, xshift=0.1cm] (cpd_t) {
    \begin{tabular}{|c|c|}
        \hline
        $t^0$ & $t^1$ \\ \hline
       $0.958$ & $0.042$  \\ \hline
    \end{tabular}
    };
    \node[const, above= of cpd_t] (n_t) {$P(\text{Terremoto})$};
    }
    \only<3>{
    \node[const, right= of t, yshift=0.6cm, xshift=0.1cm] (cpd_t) {
    \begin{tabular}{|c|c|}
        \hline
        $t^0$ & $t^1$ \\ \hline
       $0.959$ & $0.041$  \\ \hline
    \end{tabular}
    };
    \node[const, above= of cpd_t] (n_t) {$P(\text{Terremoto}|\text{Entradera}=0)$};
    }
    \only<4>{
    \node[const, right= of t, yshift=0.6cm, xshift=0.1cm] (cpd_t) {
    \begin{tabular}{|c|c|}
        \hline
        $t^0$ & $t^1$ \\ \hline
       $0.990$ & $0.010$  \\ \hline
    \end{tabular}
    };
    \node[const, above= of cpd_t] (n_t) {$P(\text{Terremoto}|\text{Ciudad}=0)$};
    }


    \only<1>{
    \node[const, left= of a, yshift=-1.6cm, xshift=-0.5cm] (cpd_a) {
    \begin{tabular}{|c|c|c|}
        \hline
        & $a^0$ & $a^1$ \\ \hline
       ($e^0, t^0$) & $0.99$ & $0.01$  \\ \hline
       ($e^1, t^0$) & $0.01$ & $0.99$  \\ \hline
       ($e^0, t^1$) & $0.01$ & $0.99$  \\ \hline
       ($e^1, t^1$) & $0.0001$ & $0.9999$  \\ \hline
    \end{tabular}
    };
    \node[const, above= of cpd_a] (n_a) {$P(\text{Alarma}|\text{Entradera},\text{Terremoto})$};
    }
    \only<2>{
    \node[const, left= of a, yshift=-0.6cm, xshift=-0.5cm] (cpd_a) {
    \begin{tabular}{|c|c|}
        \hline
         $a^0$ & $a^1$ \\ \hline
        $0.911$ & $0.089$  \\ \hline
    \end{tabular}
    };
    \node[const, above= of cpd_a] (n_a) {$P(\text{Alarma})$};
    }
    \only<3>{
    \node[const, left= of a, yshift=-0.6cm, xshift=-0.5cm] (cpd_a) {
    \begin{tabular}{|c|c|}
        \hline
         $a^0$ & $a^1$ \\ \hline
        $0.949$ & $0.051$  \\ \hline
    \end{tabular}
    };
    \node[const, above= of cpd_a] (n_a) {$P(\text{Alarma}|\text{Entradera}=0)$\hspace{1.5cm}\phantom{.}};
    }
    \only<4>{
    \node[const, left= of a, yshift=-0.6cm, xshift=-0.5cm] (cpd_a) {
    \begin{tabular}{|c|c|}
        \hline
         $a^0$ & $a^1$ \\ \hline
        $0.980$ & $0.020$  \\ \hline
    \end{tabular}
    };
    \node[const, above= of cpd_a] (n_a) {$P(\text{Alarma}|\text{Entradera}=0, \text{Ciudad}=0)$\hspace{3cm}\phantom{.}};
    }

    \only<1>{
    \node[const, right= of r, yshift=0.2cm, xshift=0.1cm] (cpd_r) {
    \begin{tabular}{|c|c|c|}
        \hline
        & \, $r^0$ \, & \, $r^1$ \,  \\ \hline
       $(t^0)$ & $0.99$ & $0.01$   \\ \hline
       $(t^1)$ & $0.01$ & $0.99$   \\ \hline
    \end{tabular}
    };
    \node[const, above= of cpd_r] (n_r) {$P(\text{Redes}|\text{Terremoto})$};
    }
    \only<2>{
    \node[const, right= of r, yshift=0.2cm, xshift=0.1cm] (cpd_r) {
    \begin{tabular}{|c|c|}
        \hline
        \, $r^0$ \, & \, $r^1$ \,  \\ \hline
       $0.948$ & $0.052$   \\ \hline
    \end{tabular}
    };
    \node[const, above= of cpd_r] (n_r) {$P(\text{Redes})$};
    }
    \only<3>{
    \node[const, right= of r, yshift=0.2cm, xshift=0.1cm] (cpd_r) {
    \begin{tabular}{|c|c|}
        \hline
        \, $r^0$ \, & \, $r^1$ \,  \\ \hline
       $0.949$ & $0.051$   \\ \hline
    \end{tabular}
    };
    \node[const, above= of cpd_r] (n_r) {$P(\text{Redes}|\text{Entradera}=0)$};
    }
    \only<4>{
    \node[const, right= of r, yshift=0.2cm, xshift=0.1cm] (cpd_r) {
    \begin{tabular}{|c|c|}
        \hline
        \, $r^0$ \, & \, $r^1$ \,  \\ \hline
       $0.980$ & $0.020$   \\ \hline
    \end{tabular}
    };
    \node[const, above= of cpd_r] (n_r) {$P(\text{Redes}|\text{Ciudad}=0)$};
    }

    \only<1>{
    \node[const, right= of l, yshift=-0.4cm,xshift=0.1cm] (cpd_l) {
    \begin{tabular}{|c|c|c|}
        \hline
        & \, $l^0$ \, & \, $l^1$ \,  \\ \hline
       $(a^0)$ & $0.99$ & $0.01$   \\ \hline
       $(a^1)$ & $0.01$ & $0.99$   \\ \hline
    \end{tabular}
    };
    \node[const, above= of cpd_l] (n_l) {$P(\text{Llamada}|\text{Alarma})$};
    }
    \only<2>{
    \node[const, right= of l, yshift=-0.4cm,xshift=0.1cm] (cpd_l) {
    \begin{tabular}{|c|c|}
        \hline
         \, $l^0$ \, & \, $l^1$ \,  \\ \hline
        $0.903$ & $0.097$   \\ \hline
    \end{tabular}
    };
    \node[const, above= of cpd_l] (n_l) {$P(\text{Llamada})$};
    }
    \only<3>{
    \node[const, right= of l, yshift=-0.4cm,xshift=0.1cm] (cpd_l) {
    \begin{tabular}{|c|c|}
        \hline
         \, $l^0$ \, & \, $l^1$ \,  \\ \hline
        $0.940$ & $0.060$   \\ \hline
    \end{tabular}
    };
    \node[const, above= of cpd_l] (n_l) {$\phantom{.}\hspace{1.5cm}P(\text{Llamada}|\text{Entradera}=0)$};
    }
    \only<4>{
    \node[const, right= of l, yshift=-0.4cm,xshift=0.1cm] (cpd_l) {
    \begin{tabular}{|c|c|}
        \hline
         \, $l^0$ \, & \, $l^1$ \,  \\ \hline
        $0.970$ & $0.030$   \\ \hline
    \end{tabular}
    };
    \node[const, above= of cpd_l] (n_l) {$\phantom{.}\hspace{3.6cm}P(\text{Llamada}|\text{Entradera}=0, \text{Ciudad}=0)$};
    }

    \node[invisible, left=of e, xshift=-5.5cm] (il) {};
    \node[invisible, right=of r, xshift=4.8cm] (ir) {};
 }
\end{textblock}

\end{frame}


\begin{frame}[plain]
\begin{textblock}{160}(0,4)
 \centering \LARGE Controles \only<1-6>{buenos}\only<7-12,16>{malos}\only<13-15>{neutrales} \\
 \large \only<12>{Sesgo de selección}\only<13-14>{Mejoran precisión}\only<15>{Reducen precisión}
 \end{textblock}
 \vspace{0.75cm}


\only<1>{
\begin{textblock}{140}(10,30) \centering
\tikz{
    \node[latent, fill=red!30] (z) {$z$} ; %
    \node[latent, fill=black!15 , below=of z, xshift=-1.5cm] (x) {$x$} ; %
    \node[latent, fill=black!15, below=of z, xshift=1.5cm] (y) {$y$} ; %

    \phantom{\node[latent, right=of y] (il) {$l$} ; }
    \phantom{\node[latent, left=of x] (ir) {$r$} ; }

    \edge {z} {x,y};
    \edge {x} {y};
}
\end{textblock}
}


\only<2>{
\begin{textblock}{140}(10,30) \centering
\tikz{
    \node[latent] (u) {$u$} ; %

    \node[latent, fill=red!30,  below=of u, xshift=-1cm,yshift=0.6cm] (z) {$z$} ; %
    \node[latent, fill=black!15 , below=of z, xshift=-1cm,yshift=0.6cm] (x) {$x$} ; %
    \node[latent, fill=black!15, right=of x, xshift=2cm] (y) {$y$} ; %

    \phantom{\node[latent, right=of y] (il) {$l$} ; }
    \phantom{\node[latent, left=of x] (ir) {$r$} ; }


    \edge {u} {z};
    \edge {z} {x};
    \edge {u,x} {y};
}
\end{textblock}
}

\only<3>{
\begin{textblock}{140}(10,30) \centering
\tikz{
    \node[latent] (u) {$u$} ; %

    \node[latent, fill=red!30,  below=of u, xshift=1cm,yshift=0.6cm] (z) {$z$} ; %
    \node[latent, fill=black!15, below=of z, xshift=1cm, yshift=0.6cm] (y) {$y$} ; %
    \node[latent, fill=black!15 , left=of y, xshift=-2cm] (x) {$x$} ; %

    \phantom{\node[latent, right=of y] (il) {$l$} ; }
    \phantom{\node[latent, left=of x] (ir) {$r$} ; }

    \edge {u} {z,x};
    \edge {z,x} {y};
}
\end{textblock}
}

\only<4>{
\begin{textblock}{140}(10,30) \centering
\tikz{
    \node[latent] (u) {$u$} ; %

    \node[latent, fill=red!30,  below=of u, xshift=1cm,yshift=0.6cm] (z) {$z$} ; %
    \node[latent, below=of z, xshift=1cm, yshift=0.6cm] (m) {$m$} ; %
    \node[latent, fill=black!15 , left=of m, xshift=-2cm] (x) {$x$} ; %
    \node[latent, fill=black!15, right=of m] (y) {$y$} ; %

    \phantom{\node[latent, left=of x] (ir) {$r$} ; }

    \edge {u} {z,x};
    \edge {z,x} {m};
    \edge {m} {y};
}
\end{textblock}
}

\only<5>{
\begin{textblock}{140}(10,30) \centering
\tikz{
    \node[latent] (u) {$u$} ; %

    \node[latent, fill=red!30,  below=of u, xshift=-1cm,yshift=0.6cm] (z) {$z$} ; %
    \node[latent, fill=black!15 , below=of z, xshift=-1cm,yshift=0.6cm] (x) {$x$} ; %
    \node[latent, right=of x, xshift=2cm] (m) {$m$} ; %

    \node[latent, fill=black!15, right=of m] (y) {$y$} ;
    \phantom{\node[latent, left=of x] (ir) {$r$} ; }


    \edge {u} {m,z};
    \edge {z} {x};
    \edge {x} {m};
    \edge {m} {y};
}
\end{textblock}
}


\only<6>{
\begin{textblock}{140}(10,30) \centering
\tikz{
    \node[latent, fill=red!30] (z) {$z$} ; %
    \node[latent, fill=black!15 , below=of z, xshift=-1.5cm] (x) {$x$} ; %
    \node[latent, below=of z, xshift=1.5cm] (m) {$m$} ; %

    \node[latent, fill=black!15, right=of m] (y) {$y$} ;
    \phantom{\node[latent, left=of x] (ir) {$r$} ; }

    \edge {x,z} {m};
    \edge {z} {x};
    \edge {m} {y};

}
\end{textblock}
}


\only<7>{
\begin{textblock}{140}(10,30) \centering
\tikz{
    \node[latent, fill=red!30] (z) {$z$} ; %
    \node[latent, fill=black!15 , below=of z, xshift=-1.5cm] (x) {$x$} ; %
    \node[latent, below=of z, xshift=1.5cm] (m) {$m$} ; %

    \node[latent, above=of x] (ux) {$u_1$} ; %
    \node[latent, above=of m] (um) {$u_2$} ; %


    \node[latent, fill=black!15, right=of m] (y) {$y$} ;
    \phantom{\node[latent, left=of x] (ir) {$r$} ; }

    \edge {um,ux} {z};
    \edge {um,x} {m};
    \edge {ux} {x};
    \edge {m} {y};
}
\end{textblock}
}

\only<8>{
\begin{textblock}{140}(10,30) \centering
\tikz{
    \phantom{\node[latent] (ia) {$a$} ; }
    \node[latent, fill=black!15 , below=of ia, xshift=-1.5cm] (x) {$x$} ; %

    \node[latent, fill=black!15 , below=of ia, xshift=-1.5cm] (x) {$x$} ; %
    \node[latent, fill=red!30, below=of ia] (z) {$z$} ;

    \node[latent, fill=black!15, below=of ia, xshift=1.5cm] (y) {$y$} ;

    \phantom{\node[latent, right=of y] (il) {$l$} ; }
    \phantom{\node[latent, left=of x] (ir) {$r$} ; }

    \edge {x} {z};
    \edge {z} {y};
}
\end{textblock}
}


\only<9>{
\begin{textblock}{140}(10,30) \centering
\tikz{
    \phantom{\node[latent] (ia) {$a$} ; }
    \node[latent, fill=black!15 , below=of ia, xshift=-1.5cm] (x) {$x$} ; %

    \node[latent, fill=black!15 , below=of ia, xshift=-1.5cm] (x) {$x$} ; %
    \node[latent, below=of ia] (m) {$m$} ;

    \node[latent, fill=red!30, below=of m] (z) {$z$} ;
    \node[latent, fill=black!15, below=of ia, xshift=1.5cm] (y) {$y$} ;

    \phantom{\node[latent, right=of y] (il) {$l$} ; }
    \phantom{\node[latent, left=of x] (ir) {$r$} ; }

    \edge {x} {m};
    \edge {m} {z,y};

}
\end{textblock}
}

\only<10>{
\begin{textblock}{140}(10,30) \centering
\tikz{
    \phantom{\node[latent] (ia) {$a$} ; }
    \node[latent, fill=black!15 , below=of ia, xshift=-1.5cm] (x) {$x$} ; %

    \node[latent, fill=black!15 , below=of ia, xshift=-1.5cm] (x) {$x$} ; %
    \phantom{\node[latent, below=of ia] (m) {$m$} ;}

    \node[latent, fill=red!30, below=of m] (z) {$z$} ;
    \node[latent, fill=black!15, below=of ia, xshift=1.5cm] (y) {$y$} ;

    \phantom{\node[latent, right=of y] (il) {$l$} ; }
    \phantom{\node[latent, left=of x] (ir) {$r$} ; }

    \edge {x} {y,z};
    \edge {y} {z};

}
\end{textblock}
}

\only<11>{
\begin{textblock}{140}(10,30) \centering
\tikz{
    \node[latent] (ia) {$u$} ;
    \node[latent, fill=black!15 , below=of ia, xshift=-1.5cm] (x) {$x$} ; %

    \node[latent, fill=black!15 , below=of ia, xshift=-1.5cm] (x) {$x$} ; %
    \phantom{\node[latent, below=of ia] (m) {$m$} ;}



    \node[latent, fill=red!30, below=of m] (z) {$z$} ;
    \node[latent, fill=black!15, below=of ia, xshift=1.5cm] (y) {$y$} ;

    \phantom{\node[latent, right=of y] (il) {$l$} ; }
    \phantom{\node[latent, left=of x] (ir) {$r$} ; }

    \edge {x} {y,z};
    \edge {ia} {z,y};

}
\end{textblock}
}

\only<12>{
\begin{textblock}{140}(10,30) \centering
\tikz{
    \phantom{\node[latent] (ia) {$u$} ;}
    \node[latent, fill=black!15 , below=of ia, xshift=-1.5cm] (x) {$x$} ; %

    \node[latent, fill=black!15 , below=of ia, xshift=-1.5cm] (x) {$x$} ; %
    \phantom{\node[latent, below=of ia] (m) {$m$} ;}



    \node[latent, fill=red!30, below=of m] (z) {$z$} ;
    \node[latent, fill=black!15, below=of ia, xshift=1.5cm] (y) {$y$} ;

    \phantom{\node[latent, right=of y] (il) {$l$} ; }
    \phantom{\node[latent, left=of x] (ir) {$r$} ; }

    \edge {x} {y};
    \edge {y} {z};
}
\end{textblock}
}

\only<13>{
\begin{textblock}{140}(10,30) \centering
\tikz{
    \phantom{\node[latent] (ia) {$u$} ;}
    \node[latent, fill=black!15 , below=of ia, xshift=-1.5cm] (x) {$x$} ; %

    \node[latent, fill=black!15 , below=of ia, xshift=-1.5cm] (x) {$x$} ; %
    \phantom{\node[latent, below=of ia] (m) {$m$} ;}



    \node[latent, fill=red!30, right=of ia,xshift=1.5cm] (z) {$z$} ;
    \node[latent, fill=black!15, below=of ia, xshift=1.5cm] (y) {$y$} ;

    \phantom{\node[latent, right=of y] (il) {$l$} ; }
    \phantom{\node[latent, left=of x] (ir) {$r$} ; }

    \edge {x} {y};
    \edge {z} {y};
}
\end{textblock}
}


\only<14>{
\begin{textblock}{140}(10,30) \centering
\tikz{
    \node[latent, fill=red!30] (ia) {$z$} ;
    \node[latent, fill=black!15 , below=of ia, xshift=-1.5cm] (x) {$x$} ; %

    \node[latent, fill=black!15 , below=of ia, xshift=-1.5cm] (x) {$x$} ; %
    \node[latent, below=of ia] (m) {$m$} ;

    \node[latent, fill=black!15, below=of ia, xshift=1.5cm] (y) {$y$} ;

    \phantom{\node[latent, right=of y] (il) {$l$} ; }
    \phantom{\node[latent, left=of x] (ir) {$r$} ; }

    \edge {x} {m};
    \edge {m} {y};
    \edge {ia} {m};
}
\end{textblock}
}


\only<15>{
\begin{textblock}{140}(10,30) \centering
\tikz{
    \phantom{\node[latent] (ia) {$u$} ;}
    \node[latent, fill=black!15 , below=of ia, xshift=-1.5cm] (x) {$x$} ; %

    \node[latent, fill=black!15 , below=of ia, xshift=-1.5cm] (x) {$x$} ; %
    \phantom{\node[latent, below=of ia] (m) {$m$} ;}



    \node[latent, fill=red!30, left=of ia,xshift=-1.5cm] (z) {$z$} ;
    \node[latent, fill=black!15, below=of ia, xshift=1.5cm] (y) {$y$} ;

    \phantom{\node[latent, right=of y] (il) {$l$} ; }
    \phantom{\node[latent, left=of x] (ir) {$r$} ; }

    \edge {x} {y};
    \edge {z} {x};
}
\end{textblock}
}


\only<16>{
\begin{textblock}{140}(10,30) \centering
\tikz{
    \node[latent] (ia) {$u$} ;
    \node[latent, fill=black!15 , below=of ia, xshift=-1.5cm] (x) {$x$} ; %

    \node[latent, fill=black!15 , below=of ia, xshift=-1.5cm] (x) {$x$} ; %
    \phantom{\node[latent, below=of ia] (m) {$m$} ;}



    \node[latent, fill=red!30, left=of ia,xshift=-1.5cm] (z) {$z$} ;
    \node[latent, fill=black!15, below=of ia, xshift=1.5cm] (y) {$y$} ;

    \phantom{\node[latent, right=of y] (il) {$l$} ; }
    \phantom{\node[latent, left=of x] (ir) {$r$} ; }

    \edge {ia} {x,y};
    \edge {x} {y};
    \edge {z} {x};
}
\end{textblock}
}




\end{frame}

 \begin{frame}[plain]
\begin{textblock}{160}(0,4)
 \centering \LARGE Estimación de efecto causal\\
 \large \only<3->{Modelo lineales}
 \end{textblock}
 \vspace{0.75cm}

 \only<1-3>{
\begin{textblock}{160}(0,24) \centering
\tikz{
  \node[latent] (z1) {$z_1$} ;
  \node[latent, below=of z1] (w1) {$w_1$} ;
  \only<1>{\node[latent, right=of w1] (z3) {$z_3$} ;}
  \only<2->{\node[latent, fill=red!30, right=of w1] (z3) {$z_3$} ;}
  \only<1>{\node[latent, right=of z3] (w2) {$w_2$} ;}
  \only<2->{\node[latent, fill=red!30, right=of z3] (w2) {$w_2$} ;}
  \node[latent, above=of w2] (z2) {$z_2$} ;
  \node[latent, fill=black!15, below=of w1] (x) {$x$} ;
  \node[latent, below=of z3] (w3) {$w_3$} ;
  \node[latent, fill=black!15, below=of w2] (y) {$y$} ;

  \onslide<3>{
    \node[const, below=of z1, xshift=-0.2cm, yshift=-0.15cm] (z1_w1) {$6$};
    \node[const, below=of z2, xshift=0.2cm, yshift=-0.15cm] (z2_w2) {$5$};
    \node[const, above=of z3, xshift=-0.6cm, yshift=0.4cm] (z3_z1) {$-4$};
    \node[const, above=of z3, xshift=0.63cm, yshift=0.43cm] (z2_z1) {$3$};
    \node[const, below=of z3, xshift=-1cm, yshift=-0.1cm] (z2_x) {$2$};
    \node[const, below=of z3, xshift=1.05cm, yshift=-0.1cm] (z2_y) {$-1$};
    \node[const, below=of w1, xshift=-0.35cm, yshift=-0.15cm] (w1_x) {$-1$};
    \node[const, below=of w2, xshift=0.2cm, yshift=-0.15cm] (w2_y) {$1$};
    \node[const, below=of x, xshift=0.8cm, yshift=0.3cm] (x_w3) {$2$};
    \node[const, below=of w3, xshift=0.8cm, yshift=0.3cm] (w3_y) {$-1$};
  }

  \edge {w3,z3,w2} {y}
  \edge {x} {w3}
  \edge {z1} {w1}
  \edge {z2} {w2}
  \edge {z1,z2} {z3}
  \edge {w1,z3} {x}

}
\end{textblock}
}

\only<4>{
\begin{textblock}{120}(20,18) \centering
$y \sim c_0 + c_1\,x + c_2\,z_3 + c_3\,w_2$

\includegraphics[width=0.7\textwidth]{figuras/pdf/controles-modeloComplejos.pdf}
\end{textblock}
}



% \begin{textblock}{160}(0,24) \centering
%  \tikz{
%     % MODELO DE DISCRIMINACION LABORAL
%     \node[det] (m) {$m$} ; % Mujer
%     \node[latent, right=of m] (d) {$d$} ; % Discriminacion
%     \node[det, right=of d,yshift=1.5cm] (o) {$o$} ; % Cargo
%     \node[latent, right=of d,yshift=-1.5cm] (s) {$s$} ; % Sueldo
%     \node[latent, right=of d, xshift = 1.8cm] (a) {$a$} ; % Habilidad
%
%     \edge {m} {d};
%     \edge {d} {o,s};
%     \edge {o} {s};
%     \edge {a} {o,s};
%     }
% \end{textblock}


 \end{frame}


\begin{frame}[plain,noframenumbering]
\centering \vspace{0.5cm}
\includegraphics[width=1\textwidth]{../../auxiliar/static/BP.png}
\end{frame}

%
% \begin{frame}[plain]
% \begin{textblock}{96}(0,6.5)\centering
% {\transparent{0.9}\includegraphics[width=0.8\textwidth]{../../auxiliar/static/inti.png}}
% \end{textblock}
%
% \begin{textblock}{160}(96,5.5)
% \includegraphics[width=0.35\textwidth]{../../auxiliar/static/pachacuteckoricancha}
% \end{textblock}
% \end{frame}





\end{document}



