\newif\ifen
\newif\ifes
\newif\iffr
\newcommand{\fr}[1]{\iffr#1 \fi}
\newcommand{\En}[1]{\ifen#1\fi}
\newcommand{\Es}[1]{\ifes#1\fi}
\estrue
\documentclass[shownotes,aspectratio=169]{beamer}

\usepackage{siunitx}
\input{../../auxiliar/tex/diapo_encabezado.tex}
\input{../../auxiliar/tex/tikzlibrarybayesnet.code.tex}
 \mode<presentation>
 {
 %   \usetheme{Madrid}      % or try Darmstadt, Madrid, Warsaw, ...
 %   \usecolortheme{default} % or try albatross, beaver, crane, ...
 %   \usefonttheme{serif}  % or try serif, structurebold, ...
  \usetheme{Antibes}
  \setbeamertemplate{navigation symbols}{}
 }
\estrue
\usepackage{todonotes}
\setbeameroption{show notes}
%
\newcommand{\gray}{\color{black!55}}

\usepackage{mdframed}
\usepackage{listings}
\lstset{
  aboveskip=3mm,
  belowskip=3mm,
  showstringspaces=true,
  columns=flexible,
  basicstyle={\ttfamily},
  breaklines=true,
  breakatwhitespace=true,
  tabsize=4,
  showlines=true
}


\begin{document}

\color{black!85}
\large


\begin{frame}[plain,noframenumbering]

\begin{textblock}{160}(0,0)
\includegraphics[width=1\textwidth]{../../auxiliar/static/peligro_predador}
\end{textblock}

\begin{textblock}{160}(127,67)
\LARGE \textcolor{black!5}{\fontsize{22}{0}\selectfont \textbf{Inferencia  \\[-0.1cm] \hspace{0.5cm} causal}}
\end{textblock}

\begin{textblock}{55}(2,3)
\begin{turn}{0}
\parbox{15cm}{\small
\textcolor{black!95}{Flujos de inferencia en modelos causales. Efecto de}\\
\textcolor{black!95}{las intervenciones en modelos causales. Conclusiones} \\
\textcolor{black!95}{causales a partir de datos observables. Identificación} \\
\textcolor{black!95}{de modelo causal.} \\
\normalsize\textcolor{black!95}{Unidad 2} \\
}
\end{turn}
\end{textblock}

\end{frame}


\begin{frame}[plain]

\centering
\LARGE

¿La aplicación estricta de la probabilidad  \\

produce sobreajuste (\emph{overfitting})?





\end{frame}


\begin{frame}[plain]
\begin{textblock}{160}(0,4)
\centering  \LARGE Modelo lineal (bayesiano)
\end{textblock}

\vspace{1.5cm}

\begin{equation*}
\begin{split}
y & = w_0 + w_1 x \\[0.2cm]
p(t | x, \bm{w}, \beta ) &= \N(t \,|\, y, \beta^2) \\[0.6cm]
\onslide<2->{
p(w_0) &= \N(w_0 \,|\, 0, \sigma_{0}^2) \\[0cm]
p(w_1) &= \N(w_1 \,|\, 0, \sigma_{1}^2) \\}
\end{split}
\end{equation*}



\end{frame}


\begin{frame}[plain]

\Wider[-3cm]{
 \begin{figure}
\begin{subfigure}[t]{0.32\textwidth}
\onslide<3->{\caption*{Verosimilitud}}
\end{subfigure}
\begin{subfigure}[t]{0.32\textwidth}
\caption*{Priori\onslide<3->{/Posteriori}}
\includegraphics[width=\textwidth]{figuras/pdf/linearRegression_posterior_0.pdf}
\end{subfigure}
\begin{subfigure}[t]{0.32\textwidth}
\onslide<2->{
\caption*{Espacio de datos}
\includegraphics[width=\textwidth]{figuras/pdf/linearRegression_dataSpace_0.pdf}}
\end{subfigure}


\begin{subfigure}[c]{0.32\textwidth}
\onslide<3->{\includegraphics[width=\textwidth]{figuras/pdf/linearRegression_likelihood_1.pdf}}
\end{subfigure}
\begin{subfigure}[c]{0.32\textwidth}
\onslide<3->{\includegraphics[width=\textwidth]{figuras/pdf/linearRegression_posterior_1.pdf}}
\end{subfigure}
\begin{subfigure}[c]{0.32\textwidth}
\onslide<3->{\includegraphics[width=\textwidth]{figuras/pdf/linearRegression_dataSpace_1.pdf}}
\end{subfigure}

\begin{subfigure}[c]{0.32\textwidth}
\onslide<4->{\includegraphics[width=\textwidth]{figuras/pdf/linearRegression_likelihood_2.pdf}}
\end{subfigure}
\begin{subfigure}[c]{0.32\textwidth}
\onslide<4->{\includegraphics[width=\textwidth]{figuras/pdf/linearRegression_posterior_2.pdf}}
\end{subfigure}
\begin{subfigure}[c]{0.32\textwidth}
\onslide<4->{\includegraphics[width=\textwidth]{figuras/pdf/linearRegression_dataSpace_2.pdf}}
\end{subfigure}

\end{figure}
}
\end{frame}



\begin{frame}[plain]
\begin{textblock}{160}(0,4)
\centering \Large Modelos lineales
\end{textblock}
 \vspace{1.25cm}

\begin{equation*}
y(\bm{x},\bm{w}) = \sum_{i=0}^{M-1} w_i \phi_i(\bm{x}) = \bm{w}^T \bm{\phi}(\bm{x})
\end{equation*}

\vspace{0.5cm}
\pause
%
% \begin{equation*}
%  t = y(\vm{x},\vm{w}) + \epsilon  \ \ \ \ \  \text{con} \ \ \  \epsilon \sim \N(0,\beta^{-1})
% \end{equation*}

 \begin{equation*}
p(t \, | \, \bm{x}, \bm{w}, \beta) = \N(t \, | \, y(\bm{x},\bm{w}) , \beta^{-1})
\end{equation*}
\vspace{0.025cm}
\pause

\begin{equation*}
P(\bm{t} | \bm{x}, \bm{w}, \beta) = \prod_{i=1}^n \N(t_i | \bm{w}^T \bm{\phi}(\bm{x}_i) , \beta^{-1}) = \N(\bm{t}|\bm{w}^T \bm{\Phi}, \beta^{-1} \vm{I})
\end{equation*}
\vspace{0.05cm}
\pause

\begin{equation*}
 \bm{\Phi} =
  \begin{pmatrix}
    \phi_0(\bm{x}_1) & \phi_1(\bm{x}_1) & \dots & \phi_{M-1}(\bm{x}_1)\\
    \vdots & \vdots & \ddots & \vdots \\
    \phi_0(\bm{x}_N) & \phi_1(\bm{x}_N) & \dots & \phi_{M-1}(\bm{x}_N)
  \end{pmatrix}
  =
  \begin{pmatrix}
   \bm{\phi}(\vm{x}_1)^T \\
   \vdots \\
   \bm{\phi}(\vm{x}_N)^T \\
  \end{pmatrix}
\end{equation*}


\end{frame}



\begin{frame}[plain]
\begin{textblock}{160}(0,4)
 \centering \Large Solución fecuentista
\end{textblock}
\vspace{1.25cm}


\begin{equation*}
 \underset{\bm{w}}{\text{ max }} P(\bm{t} | \bm{x}, \bm{w}, \beta) = \underset{\bm{w}}{\text{ min }} \sum_{i=1}^{n}  (t_i - \bm{w}^T\bm{\phi}(\bm{x}_i))^2
\end{equation*}

\end{frame}

\begin{frame}[plain]
\begin{textblock}{160}(0,4)
 \Large \centering Soluci\'on completa
\end{textblock}
 \vspace{1cm}

\begin{equation*}
 p(\vm{w}) = N(\vm{w}| \vm{0}, \alpha^{-1} \vm{I})
\end{equation*}

\begin{figure}[H]
    \centering
    \tikz{

    \node[latent, fill=black!100, minimum size=2pt] (x) {} ; %
    \node[const, right=of x] (c_x) {$\vm{X}$};
    \node[latent, fill=black!20, yshift=-1.5cm] (t) {$\bm{t}$} ; %
    \node[latent, fill=black!100, yshift=-1.5cm , xshift=-2cm,minimum size=2pt] (beta)
    {} ; %
    \node[const, above=of beta] (c_beta) {$\beta$};
    \node[latent, fill=black!0, yshift=-1.5cm, xshift=2cm] (w) {$\vm{w}$};
    \node[latent, fill=black!100, xshift=2cm, minimum size=2pt] (alpha) {} ; %
    \node[const, right=of alpha] (c_alpha) {$\alpha$};

    \edge {x,beta,w} {t};
    \edge {alpha} {w};

    \node[invisible, fill=black!0, minimum size=0pt, xshift=-0.52cm] (data_inv) {} ; %

    }
\end{figure}
\end{frame}


\begin{frame}[plain]
\begin{textblock}{160}(0,4)
 \Large \centering Soluci\'on completa
\end{textblock}
 \vspace{0.75cm}

La distribuci\'on \textbf{posteriori} sobre $\vm{w}$ tiene como media

\begin{equation}
 \vm{m}_N = \beta  \vm{S}_N\vm{\Phi}^T \vm{t}
\end{equation}

y como covarianzas

\begin{equation}
 \vm{S}_N^{-1} = \alpha \vm{I} + \beta \vm{\Phi}^T\vm{\Phi}
\end{equation}

\pause

Y la \textbf{evidencia} del modelo es

\begin{equation}
 P(t) = \N(\bm{t}| \vm{0}, \beta^{-1} \vm{I} + \alpha^{-1}\bm{\Phi}\bm{\Phi}^T  )
\end{equation}

\vspace{0.75cm}

\small
\Wider[-8cm]{
\begin{mdframed}[backgroundcolor=black!15]\centering
 Cap\'itulo 2 de Bishop
\end{mdframed}
}

\end{frame}


\begin{frame}[plain]
\begin{textblock}{160}(0,4)
\centering  \Large Regresi\'on lineal (bayesiana)
\end{textblock}

\only<2->{
\begin{textblock}{160}(0,18)
\begin{equation*}
y = \beta_0 + \beta_1 x + \beta_2 x^2 + \dots
\end{equation*}
\end{textblock}
}


\begin{textblock}{80}(0,28)
 \centering
 \onslide<1->{Funci\'on objetivo}
\end{textblock}

\begin{textblock}{80}(80,28)
 \centering
 \onslide<2>{Modelos polinomiales}
\end{textblock}

\begin{textblock}{160}(0,32)
     \centering
       \onslide<1->{\includegraphics[width=0.45\textwidth]{figuras/pdf/model_selection_true_and_sample} }
       \onslide<2>{\includegraphics[width=0.45\textwidth]{figuras/pdf/model_selection_MAP_non-informative} }
\end{textblock}

\end{frame}

\begin{frame}[plain]
\begin{textblock}{160}(0,4)
\centering  \Large Regresi\'on lineal (bayesiana) \\
\large Evaluación de modelo
\end{textblock}



\begin{textblock}{160}(0,18)
\begin{equation*}
y = \beta_0 + \beta_1 x + \beta_2 x^2 + \dots
\end{equation*}
\end{textblock}

\begin{textblock}{80}(0,28)
 \centering
 M\'axima verosimilitud
\end{textblock}

\only<2->{
\begin{textblock}{80}(80,28)
\centering
Aplicación estricta de la probabilidad
\end{textblock}
}

\begin{textblock}{160}(0,29)
     \centering
  \begin{figure}[H]
     \centering
      \begin{subfigure}[b]{0.45\textwidth}
       \includegraphics[width=1\textwidth]{figuras/pdf/model_selection_maxLikelihood}
     \end{subfigure}
     \onslide<2->{
    \begin{subfigure}[b]{0.45\textwidth}
       \includegraphics[width=1\textwidth]{figuras/pdf/model_selection_evidence}
     \end{subfigure}
    }
\end{figure}
\end{textblock}

\end{frame}



\begin{frame}[plain]
\begin{textblock}{160}(0,4)
\centering \LARGE La función de costo epistémica \\
\end{textblock}


\begin{textblock}{160}(0,20)
\begin{equation*}
\underbrace{P(\text{Modelo},\text{\En{Data}\Es{Datos}})}_{\hfrac{\text{\footnotesize\En{Initial belief compatible}\Es{Creencia compatible }}}{\text{\footnotesize \En{with the data}\Es{con los datos}}}} = \underbrace{P(\text{Modelo})}_{\hfrac{\text{\footnotesize\En{Initial intersubjective}\Es{Acuerdo intersubjetivo}}}{\text{\footnotesize\En{agreement}\Es{inicial}}}} \underbrace{P(\text{data}_1 |\text{Modelo})}_{\text{\footnotesize Predic\En{tion}\Es{ción} 1}} \, \underbrace{P(\text{data}_2 | \text{data}_1 , \text{Modelo})}_{\text{\footnotesize Predic\En{tion}\Es{ción} 2}} \dots
\end{equation*}
\end{textblock}

\only<2->{
\begin{textblock}{140}(10,50)\centering
Un 0 (cero) en la secuencia hace falsa la hipótesis para siempre \\

\only<3->{\textbf{Ventaja a favor de las variantes que reducen fluctuaciones}}
\end{textblock}
}


\only<4->{
\begin{textblock}{140}(10,66)\centering
\begin{equation*}
P(\text{data}_1|\text{Modelo}) = \only<5->{\phantom}{\sum_{\text{hipótesis}}}  P(\text{data}_1| \text{hipótesis}, \text{Modelo}) \only<5->{\phantom}{P(\text{hipótesis} | \text{Modelo})}
\end{equation*}
\end{textblock}
}


\end{frame}


\begin{frame}[plain,noframenumbering]
\centering \vspace{0.5cm}
\includegraphics[width=1\textwidth]{../../auxiliar/static/BP.png}
\end{frame}

%
% \begin{frame}[plain]
% \begin{textblock}{96}(0,6.5)\centering
% {\transparent{0.9}\includegraphics[width=0.8\textwidth]{../../auxiliar/static/inti.png}}
% \end{textblock}
%
% \begin{textblock}{160}(96,5.5)
% \includegraphics[width=0.35\textwidth]{../../auxiliar/static/pachacuteckoricancha}
% \end{textblock}
% \end{frame}





\end{document}



