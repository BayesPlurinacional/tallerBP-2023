\newif\ifen
\newif\ifes
\newif\iffr
\newcommand{\fr}[1]{\iffr#1 \fi}
\newcommand{\En}[1]{\ifen#1\fi}
\newcommand{\Es}[1]{\ifes#1\fi}
\estrue
\documentclass[shownotes,aspectratio=169]{beamer}

\usepackage{siunitx}

\usepackage{ragged2e} %\justifying
\usepackage{paracol}
\usepackage[utf8]{inputenc} %Para acentos en UTF8 (Prueba: á é í ó ú Á É Í Ó Ú ñ Ñ)
\usepackage{url}
%\usepackage{mathtools}
\usepackage{graphicx}
\usepackage{caption}
\usepackage{float} % para que los gr\'aficos se queden en su lugar con [H]
\usepackage[fleqn]{mathtools} % \coloneqq, flalign
\usepackage{subcaption}
\usepackage{wrapfig}
\usepackage{soul,color} %\st{Hellow world}
\usepackage{xcolor} %\st{Hellow world}
\usepackage[fleqn]{amsmath} %para escribir funci\'on partida
\usepackage{blkarray}
\usepackage{hyperref} % para inlcuir links dentro del texto
\usepackage{tabu} 
\usepackage{comment}
\usepackage{amsfonts} % mathbb{N} -> conjunto de los n\'umeros naturales  
\usepackage{enumerate}
\usepackage{listings}
\usepackage[shortlabels]{enumitem} %  shortlabels option to have compatibility with the enumerate-like scheme for label
\usepackage{framed}
\usepackage{mdframed}
\usepackage{multicol}
\usepackage{transparent} % \transparent{1.0}
\usepackage{bm} 
\usepackage[makeroom]{cancel} % \cancel{} \bcancel{} etc
\usepackage[absolute,overlay]{textpos} %no funciona
\setlength{\TPHorizModule}{1mm} %128mm  mitad: 64 
\setlength{\TPVertModule}{1mm}	%96mm  mitad 48

\newif\ifen
\newif\ifes
\newcommand{\en}[1]{\ifen#1\fi}
\newcommand{\es}[1]{\ifes#1\fi}
\estrue


\usepackage{todonotes}
\setbeameroption{show notes}
\usepackage{rotating}
\usepackage{transparent}


\newcommand{\E}{\en{S}\es{E}}
\newcommand{\A}{\en{E}\es{A}}
\newcommand{\Ee}{\en{s}\es{e}}
\newcommand{\Aa}{\en{e}\es{a}}

\hypersetup{
    colorlinks=true,
    linkcolor={red!50!black},
    citecolor={blue!35!black},
    urlcolor={blue!35!black}
}

\newcommand\hfrac[2]{\genfrac{}{}{0pt}{}{#1}{#2}} %\frac{}{} sin la linea del medio

\newcommand{\indep}{\perp \!\!\! \perp}
\newcommand{\N}{\mathcal{N}}
\newcommand{\vm}[1]{\mathbf{#1}}

\newtheorem{midef}{Definition}
\newtheorem{miteo}{Theorem}
\newtheorem{mipropo}{Proposition}

\usefonttheme[onlymath]{serif}


\usepackage{tikz} % Para graficar, por ejemplo bayes networks
%\usetikzlibrary{bayesnet} % Para que ande se necesita copiar el archivo  tikzlibrarybayesnet.code.tex en la misma carpeta

%%%%%%%%%%%%%%%%%%%%%%%%%%%%%%%%%5
%
% Incompatibles con textpos
%
%\usepackage{todonotes}
%\usepackage{tikz} % Para graficar, por ejemplo bayes networks
%
%%%%%%%%%%%%%%%%%%%%%%%%%%%%%%%%%%



\usepackage[absolute,overlay]{textpos} %no funciona
\setlength{\TPHorizModule}{1mm} %128mm  mitad: 64 
\setlength{\TPVertModule}{1mm}	%96mm  mitad 48
% 
% 
\captionsetup[figure]{labelformat=empty}

% 
% http://latexcolor.com/
\definecolor{lightseagreen}{rgb}{0.13, 0.7, 0.6.5}
\definecolor{greenblue}{rgb}{0.1, 0.55, 0.5}
\definecolor{redgreen}{rgb}{0.6, 0.4, 0.}
\definecolor{greenred}{rgb}{0.4, 0.7, 0.}
\definecolor{redblue}{rgb}{0.4, 0., .4}
\definecolor{tangelo}{rgb}{0.98, 0.3, 0.0}
\definecolor{git}{rgb}{0.94, 0.309, 0.2}
% 
\setbeamercolor{structure}{fg=greenblue}


%http://latexcolor.com/
\definecolor{azul}{rgb}{0.36, 0.54, 0.66}
\definecolor{rojo}{rgb}{0.7, 0.2, 0.116}
\definecolor{rojopiso}{rgb}{0.8, 0.25, 0.17}
\definecolor{verdeingles}{rgb}{0.12, 0.5, 0.17}
\definecolor{ubuntu}{rgb}{0.44, 0.16, 0.39}
\definecolor{debian}{rgb}{0.84, 0.04, 0.33}
\definecolor{dkgreen}{rgb}{0,0.6,0}
\definecolor{gray}{rgb}{0.5,0.5,0.5}
\definecolor{mauve}{rgb}{0.58,0,0.82}




\newcommand\Wider[2][3em]{%
\makebox[\linewidth][c]{%
  \begin{minipage}{\dimexpr\textwidth+#1\relax}
  \raggedright#2
  \end{minipage}%
  }%
}

\newenvironment{ejercicio}[1]{
% \setbeamercolor{block title}{bg=tangelo, fg=white}
\begin{exampleblock}{#1}
}{
\end{exampleblock}
}

\newenvironment{resumen}[1]{
\setbeamercolor{block title}{bg=git, fg=white}
\begin{block}{#1}
}{
\end{block}
}

\newenvironment{comando}{
\setbeamercolor{block body}{bg=git, fg=white}
\begin{block}{}
\begin{center}
\LARGE
\begin{texttt}
}{
\end{texttt}
\end{center}
\end{block}
}



% tikzlibrary.code.tex
%
% Copyright 2010-2011 by Laura Dietz
% Copyright 2012 by Jaakko Luttinen
%
% This file may be distributed and/or modified
%
% 1. under the LaTeX Project Public License and/or
% 2. under the GNU General Public License.
%
% See the files LICENSE_LPPL and LICENSE_GPL for more details.

% Load other libraries

%\newcommand{\vast}{\bBigg@{2.5}}
% newcommand{\Vast}{\bBigg@{14.5}}
% \usepackage{helvet}
% \renewcommand{\familydefault}{\sfdefault}

\usetikzlibrary{shapes}
\usetikzlibrary{fit}
\usetikzlibrary{chains}
\usetikzlibrary{arrows}

% Latent node
\tikzstyle{latent} = [circle,fill=white,draw=black,inner sep=1pt,
minimum size=20pt, font=\fontsize{10}{10}\selectfont, node distance=1]
% Observed node
\tikzstyle{obs} = [latent,fill=gray!25]
% Invisible node
\tikzstyle{invisible} = [latent,minimum size=0pt,color=white, opacity=0, node distance=0]
% Constant node
\tikzstyle{const} = [rectangle, inner sep=0pt, node distance=0.1]
%state
\tikzstyle{estado} = [latent,minimum size=8pt,node distance=0.4]
%action
\tikzstyle{accion} =[latent,circle,minimum size=5pt,fill=black,node distance=0.4]


% Factor node
\tikzstyle{factor} = [rectangle, fill=black,minimum size=10pt, draw=black, inner
sep=0pt, node distance=1]
% Deterministic node
\tikzstyle{det} = [latent, rectangle]

% Plate node
\tikzstyle{plate} = [draw, rectangle, rounded corners, fit=#1]
% Invisible wrapper node
\tikzstyle{wrap} = [inner sep=0pt, fit=#1]
% Gate
\tikzstyle{gate} = [draw, rectangle, dashed, fit=#1]

% Caption node
\tikzstyle{caption} = [font=\footnotesize, node distance=0] %
\tikzstyle{plate caption} = [caption, node distance=0, inner sep=0pt,
below left=5pt and 0pt of #1.south east] %
\tikzstyle{factor caption} = [caption] %
\tikzstyle{every label} += [caption] %

\tikzset{>={triangle 45}}

%\pgfdeclarelayer{b}
%\pgfdeclarelayer{f}
%\pgfsetlayers{b,main,f}

% \factoredge [options] {inputs} {factors} {outputs}
\newcommand{\factoredge}[4][]{ %
  % Connect all nodes #2 to all nodes #4 via all factors #3.
  \foreach \f in {#3} { %
    \foreach \x in {#2} { %
      \path (\x) edge[-,#1] (\f) ; %
      %\draw[-,#1] (\x) edge[-] (\f) ; %
    } ;
    \foreach \y in {#4} { %
      \path (\f) edge[->,#1] (\y) ; %
      %\draw[->,#1] (\f) -- (\y) ; %
    } ;
  } ;
}

% \edge [options] {inputs} {outputs}
\newcommand{\edge}[3][]{ %
  % Connect all nodes #2 to all nodes #3.
  \foreach \x in {#2} { %
    \foreach \y in {#3} { %
      \path (\x) edge [->,#1] (\y) ;%
      %\draw[->,#1] (\x) -- (\y) ;%
    } ;
  } ;
}

% \factor [options] {name} {caption} {inputs} {outputs}
\newcommand{\factor}[5][]{ %
  % Draw the factor node. Use alias to allow empty names.
  \node[factor, label={[name=#2-caption]#3}, name=#2, #1,
  alias=#2-alias] {} ; %
  % Connect all inputs to outputs via this factor
  \factoredge {#4} {#2-alias} {#5} ; %
}

% \plate [options] {name} {fitlist} {caption}
\newcommand{\plate}[4][]{ %
  \node[wrap=#3] (#2-wrap) {}; %
  \node[plate caption=#2-wrap] (#2-caption) {#4}; %
  \node[plate=(#2-wrap)(#2-caption), #1] (#2) {}; %
}

% \gate [options] {name} {fitlist} {inputs}
\newcommand{\gate}[4][]{ %
  \node[gate=#3, name=#2, #1, alias=#2-alias] {}; %
  \foreach \x in {#4} { %
    \draw [-*,thick] (\x) -- (#2-alias); %
  } ;%
}

% \vgate {name} {fitlist-left} {caption-left} {fitlist-right}
% {caption-right} {inputs}
\newcommand{\vgate}[6]{ %
  % Wrap the left and right parts
  \node[wrap=#2] (#1-left) {}; %
  \node[wrap=#4] (#1-right) {}; %
  % Draw the gate
  \node[gate=(#1-left)(#1-right)] (#1) {}; %
  % Add captions
  \node[caption, below left=of #1.north ] (#1-left-caption)
  {#3}; %
  \node[caption, below right=of #1.north ] (#1-right-caption)
  {#5}; %
  % Draw middle separation
  \draw [-, dashed] (#1.north) -- (#1.south); %
  % Draw inputs
  \foreach \x in {#6} { %
    \draw [-*,thick] (\x) -- (#1); %
  } ;%
}

% \hgate {name} {fitlist-top} {caption-top} {fitlist-bottom}
% {caption-bottom} {inputs}
\newcommand{\hgate}[6]{ %
  % Wrap the left and right parts
  \node[wrap=#2] (#1-top) {}; %
  \node[wrap=#4] (#1-bottom) {}; %
  % Draw the gate
  \node[gate=(#1-top)(#1-bottom)] (#1) {}; %
  % Add captions
  \node[caption, above right=of #1.west ] (#1-top-caption)
  {#3}; %
  \node[caption, below right=of #1.west ] (#1-bottom-caption)
  {#5}; %
  % Draw middle separation
  \draw [-, dashed] (#1.west) -- (#1.east); %
  % Draw inputs
  \foreach \x in {#6} { %
    \draw [-*,thick] (\x) -- (#1); %
  } ;%
}


 \mode<presentation>
 {
 %   \usetheme{Madrid}      % or try Darmstadt, Madrid, Warsaw, ...
 %   \usecolortheme{default} % or try albatross, beaver, crane, ...
 %   \usefonttheme{serif}  % or try serif, structurebold, ...
  \usetheme{Antibes}
  \setbeamertemplate{navigation symbols}{}
 }
\estrue
\usepackage{todonotes}
\setbeameroption{show notes}
%
\newcommand{\gray}{\color{black!55}}
\usepackage{ulem} % sout
\usepackage{mdframed}
\usepackage{listings}
\lstset{
  aboveskip=3mm,
  belowskip=3mm,
  showstringspaces=true,
  columns=flexible,
  basicstyle={\footnotesize\ttfamily},
  breaklines=true,
  breakatwhitespace=true,
  tabsize=4,
  showlines=true,
}

\begin{document}

\color{black!85}
\large
%
% \begin{frame}[plain,noframenumbering]
%
%
% \begin{textblock}{160}(0,0)
% \includegraphics[width=1\textwidth]{../../auxiliar/static/deforestacion}
% \end{textblock}
%
% \begin{textblock}{80}(18,9)
% \textcolor{black!15}{\fontsize{44}{55}\selectfont Verdades}
% \end{textblock}
%
% \begin{textblock}{47}(85,70)
% \centering \textcolor{black!15}{{\fontsize{52}{65}\selectfont Empíricas}}
% \end{textblock}
%
% \begin{textblock}{80}(100,28)
% \LARGE  \textcolor{black!15}{\rotatebox[origin=tr]{-3}{\scalebox{9}{\scalebox{1}[-1]{$p$}}}}
% \end{textblock}
%
% \begin{textblock}{80}(66,43)
% \LARGE  \textcolor{black!15}{\scalebox{6}{$=$}}
% \end{textblock}
%
% \begin{textblock}{80}(36,29)
% \LARGE  \textcolor{black!15}{\scalebox{9}{$p$}}
% \end{textblock}
%
% %
% %
% % \begin{textblock}{160}(01,81)
% % \footnotesize \textcolor{black!5}{\textbf{\small Seminario ``Acuerdos intersubjetivos''\\
% % Comunidad Bayesiana Plurinacional} \\}
% % \end{textblock}
%
% \end{frame}

%%%%%%%%%%%%%%%%%%%%%%%%%%%%%%%%%%%%%%%%%


\begin{frame}[plain,noframenumbering]

\begin{textblock}{160}(0,-15)
\includegraphics[width=1\textwidth]{../../auxiliar/static/tsimane}
\end{textblock}


% VERSION 2
\begin{textblock}{160}(6,36)
\LARGE \rotatebox[origin=tr]{18}{\textcolor{black!95}{\fontsize{22}{0}\selectfont \textbf{La función}}}
\end{textblock}
\begin{textblock}{160}(41,32)
\LARGE \rotatebox[origin=tr]{23}{\textcolor{black!95}{\fontsize{22}{0}\selectfont \textbf{de}}}
\end{textblock}
\begin{textblock}{160}(50.5,23)
\LARGE \rotatebox[origin=tr]{28}{\textcolor{black!95}{\fontsize{22}{0}\selectfont \textbf{costo}}}
\end{textblock}
\begin{textblock}{160}(68,5.3)
\LARGE \rotatebox[origin=tr]{26}{\textcolor{black!95}{\fontsize{22}{0}\selectfont \textbf{epistémico}}}
\end{textblock}
\begin{textblock}{160}(104,5.5)
\LARGE \rotatebox[origin=tr]{8}{\textcolor{black!95}{\fontsize{22}{0}\selectfont \textbf{-}}}
\end{textblock}
\begin{textblock}{160}(110,3)
\LARGE \rotatebox[origin=tr]{-14}{\textcolor{black!95}{\fontsize{22}{0}\selectfont \textbf{evolutiva}}}
\end{textblock}


\begin{textblock}{55}[0,0](119,22)
\begin{turn}{-57}
\parbox{7cm}{\sloppy\setlength\parfillskip{0pt}
\textcolor{black!0}{\ \ \ \ \ Unidad 5} \\
\small\textcolor{black!5}{\hspace{-0.15cm} Apuestas óptimas.} \\
\small\textcolor{black!5}{\hspace{-0.85cm} Ventajas a favor de la:} \\
\small\textcolor{black!5}{\hspace{-1.45cm} Diversificación (propiedad epistémica)}\\
\small\textcolor{black!5}{\hspace{-1.7cm} Cooperación (propiedad evolutiva)}\\
\small\textcolor{black!5}{ \hspace{-1.75cm}Especialización (propiedad de especiación)} \\
\small\textcolor{black!5}{\hspace{-2cm} Heterogeniedad (propiedad ecológica).\\ }}
\end{turn}
\end{textblock}


\end{frame}



\begin{frame}[plain]
\begin{textblock}{160}(0,4)
\centering \LARGE Toma de decisiones \\
\large Las funciones de costo
\end{textblock}


\only<1->{
\begin{textblock}{160}(0,24) \centering

\Large \only<1>{La función de costo epistémica\phantom{evolutiva}}\only<2->{La función de costo evolutiva\phantom{epistémica}}

\large
\only<1>{\begin{equation*}
\underbrace{P(\text{Hipótesis},\text{\En{Data}\Es{Datos}})}_{\hfrac{\text{\footnotesize\En{Initial belief compatible}\Es{Creencia compatible }}}{\text{\footnotesize \En{with the data}\Es{con los datos}}}} = \underbrace{P(\text{Hipótesis})}_{\hfrac{\text{\footnotesize\En{Initial intersubjective}\Es{Acuerdo inter-}}}{\text{\footnotesize\En{agreement}\Es{subjetivo inicial}}}} \ \underbrace{P(\text{dato}_1 |\text{Hipótesis})}_{\text{\footnotesize Predic\En{tion}\Es{ción} 1}} \, \underbrace{P(\text{dato}_2 | \text{dato}_1 , \text{Hipótesis})}_{\text{\footnotesize Predic\En{tion}\Es{ción} 2}} \dots
\end{equation*}
}\only<2->{\begin{equation*}
\underbrace{\text{P}(\text{Variante},\text{\En{Data}\Es{Datos}})}_{\hfrac{\text{\footnotesize\En{Initial belief compatible}\Es{Tamaño actual}}}{\text{\footnotesize \En{with the data}\Es{de la población}}}} = \underbrace{\text{P}(\text{Variante})}_{\hfrac{\text{\footnotesize\En{Initial intersubjective}\Es{Tamaño inicial}}}{\text{\footnotesize\En{agreement}\Es{de la población}}}} \underbrace{\text{ R}(\text{dato}_1|\text{Variante})}_{\text{\footnotesize Reproducción $\geq 1$}} \, \underbrace{\text{ S}(\text{dato}_2|\text{dato}_1,\text{Variante}) }_{\text{\footnotesize $0 \leq$ Supervivencia $\leq 1$  }} \dots
\end{equation*}
}

\end{textblock}
}



\only<3->{
\begin{textblock}{160}(0,60) \centering \Large
Un 0 en la secuencia de tasas de reproducción y

supervivencia produce una extinción irreversible

\vspace{0.6cm}

\only<4>{
\textbf{¿Cuáles son las variantes que más crecen?}
}

\end{textblock}
}


\end{frame}



\begin{frame}[plain]
\begin{textblock}{160}(0,4)
\centering \LARGE La función de costo evolutiva \\
\large \only<2->{Apuestas de vida}
\end{textblock}


\only<1-3>{
\begin{textblock}{160}(0,24)
\begin{equation*}
\underbrace{\text{P}(\text{Variante},\text{\En{Data}\Es{Datos}})}_{ \only<1-2>{\hfrac{\text{\footnotesize\En{Initial belief compatible}\Es{Tamaño actual}}}{\text{\footnotesize \En{with the data}\Es{de la población}}}}\only<3>{\text{\footnotesize $\omega_t(b)$ recursos finales}}} = \underbrace{\text{P}(\text{Variante})}_{\only<1-2>{\hfrac{\text{\footnotesize\En{Initial intersubjective}\Es{Tamaño inicial}}}{\text{\footnotesize\En{agreement}\Es{de la población}}}}\only<3>{\text{\footnotesize $\omega_0$ $\hfrac{\text{\footnotesize recursos}}{\text{\footnotesize iniciales}}$}}} \underbrace{\text{ R}(\text{dato}_1|\text{Variante})}_{\only<1-2>{\text{\footnotesize Reproducción $\geq 1$}}\only<3>{\text{\footnotesize $b \, Q_c$ Reproducción }}} \, \underbrace{\text{ S}(\text{dato}_2|\text{dato}_1,\text{Variante}) }_{\only<1-2>{\text{\footnotesize $0 \leq$ Supervivencia $\leq 1$  }}\only<3>{\text{\footnotesize $(1-b) Q_s$ Supervivencia}}} \dots
\end{equation*}
\end{textblock}
}



\only<4->{
\begin{textblock}{160}(55,16)
\begin{flalign*}
\omega_2(b) & = \underbrace{\omega_0 \, \overbrace{b \,  Q_c}^{\text{Cara}}}_{\omega_1(b)} \, \overbrace{(1-b) \, Q_s}^{\text{Seca}}
&&
\end{flalign*}
\end{textblock}
}



\only<2-8>{
\begin{textblock}{150}(10,48)
Casa de apuestas paga: \\[0.2cm] \normalsize

\ \ $\bullet$ Por \textbf{Cara}. $Q_c = 3$

\ \ $\bullet$ Por \textbf{Seca}. $Q_s = 1.2$

\vspace{0.3cm} \large

Apuestamos proporciones\\[0.2cm] \normalsize

\ \ $\bullet$ A \textbf{Cara}. $b \in [0,1]$

\ \ $\bullet$ A \textbf{Seca}. $(1-b)$


\end{textblock}
}

\only<9-16>{
\begin{textblock}{74}(2,48) \centering
Regla de la suma \\ \footnotesize
(Predecir integrando todas las hipótesis)

\begin{align*}
 \widehat{\omega}_1(b=0.5) =  & P(\text{Cara}) 1.5 + P(\text{Seca}) 0.6 = \bm{1.05} \\[0.2cm]
\only<10->{\widehat{\omega}_2(0.5) = & P(\text{CC}) 1.5\cdot1.5  + \\ & 2\cdot P(\text{CS}) 1.5\cdot0.6 + \\
 & P(\text{SS}) 0.6\cdot0.6 = \bm{1.05^2}}
\end{align*}

\end{textblock}
}

\only<17->{
\begin{textblock}{74}(2,56) \centering \Large
Los impactos de las caídas son

más fuertes que los de la ganancias
\end{textblock}
}


\only<2-4>{
\begin{textblock}{95}(60,48)
\raggedleft
\includegraphics[width=0.82\textwidth]{../../auxiliar/static/plata-potosi.jpg}
\end{textblock}
}

\only<5-7>{
\begin{textblock}{85}(70,48) \centering
\Large
Ya conocemos bien esta función de costo

¿Nos conviene jugar? \\[0.5cm] \normalsize


\only<6->{
Si apostamos todo a una opción y sale

la contraria, perdemos todos.
}

\vspace{0.5cm}

\only<7->{\large

¿Qué pasa si apostamos $b=0.5$?
}

\end{textblock}
}

\only<8>{
\begin{textblock}{78}(81,40) \centering
\includegraphics[page=1,width=1\textwidth]{figuras/apuestasParalelas.pdf}
\end{textblock}
}
\only<9>{
\begin{textblock}{78}(81,40) \centering
\includegraphics[page=2,width=1\textwidth]{figuras/apuestasParalelas.pdf}
\end{textblock}
}
\only<10>{
\begin{textblock}{78}(81,40) \centering
\includegraphics[page=3,width=1\textwidth]{figuras/apuestasParalelas.pdf}
\end{textblock}
}
\only<11>{
\begin{textblock}{78}(81,40) \centering
\includegraphics[page=4,width=1\textwidth]{figuras/apuestasParalelas.pdf}
\end{textblock}
}
\only<12>{
\begin{textblock}{78}(81,40) \centering
\includegraphics[page=5,width=1\textwidth]{figuras/apuestasParalelas.pdf}
\end{textblock}
}
\only<13>{
\begin{textblock}{78}(81,40) \centering
\includegraphics[page=6,width=1\textwidth]{figuras/apuestasParalelas.pdf}
\end{textblock}
}
\only<14>{
\begin{textblock}{78}(81,40) \centering
\includegraphics[page=9,width=1\textwidth]{figuras/apuestasParalelas.pdf}
\end{textblock}
}
\only<15>{
\begin{textblock}{78}(81,40) \centering
\includegraphics[page=8,width=1\textwidth]{figuras/apuestasParalelas.pdf}
\end{textblock}
}
\only<16->{
\begin{textblock}{78}(81,40) \centering
\includegraphics[page=7,width=1\textwidth]{figuras/apuestasParalelas.pdf}
\end{textblock}
}




\end{frame}



\begin{frame}[plain]
\begin{textblock}{160}(0,4)
\centering \LARGE La función de costo evolutiva \\
\large Tasa de crecimiento temporal
\end{textblock}


\only<1->{
\begin{textblock}{160}(10,12)
\begin{flalign*}
\phantom{\left(\frac{\omega_T(b)}{\omega_0}\right)^{\frac{1}{T}} } \onslide<4->{\lim_{T\rightarrow \infty}} \only<1-4>{\omega_T(b)}\only<5->{\left(\frac{\omega_T(b)}{\omega_0}\right)^{\onslide<6->{\frac{1}{T}}} } & =  \only<-4>{\omega_0} \, (b \,  Q_c)^{\only<1-5>{n_c}\only<6>{\frac{n_c}{T}}\only<7->{\, p\,}} \, ((1-b) \,  Q_s)^{\only<1-5>{n_s}\only<6>{\frac{n_s}{T}}\only<7->{(1-p)}} \only<1-3>{\approx}\only<4->{=} \onslide<1-3>{\text{¿}} \, \only<-4>{\omega_0} \only<-5>{\,} r(b)\only<1-5>{^T} \onslide<1-3>{\text{?}} \phantom{\left(\frac{\omega_T(b)}{\omega_0}\right)^{\frac{1}{T}} } \\
\only<8->{& = \  (\underbrace{0.5 \cdot 3.0}_{1.5}) ^{1/2} \ \ (\underbrace{0.5 \cdot 1.2}_{0.6})^{1/2} \approx 0.949}
&&
\end{flalign*}
\end{textblock}
}


\only<2>{
\begin{textblock}{100}(30,33) \centering
\includegraphics[page=10,width=0.9\textwidth]{figuras/apuestasParalelas.pdf}
\end{textblock}
}
\only<3>{
\begin{textblock}{100}(30,33) \centering
\includegraphics[page=11,width=0.9\textwidth]{figuras/apuestasParalelas.pdf}
\end{textblock}
}
\only<4-7>{
\begin{textblock}{100}(30,33) \centering
\includegraphics[page=12,width=0.9\textwidth]{figuras/apuestasParalelas.pdf}
\end{textblock}
}
\only<8>{
\begin{textblock}{100}(30,33) \centering
\includegraphics[page=13,width=0.9\textwidth]{figuras/apuestasParalelas.pdf}
\end{textblock}
}


\end{frame}



\begin{frame}[plain]
\begin{textblock}{160}(0,4)
\centering \LARGE \only<8->{\phantom{epistémico-}}La función de costo \only<8->{epistémico-}evolutiva \\
\large Comportamiento óptimo a largo plazo
\end{textblock}



\only<1->{
\begin{textblock}{160}(0,18)
\begin{align*}
\frac{r(b)}{r(d)} & = \frac{(b\, \only<1>{Q_c\,}\only<2->{\cancel{Q_c}} )^p \, ((1-b)\, \only<1>{Q_s\,}\only<2->{\bcancel{Q_s}} )^{1-p}}{(d\, \only<1>{Q_c\,}\only<2->{\cancel{Q_c}} )^p \, ((1-d)\,  \only<1>{Q_s\,}\only<2->{\bcancel{Q_s}} )^{1-p}}
\end{align*}
\end{textblock}
}


\only<3>{
\begin{textblock}{160}(0,48) \centering \Large
Para elegir la apuesta óptima,

no importa el pago que ofrezca la casa!!
\end{textblock}
}


\only<4->{
\begin{textblock}{160}(30,44)
\begin{flalign*}
& \underset{b}{\text{ arg max }} \ b^{\,p}  (1-b)^{1-p} \only<5>{= \underset{b}{\text{ arg max }} \ p \, \log b + (1-p) \, \log (1-b)}\only<6->{= \underset{b}{\text{ arg max }} \, \underbrace{p \, \log b + (1-p) \, \log (1-b)}_{-\text{\small Entropía cruzada!}} }
&&
\end{flalign*}
\end{textblock}
}


\only<7>{
\begin{textblock}{160}(0,58) \centering \huge
\begin{equation*}
b^* = p
\end{equation*}
\end{textblock}
}

\only<8->{
\begin{textblock}{160}(0,58) \centering \huge
\begin{equation*}
\underbrace{b^* = p}_{\text{\normalsize Propiedad epistémica}}
\end{equation*}
\end{textblock}
}


\only<9>{
\begin{textblock}{160}(0,83) \centering \normalsize
(Ventaja a favor de la diversificación individual)
\end{textblock}
}


\end{frame}


\begin{frame}[plain]
\begin{textblock}{160}(0,4)
\centering \LARGE El isomorfismo evolución - probabilidad \\
\large \only<1>{Replicator dynamic - Teorema de Bayes}\only<2->{Proporción actual de las poblaciones}
\end{textblock}

\only<1>{
\begin{textblock}{160}(0,28) \centering
\begin{equation*} \footnotesize
\overbrace{P\Big(\hfrac{\text{Hipótesis o}}{\text{Forma de vida}}  \Big| \text{Datos}, \hfrac{\text{Modelo}}{\text{Causal}} \Big)}^{\hfrac{\text{\scriptsize Nueva proporción}}{\text{\scriptsize de la variante}}} = \frac{ \overbrace{P\Big(\text{Datos},  \Big|  \hfrac{\text{Hipótesis o}}{\text{Forma de vida}}  , \hfrac{\text{Modelo}}{\text{Causal}} \Big)}^{\hfrac{\text{\scriptsize Adaptabilidad de la}}{\text{\scriptsize variante a la realidad}}} \overbrace{P\Big(\hfrac{\text{Hipótesis o}}{\text{Forma de vida}} \Big|  \hfrac{\text{Modelo}}{\text{Causal}} \Big)}^{\hfrac{\text{\scriptsize Vieja proporción}}{\text{\scriptsize de la variante}}}}{\underbrace{P\Big(\text{Datos},  \hfrac{\text{\small Modelo}}{\text{\small Causal}} \Big)}_{\hfrac{\text{\scriptsize Proporción}}{\text{\scriptsize sobreviviente}}}}
\end{equation*}
\end{textblock}
}

\only<2-3>{
\begin{textblock}{140}(10,22) \centering
\includegraphics[width=1\textwidth]{../../auxiliar/static/biomassBarOn.png}
\end{textblock}
}

\only<2>{
\begin{textblock}{160}(0,22)
\tikz{
  \node[det, draw=white, minimum size=9cm] (c) {};
  \node[invisible, left=of c, xshift=-6.7cm] (i) {};
  \node[det, right= of i,draw=white, minimum size=1cm, xshift=5.46cm,yshift=0.05cm] (c2) {};
}
\end{textblock}
}


\end{frame}



\begin{frame}[plain]
\begin{textblock}{160}(0,4)
\centering \LARGE La estrategia de la vida \\
\large Transiciones evolutivas mayores
\end{textblock}


\only<1>{
\begin{textblock}{160}(0,28) \centering
\begin{figure}[ht!]
    \centering
  \scalebox{1.2}{
  \tikz{
      \node[accion] (i1) {} ;
      \node[accion, yshift=0.6cm, xshift=0.4cm] (i2) {} ;
      \node[accion, yshift=0.6cm, xshift=-0.4cm] (i3) {} ;
      \node[const, yshift=0.3cm, xshift=0.4cm] (i) {};

      \node[const, yshift=-0.8cm] (ni) {$\hfrac{\text{Individuos}}{\text{solitarios}}$};

      \node[const, yshift=1.2cm, xshift=1.5cm] (m1) {$\hfrac{\text{Formación}}{\text{de grupos}}$};

      \node[const, right=of i, xshift=2cm] (c) {};
      \node[accion, below=of c, yshift=0.35cm, xshift=0.4cm] (c1) {} ;
      \node[accion, above=of c, yshift=-0.35cm, xshift=0.6cm] (c2) {} ;
      \node[accion, above=of c, yshift=-0.35cm, xshift=0.2cm] (c3) {} ;
      \node[const, right=of c, xshift=0.6cm] (cc) {};

      \node[const, right=of ni, xshift=1.3cm] (nc) {$\hfrac{\text{Grupos}}{\text{cooperativos}}$};

      \node[const, right=of m1, xshift=1.2cm] (m2) {$\hfrac{\text{Transición}}{\text{mayor}}$};

      \node[const, right=of cc, xshift=2cm] (t) {};
      \node[accion, below=of t, yshift=0.35cm, xshift=0.4cm] (t1) {} ;
      \node[accion, above=of t, yshift=-0.35cm, xshift=0.6cm] (t2) {} ;
      \node[accion, above=of t, yshift=-0.35cm, xshift=0.2cm] (t3) {} ;

      \node[const, right=of nc, xshift=1.1cm] (nt) {$\hfrac{\text{Unidad de}}{\text{nivel superior}}$};

      \edge {i} {c};
      \edge {cc} {t};

      \plate {transition} {(t1)(t2)(t3)} {}; %
      }
  }
\end{figure}
\end{textblock}
}


\only<2->{
\begin{textblock}{160}(0,22) \centering \Large


La emergencia de unidades cooperativas de

nivel superior es un fenómeno permanente

\vspace{0.8cm} \large

\only<3->{
Nuestra propia vida depende de al menos 4 niveles:
\begin{figure}[H]
\centering
 \begin{subfigure}[b]{0.25\textwidth} \centering
 \onslide<4->{\includegraphics[width=1\linewidth]{../../auxiliar/static/cloroplastos.jpg}
  \caption*{\En{Eukaryotic cells}\Es{Células eucariota}}}
  \end{subfigure}
 \begin{subfigure}[b]{0.23\textwidth} \centering
  \onslide<5->{\includegraphics[width=1\linewidth]{../../auxiliar/static/fotosintesis.jpg}
  \caption*{\En{Organisms}\Es{Organismos}}}
  \end{subfigure}
  \begin{subfigure}[b]{0.235\textwidth} \centering
 \onslide<6->{\includegraphics[width=1\linewidth]{../../auxiliar/static/hormigas2.jpg}
  \caption*{\En{Societies}\Es{Sociedades}}}
 \end{subfigure}
 \begin{subfigure}[b]{0.235\textwidth} \centering
 \onslide<7->{\includegraphics[width=1\linewidth]{../../auxiliar/static/tsimane2.jpg}
  \caption*{\En{Ecosystems}\Es{Ecosistemas}}}
 \end{subfigure}
\end{figure}
}
\end{textblock}
}

\end{frame}


\begin{frame}[plain]
\begin{textblock}{160}(0,4)
\centering \LARGE \only<1>{Sin cooperación}\only<2->{Cooperación} \\
\large \only<7>{Propiedad evolutiva}
\end{textblock}


\only<1>{
\begin{textblock}{160}(0,24)
\centering
  \begin{tabular}{|l|c|c|c|c|c|}
     \hline
         & {\small $\omega_0$} & {\small \  $\Delta$}  & {\small \, $\omega_1(b)$ } & {\small \  $\Delta$}  & {\small \,  $\omega_2(b)$ }  \\ \hline \hline
        A no-coop& $1$ & $1.5$ &  $1.5$ & $0.6$ & $\bm{0.9}$ \\ \hline
        B no-coop & $1$ & $0.6$ & $0.6$ & $1.5$ & $\bm{0.9}$ \\ \hline
\end{tabular}
\end{textblock}
}
\only<2>{
\begin{textblock}{160}(0,24)
\centering
  \begin{tabular}{|l|c|c|c|c|c|}
     \hline
         & {\small $\omega_0$} & {\small \  $\Delta$}  & {\small \, $\omega_1(b)$ } & {\small \  $\Delta$}  & {\small \,  $\omega_2(b)$ }  \\ \hline \hline
        A no-coop& $1$ & $1.5$ &  $1.5$ & $0.6$ & $\bm{0.9}$ \\ \hline
        B no-coop & $1$ & $0.6$ & $0.6$ & $1.5$ & $\bm{0.9}$ \\ \hline\hline
        A coop & $1$ & $1.5$ & $1.05$ & $0.6$ & $\bm{1.1}$ \\ \hline
        B coop & $1$ & $0.6$ & $1.05$ & $1.5$ & $\bm{1.1}$\\ \hline
\end{tabular}
\end{textblock}
}

\only<3-6>{
\begin{textblock}{160}(0,18)
\begin{align*}
\omega_{T+1}(N,b) & = \frac{1}{N} \overbrace{\bigg( n_c \, \omega_{T}(N,b) \, b \, Q_c  + n_s \, \omega_{T}(N,b) \, (1-b) \, Q_s   \bigg)}^{\hfrac{\text{\footnotesize \phantom{g}\En{Sum of all resources}\Es{Fondo común de todos los recursos}\phantom{g} }}{\text{\scriptsize (\En{with}\Es{con} $n_s$ \En{Heads}\Es{Caras} y $n_s$ \En{Tails}\Es{Secas})}}} \\
\uncover<4->{& =\omega_T(N,b) \, \underbrace{\left( \frac{n_c}{N}     \, b \,Q_c + \frac{m_s}{N}  \, (1-b)\, Q_s  \right)}_{\text{\scriptsize\phantom{Tg} $r(N,b)$: \En{growth rate}\Es{Tasa de crecimiento}  \phantom{Tg}}} \\[0.3cm]}
\end{align*}
\end{textblock}
}

\only<5-6>{
\begin{textblock}{160}(0,62)
\begin{align*}
\lim_{N \rightarrow \infty} r(N,b) & = p \, b \, Q_c + (1-p) \, (1-b) \, Q_s \\
\uncover<6>{& = 0.5 \cdot 0.5 \cdot 1.5 + 0.5 \cdot 0.5 \cdot 0.6 = 1.05}
\end{align*}
\end{textblock}
}

\only<7>{
\begin{textblock}{100}(30,20) \centering
\includegraphics[page=10,width=1\textwidth]{figuras/apuestasParalelas.pdf}
\end{textblock}
}

\only<7>{
\begin{textblock}{100}(130,22)
Cooperación
\end{textblock}
}
\only<7>{
\begin{textblock}{100}(130,68)
Individual
\end{textblock}
}

\end{frame}



\begin{frame}[plain]
\begin{textblock}{160}(0,4)
\centering \LARGE \only<1-4>{Tragedia de los comunes} \only<5->{Ventaja de la cooperación} \\
\large \only<1-4>{Dilema del prisionero}
\end{textblock}


\only<1-2>{
\begin{textblock}{160}(0,20) \centering
 \begin{equation*}
  \bordermatrix{ _{\text{\tiny Focal}}{\rotatebox{45}{\text{$\mid$}}}^{\text{\tiny \En{Other}\Es{Otro}}} \hspace{-0.5cm} & C & D \cr
      \ \ \   C & v-c & -c \cr
      \ \ \ D & v & 0 }
\end{equation*}
\end{textblock}

\begin{textblock}{80}(52,50)
$C$: Cooperar

$D$: Desertar
\end{textblock}
\begin{textblock}{80}(92,50)
$v$: Ventaja

$c$: Costo
\end{textblock}
}

 \only<2>{
\begin{textblock}{160}(0,70) \centering \Large
¿Y si dejamos de aportar al fondo común

y seguimos recibiendo la cuota común?
\end{textblock}
}

\only<3>{
\begin{textblock}{120}(20,14) \centering
\includegraphics[page=1,width=1\textwidth]{figuras/dilema.pdf}
\end{textblock}
}
\only<4>{
\begin{textblock}{120}(20,14) \centering
\includegraphics[page=2,width=1\textwidth]{figuras/dilema.pdf}
\end{textblock}
}
\only<5->{
\begin{textblock}{120}(20,14) \centering
\includegraphics[page=3,width=1\textwidth]{figuras/dilema.pdf}
\end{textblock}
}



\only<6->{
\begin{textblock}{160}(35,40)
\tikz{
  \node[invisible] (i1) {};
  \node[invisible, xshift=1cm, yshift=0.5cm] (i2) {};
  \plate {menor} {(i1)(i2)} {};
  \node[const,below=of menor] (nmenor) {$\hfrac{\text{Propiedad}}{\text{menor}}$};
}
\end{textblock}
}


\only<7>{
\begin{textblock}{160}(68,15)
\tikz{
  \node[invisible] (i1) {};
  \node[invisible, xshift=7cm, yshift=4cm] (i2) {};
  \plate {mayor} {(i1)(i2)} {};
  \node[const,below=of mayor] (nmayor) {\large \ \hspace{4cm} \ Propiedad Mayor};
}
\end{textblock}
}


\end{frame}


\begin{frame}[plain]
\begin{textblock}{160}(0,4)
\centering \LARGE Especialización cooperativa \\
\large \only<7->{Propiedad de especiación}
\end{textblock}

\only<1-6>{
\begin{textblock}{120}(20,10) \centering
\begin{equation*}
r(N,b) = \frac{n_c}{N}     \, b \,Q_c + \frac{m_s}{N}  \, (1-b)\, Q_s¨
\end{equation*}
\end{textblock}
}

\only<2>{
\begin{textblock}{120}(20,22) \centering \huge
\includegraphics[page=1,width=0.9\textwidth]{figuras/especializacion.pdf}
\end{textblock}
}
\only<3>{
\begin{textblock}{120}(20,22) \centering \huge
\includegraphics[page=2,width=0.9\textwidth]{figuras/especializacion.pdf}
\end{textblock}
}
\only<4>{
\begin{textblock}{120}(20,22) \centering \huge
\includegraphics[page=3,width=0.9\textwidth]{figuras/especializacion.pdf}
\end{textblock}
}
\only<5>{
\begin{textblock}{120}(20,22) \centering \huge
\includegraphics[page=4,width=0.9\textwidth]{figuras/especializacion.pdf}
\end{textblock}
}
\only<6>{
\begin{textblock}{120}(20,22) \centering \huge
\includegraphics[page=5,width=0.9\textwidth]{figuras/especializacion.pdf}
\end{textblock}
}


 \only<7->{
\begin{textblock}{160}(40,24)
\begin{flalign*}
\lim_{N \rightarrow \infty} & r(N,b)  = p \,  Q_c  \, b + (1-p) \, Q_s  \, (1-b) \\[0.5cm]
\only<8->{& \underset{b}{\text{ arg max }} \  0.5 \cdot 3.0  \cdot b + 0.5 \cdot 1.2  \cdot (1-b)}
&&
\end{flalign*}
\end{textblock}
}

\only<9>{
\begin{textblock}{160}(0,54) \centering \huge
\begin{equation*}
\underbrace{b^* = 1}_{\text{\normalsize Especialización total!}}
\end{equation*}
\end{textblock}
}


\end{frame}



\begin{frame}[plain]
\begin{textblock}{160}(0,4)
\centering \LARGE Heterogenidad cooperativa \\
\large \only<2>{Propiedad ecológica}
\end{textblock}

\only<1>{
\begin{textblock}{160}(0,20) \centering \Large
¿Si hay heterogenidad? \\[0.3cm] \large


Por ejemplo, si existen "Estaciones" en los hemisferios \small

(La moneda tiene una probabilidad en tiempos pares y otra en impares)
\end{textblock}
}

\only<2>{
\begin{textblock}{160}(0,35) \centering \Large
\includegraphics[width=0.45\textwidth]{figuras/pdf/diversificacionCooperativa}
\end{textblock}
}
\end{frame}



\begin{frame}[plain]
\begin{textblock}{160}(0,4)
\centering \LARGE Tecnología de reciprocidad \\
\large Teoría de la probabilidad
\end{textblock}
\vspace{1.2cm} \centering

\only<1>{
\begin{textblock}{160}(0,30) \Large
El problema del \textbf{valor justo} da inicio

a la teoría de la probabilidad \\[0.2cm]

\large (Pascal - Fermat 1648)
\end{textblock}
}

\only<2->{
\begin{textblock}{160}(0,22)
Probabilidad de hacer favores:

\begin{figure}[ht!]
    \centering
    \tikz{
    \node[latent, draw=white, yshift=0.7cm, minimum size=0.1cm] (b0) {};
    \node[latent,below=of b0,yshift=0.7cm, xshift=-1cm] (r1) {$S$};
    \node[latent,below=of b0,yshift=0.7cm, xshift=1cm] (r2) {$C$};

    \node[latent, below=of r1, draw=white, yshift=0.8cm, minimum size=0.1cm] (bc11) {};
    \node[accion, below=of r2, draw=white, yshift=0cm, color=blue!70] (bc12) {};
    \node[latent,below=of bc11,yshift=0.8cm, xshift=-0.5cm] (r1d2) {$S$};
    \node[latent,below=of bc11,yshift=0.8cm, xshift=0.5cm] (r1d3) {$C$};

    \node[accion,below=of r1d2,yshift=0cm, color=red!70] (br1d2) {};
    \node[accion,below=of r1d3,yshift=0cm, color=blue!70] (br1d3) {};
    \edge[-] {b0} {r1,r2};
    \edge[-] {r1} {bc11};
    \edge[-] {r2} {bc12};
    \edge[-] {bc11} {r1d2,r1d3};
    \edge[-] {r1d2} {br1d2};
    \edge[-] {r1d3} {br1d3};
    }
\end{figure}
\end{textblock}
}


\only<3>{
\begin{textblock}{160}(0,68)
El pago del favor debe ser inverso a la probabilidad
\end{textblock}
}


\only<4>{
\begin{textblock}{160}(0,66)
Garantiza \textbf{coexistencia} entre la casa de apuestas y la población cooperativa
\begin{equation*}
\cancel{p_{\text{rojo}}} \,  \cancel{Q^*_{\text{rojo}}} \, b + \bcancel{(1- p_{\text{rojo}})} \, \bcancel{Q^*_{\text{azul}}}  \, (1-b) = b + (1-b)  = 1
\end{equation*}
\end{textblock}
}

\end{frame}



% \begin{frame}[plain]
% \begin{textblock}{160}(0,4)
% \centering \LARGE Transición epistémica mayor \\
% \large
% \end{textblock}
% \vspace{1.2cm} \centering
%
%
% \end{frame}

\begin{frame}[plain,noframenumbering]
\centering \vspace{0.5cm}
\includegraphics[width=1\textwidth]{../../auxiliar/static/BP.png}
\end{frame}





%
% \begin{frame}[plain]
% \begin{textblock}{96}(0,6.5)\centering
% {\transparent{0.9}\includegraphics[width=0.8\textwidth]{../../auxiliar/static/inti.png}}
% \end{textblock}
%
% \begin{textblock}{160}(96,5.5)
% \includegraphics[width=0.35\textwidth]{../../auxiliar/static/pachacuteckoricancha}
% \end{textblock}
% \end{frame}





\end{document}



