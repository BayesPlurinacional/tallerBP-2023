\documentclass[shownotes,aspectratio=169]{beamer}


\usepackage{ragged2e} %\justifying
\usepackage{paracol}
\usepackage[utf8]{inputenc} %Para acentos en UTF8 (Prueba: á é í ó ú Á É Í Ó Ú ñ Ñ)
\usepackage{url}
%\usepackage{mathtools}
\usepackage{graphicx}
\usepackage{caption}
\usepackage{float} % para que los gr\'aficos se queden en su lugar con [H]
\usepackage[fleqn]{mathtools} % \coloneqq, flalign
\usepackage{subcaption}
\usepackage{wrapfig}
\usepackage{soul,color} %\st{Hellow world}
\usepackage{xcolor} %\st{Hellow world}
\usepackage[fleqn]{amsmath} %para escribir funci\'on partida
\usepackage{blkarray}
\usepackage{hyperref} % para inlcuir links dentro del texto
\usepackage{tabu} 
\usepackage{comment}
\usepackage{amsfonts} % mathbb{N} -> conjunto de los n\'umeros naturales  
\usepackage{enumerate}
\usepackage{listings}
\usepackage[shortlabels]{enumitem} %  shortlabels option to have compatibility with the enumerate-like scheme for label
\usepackage{framed}
\usepackage{mdframed}
\usepackage{multicol}
\usepackage{transparent} % \transparent{1.0}
\usepackage{bm} 
\usepackage[makeroom]{cancel} % \cancel{} \bcancel{} etc
\usepackage[absolute,overlay]{textpos} %no funciona
\setlength{\TPHorizModule}{1mm} %128mm  mitad: 64 
\setlength{\TPVertModule}{1mm}	%96mm  mitad 48

\newif\ifen
\newif\ifes
\newcommand{\en}[1]{\ifen#1\fi}
\newcommand{\es}[1]{\ifes#1\fi}
\estrue


\usepackage{todonotes}
\setbeameroption{show notes}
\usepackage{rotating}
\usepackage{transparent}


\newcommand{\E}{\en{S}\es{E}}
\newcommand{\A}{\en{E}\es{A}}
\newcommand{\Ee}{\en{s}\es{e}}
\newcommand{\Aa}{\en{e}\es{a}}

\hypersetup{
    colorlinks=true,
    linkcolor={red!50!black},
    citecolor={blue!35!black},
    urlcolor={blue!35!black}
}

\newcommand\hfrac[2]{\genfrac{}{}{0pt}{}{#1}{#2}} %\frac{}{} sin la linea del medio

\newcommand{\indep}{\perp \!\!\! \perp}
\newcommand{\N}{\mathcal{N}}
\newcommand{\vm}[1]{\mathbf{#1}}

\newtheorem{midef}{Definition}
\newtheorem{miteo}{Theorem}
\newtheorem{mipropo}{Proposition}

\usefonttheme[onlymath]{serif}


\usepackage{tikz} % Para graficar, por ejemplo bayes networks
%\usetikzlibrary{bayesnet} % Para que ande se necesita copiar el archivo  tikzlibrarybayesnet.code.tex en la misma carpeta

%%%%%%%%%%%%%%%%%%%%%%%%%%%%%%%%%5
%
% Incompatibles con textpos
%
%\usepackage{todonotes}
%\usepackage{tikz} % Para graficar, por ejemplo bayes networks
%
%%%%%%%%%%%%%%%%%%%%%%%%%%%%%%%%%%



\usepackage[absolute,overlay]{textpos} %no funciona
\setlength{\TPHorizModule}{1mm} %128mm  mitad: 64 
\setlength{\TPVertModule}{1mm}	%96mm  mitad 48
% 
% 
\captionsetup[figure]{labelformat=empty}

% 
% http://latexcolor.com/
\definecolor{lightseagreen}{rgb}{0.13, 0.7, 0.6.5}
\definecolor{greenblue}{rgb}{0.1, 0.55, 0.5}
\definecolor{redgreen}{rgb}{0.6, 0.4, 0.}
\definecolor{greenred}{rgb}{0.4, 0.7, 0.}
\definecolor{redblue}{rgb}{0.4, 0., .4}
\definecolor{tangelo}{rgb}{0.98, 0.3, 0.0}
\definecolor{git}{rgb}{0.94, 0.309, 0.2}
% 
\setbeamercolor{structure}{fg=greenblue}


%http://latexcolor.com/
\definecolor{azul}{rgb}{0.36, 0.54, 0.66}
\definecolor{rojo}{rgb}{0.7, 0.2, 0.116}
\definecolor{rojopiso}{rgb}{0.8, 0.25, 0.17}
\definecolor{verdeingles}{rgb}{0.12, 0.5, 0.17}
\definecolor{ubuntu}{rgb}{0.44, 0.16, 0.39}
\definecolor{debian}{rgb}{0.84, 0.04, 0.33}
\definecolor{dkgreen}{rgb}{0,0.6,0}
\definecolor{gray}{rgb}{0.5,0.5,0.5}
\definecolor{mauve}{rgb}{0.58,0,0.82}




\newcommand\Wider[2][3em]{%
\makebox[\linewidth][c]{%
  \begin{minipage}{\dimexpr\textwidth+#1\relax}
  \raggedright#2
  \end{minipage}%
  }%
}

\newenvironment{ejercicio}[1]{
% \setbeamercolor{block title}{bg=tangelo, fg=white}
\begin{exampleblock}{#1}
}{
\end{exampleblock}
}

\newenvironment{resumen}[1]{
\setbeamercolor{block title}{bg=git, fg=white}
\begin{block}{#1}
}{
\end{block}
}

\newenvironment{comando}{
\setbeamercolor{block body}{bg=git, fg=white}
\begin{block}{}
\begin{center}
\LARGE
\begin{texttt}
}{
\end{texttt}
\end{center}
\end{block}
}



% tikzlibrary.code.tex
%
% Copyright 2010-2011 by Laura Dietz
% Copyright 2012 by Jaakko Luttinen
%
% This file may be distributed and/or modified
%
% 1. under the LaTeX Project Public License and/or
% 2. under the GNU General Public License.
%
% See the files LICENSE_LPPL and LICENSE_GPL for more details.

% Load other libraries

%\newcommand{\vast}{\bBigg@{2.5}}
% newcommand{\Vast}{\bBigg@{14.5}}
% \usepackage{helvet}
% \renewcommand{\familydefault}{\sfdefault}

\usetikzlibrary{shapes}
\usetikzlibrary{fit}
\usetikzlibrary{chains}
\usetikzlibrary{arrows}

% Latent node
\tikzstyle{latent} = [circle,fill=white,draw=black,inner sep=1pt,
minimum size=20pt, font=\fontsize{10}{10}\selectfont, node distance=1]
% Observed node
\tikzstyle{obs} = [latent,fill=gray!25]
% Invisible node
\tikzstyle{invisible} = [latent,minimum size=0pt,color=white, opacity=0, node distance=0]
% Constant node
\tikzstyle{const} = [rectangle, inner sep=0pt, node distance=0.1]
%state
\tikzstyle{estado} = [latent,minimum size=8pt,node distance=0.4]
%action
\tikzstyle{accion} =[latent,circle,minimum size=5pt,fill=black,node distance=0.4]


% Factor node
\tikzstyle{factor} = [rectangle, fill=black,minimum size=10pt, draw=black, inner
sep=0pt, node distance=1]
% Deterministic node
\tikzstyle{det} = [latent, rectangle]

% Plate node
\tikzstyle{plate} = [draw, rectangle, rounded corners, fit=#1]
% Invisible wrapper node
\tikzstyle{wrap} = [inner sep=0pt, fit=#1]
% Gate
\tikzstyle{gate} = [draw, rectangle, dashed, fit=#1]

% Caption node
\tikzstyle{caption} = [font=\footnotesize, node distance=0] %
\tikzstyle{plate caption} = [caption, node distance=0, inner sep=0pt,
below left=5pt and 0pt of #1.south east] %
\tikzstyle{factor caption} = [caption] %
\tikzstyle{every label} += [caption] %

\tikzset{>={triangle 45}}

%\pgfdeclarelayer{b}
%\pgfdeclarelayer{f}
%\pgfsetlayers{b,main,f}

% \factoredge [options] {inputs} {factors} {outputs}
\newcommand{\factoredge}[4][]{ %
  % Connect all nodes #2 to all nodes #4 via all factors #3.
  \foreach \f in {#3} { %
    \foreach \x in {#2} { %
      \path (\x) edge[-,#1] (\f) ; %
      %\draw[-,#1] (\x) edge[-] (\f) ; %
    } ;
    \foreach \y in {#4} { %
      \path (\f) edge[->,#1] (\y) ; %
      %\draw[->,#1] (\f) -- (\y) ; %
    } ;
  } ;
}

% \edge [options] {inputs} {outputs}
\newcommand{\edge}[3][]{ %
  % Connect all nodes #2 to all nodes #3.
  \foreach \x in {#2} { %
    \foreach \y in {#3} { %
      \path (\x) edge [->,#1] (\y) ;%
      %\draw[->,#1] (\x) -- (\y) ;%
    } ;
  } ;
}

% \factor [options] {name} {caption} {inputs} {outputs}
\newcommand{\factor}[5][]{ %
  % Draw the factor node. Use alias to allow empty names.
  \node[factor, label={[name=#2-caption]#3}, name=#2, #1,
  alias=#2-alias] {} ; %
  % Connect all inputs to outputs via this factor
  \factoredge {#4} {#2-alias} {#5} ; %
}

% \plate [options] {name} {fitlist} {caption}
\newcommand{\plate}[4][]{ %
  \node[wrap=#3] (#2-wrap) {}; %
  \node[plate caption=#2-wrap] (#2-caption) {#4}; %
  \node[plate=(#2-wrap)(#2-caption), #1] (#2) {}; %
}

% \gate [options] {name} {fitlist} {inputs}
\newcommand{\gate}[4][]{ %
  \node[gate=#3, name=#2, #1, alias=#2-alias] {}; %
  \foreach \x in {#4} { %
    \draw [-*,thick] (\x) -- (#2-alias); %
  } ;%
}

% \vgate {name} {fitlist-left} {caption-left} {fitlist-right}
% {caption-right} {inputs}
\newcommand{\vgate}[6]{ %
  % Wrap the left and right parts
  \node[wrap=#2] (#1-left) {}; %
  \node[wrap=#4] (#1-right) {}; %
  % Draw the gate
  \node[gate=(#1-left)(#1-right)] (#1) {}; %
  % Add captions
  \node[caption, below left=of #1.north ] (#1-left-caption)
  {#3}; %
  \node[caption, below right=of #1.north ] (#1-right-caption)
  {#5}; %
  % Draw middle separation
  \draw [-, dashed] (#1.north) -- (#1.south); %
  % Draw inputs
  \foreach \x in {#6} { %
    \draw [-*,thick] (\x) -- (#1); %
  } ;%
}

% \hgate {name} {fitlist-top} {caption-top} {fitlist-bottom}
% {caption-bottom} {inputs}
\newcommand{\hgate}[6]{ %
  % Wrap the left and right parts
  \node[wrap=#2] (#1-top) {}; %
  \node[wrap=#4] (#1-bottom) {}; %
  % Draw the gate
  \node[gate=(#1-top)(#1-bottom)] (#1) {}; %
  % Add captions
  \node[caption, above right=of #1.west ] (#1-top-caption)
  {#3}; %
  \node[caption, below right=of #1.west ] (#1-bottom-caption)
  {#5}; %
  % Draw middle separation
  \draw [-, dashed] (#1.west) -- (#1.east); %
  % Draw inputs
  \foreach \x in {#6} { %
    \draw [-*,thick] (\x) -- (#1); %
  } ;%
}


 \mode<presentation>
 {
 %   \usetheme{Madrid}      % or try Darmstadt, Madrid, Warsaw, ...
 %   \usecolortheme{default} % or try albatross, beaver, crane, ...
 %   \usefonttheme{serif}  % or try serif, structurebold, ...
  \usetheme{Antibes}
  \setbeamertemplate{navigation symbols}{}
 }
\usetikzlibrary{decorations.text}
\usepackage{rotating}
\usepackage{transparent}

\usepackage{todonotes}
\setbeameroption{show notes}

\newcounter{capitulo}
\setcounter{capitulo}{1}
\newcommand{\unidad}{\thecapitulo \stepcounter{capitulo}}


\estrue

%\title[Bayes del Sur]{}

\begin{document}

\color{black!85}
\large

\begin{frame}[plain,noframenumbering]


\begin{textblock}{160}(0,0)
\includegraphics[width=1\textwidth]{../auxiliar/static/deforestacion}
\end{textblock}

\begin{textblock}{80}(18,9)
\textcolor{black!15}{\fontsize{44}{55}\selectfont Verdades}
\end{textblock}

\begin{textblock}{47}(85,70)
\centering \textcolor{black!15}{{\fontsize{52}{65}\selectfont Empíricas}}
\end{textblock}

\begin{textblock}{80}(100,28)
\LARGE  \textcolor{black!15}{\rotatebox[origin=tr]{-3}{\scalebox{9}{\scalebox{1}[-1]{$p$}}}}
\end{textblock}

\begin{textblock}{80}(66,43)
\LARGE  \textcolor{black!15}{\scalebox{6}{$=$}}
\end{textblock}

\begin{textblock}{80}(36,29)
\LARGE  \textcolor{black!15}{\scalebox{9}{$p$}}
\end{textblock}

\vspace{2cm}
\maketitle



\begin{textblock}{160}(01,81)
\footnotesize \textcolor{black!5}{\textbf{Talleres ``Verdades Empíricas'' \\
Congreso Bayesiano Plurinacional 2023} \\}
\end{textblock}

\end{frame}


\begin{frame}[plain,noframenumbering]

\begin{textblock}{160}(01,03)\centering
\textcolor{black!85}{{\large
\large Talleres \textbf{Verdades empíricas} \\[-0.1cm] \footnotesize Congreso Bayesiano Plurinacional 2023}} \\[0.3cm] \scriptsize Repositorio: \texttt{https://github.com/BayesPlurinacional/tallerBP-2023}
\end{textblock}



\begin{textblock}{140}(10,19)

\vspace{0.8cm}


\normalsize Taller 1 \textbf{Causalidad} \\[0.1cm] \footnotesize
\ \ $1$. Modelos gráficos e inferencia \\
\ \ $2$. Inferencia Causal\\

 \vspace{0.5cm}

\normalsize Taller 2 \textbf{Datos temporales} \\[0.1cm] \footnotesize
\ \ $3$. Sorpresa: el problema de la comunicación con la realidad \\
\ \ $4$. Modelos de historia completa\\

\vspace{0.5cm}

\normalsize Taller 3 \textbf{Toma de decisiones} \\[0.1cm] \footnotesize
\ \ $5$. La función de costo epistémico-evolutiva \\
\ \ $6$. Competencia ``Apuestas de vida'' \\

\end{textblock}

\end{frame}


\begin{frame}[plain,noframenumbering]

\begin{textblock}{160}(00,04)\centering
\textcolor{black!85}{\Large Objetivos}
\end{textblock}

\begin{textblock}{140}(10,14)

\small

\parbox{14cm}{

Las reglas de la probabilidad se conocen desde finales del siglo 18 y desde entonces se las ha adoptado como sistema de razonamiento en todas las ciencias empíricas (ciencia con datos).
%
Sin embargo, su aplicación estricta (enfoque bayesiano) ha estado históricamente limitada debido al costo computacional asociado a la evaluación de todo el espacio de hipótesis.

\vspace{0.3cm}

En las últimas décadas se han desarrollado una gran cantidad de algoritmo de aprendizaje automático.
Bajo este marco, cada nuevo problema se ``resuelve'' con alguno de los algoritmos ya existentes.
Si bien este flujo de trabajo ha sido exitoso para muchas tareas, tiene la desventaja de ser inflexible y de dificultar la inferencia causal.

\vspace{0.3cm}

A lo largo de esas mismas décadas se fueron desarrollando técnicas que permiten crear modelos a medida del problema, de forma sencilla e intuitiva.
Con ellas podemos: expresar de forma gráfica las relaciones causales entre las variables; descomponer las reglas de la probabilidad como mensajes entre los nodos de la red causal; y delegar la inferencia a los lenguajes de programación probabilística.
Estos métodos permiten hoy computar la incertidumbre óptima dada la información disponible en todos los campos de la ciencia.

}
\end{textblock}
\end{frame}


\begin{frame}[plain,noframenumbering]

\centering \LARGE
Taller 1. Causalidad.

\end{frame}



\begin{frame}[plain,noframenumbering]
\begin{textblock}{160}(0,43)
\includegraphics[width=1\textwidth]{../auxiliar/static/modelosGraficos}
\end{textblock}


\begin{textblock}{160}(4,4)
\LARGE \textcolor{black!85}{\fontsize{22}{0}\selectfont \textbf{Modelos gráficos e inferencia}}
\end{textblock}
% \begin{textblock}{160}(4,12)
% \LARGE \textcolor{black!85}{\fontsize{22}{0}\selectfont \textbf{algoritmos de inferencia}}
% \end{textblock}


\begin{textblock}{55}[0,0](72,23)
\begin{turn}{0}
\parbox{10cm}{\sloppy\setlength\parfillskip{0pt}
\textcolor{black!85}{Unidad \unidad} \\
\small\textcolor{black!85}{Acuerdos intersubjetivos en contextos de incertidumbre.} \\
\small\textcolor{black!85}{Especificación gráfica de modelos causales. Evaluación} \\
\small\textcolor{black!85}{de modelos causales. La emergencia del sobreajuste y el} \\
\small\textcolor{black!85}{balance natural de las reglas de la probabilidad.} \\
}
\end{turn}
\end{textblock}

\end{frame}


\begin{frame}[plain,noframenumbering]

\begin{textblock}{160}(0,0)
\includegraphics[width=1\textwidth]{../auxiliar/static/peligro_predador}
\end{textblock}

\begin{textblock}{160}(127,67)
\LARGE \textcolor{black!5}{\fontsize{22}{0}\selectfont \textbf{Inferencia  \\[-0.1cm] \hspace{0.5cm} causal}}
\end{textblock}

\begin{textblock}{55}(2,3)
\begin{turn}{0}
\parbox{15cm}{\small
\textcolor{black!95}{Flujos de inferencia en modelos causales. Efecto de}\\
\textcolor{black!95}{las intervenciones en modelos causales. Conclusiones} \\
\textcolor{black!95}{causales a partir de datos observables. Identificación} \\
\textcolor{black!95}{de modelo causal.} \\
\normalsize\textcolor{black!95}{Unidad \unidad} \\
}
\end{turn}
\end{textblock}

\end{frame}



\begin{frame}[plain,noframenumbering]

\centering \LARGE
Taller 1. Datos temporales.

\end{frame}




\begin{frame}[plain,noframenumbering]

\begin{textblock}{160}(0,0)
\includegraphics[width=1\textwidth]{../auxiliar/static/fuego}
\end{textblock}

\begin{textblock}{160}(4,26)
\LARGE \textcolor{black!5}{\fontsize{22}{0}\selectfont \textbf{Sorpresa: el problema}}
\end{textblock}
\begin{textblock}{160}(4,34)
\LARGE \textcolor{black!5}{\fontsize{22}{0}\selectfont \textbf{de la comunicación}}
\end{textblock}
\begin{textblock}{160}(4,42)
\LARGE \textcolor{black!5}{\fontsize{22}{0}\selectfont \textbf{con la realidad}}
\end{textblock}


\begin{textblock}{55}[0,0](88,25)
\begin{turn}{0}
\parbox{7cm}{\sloppy\setlength\parfillskip{0pt}
\textcolor{black!0}{Unidad \unidad} \\
\small\textcolor{black!5}{\hspace{-0.3cm}La estructura invariante del dato empírico.} \\
\small\textcolor{black!5}{\hspace{-0.3cm}Construcción de un sistema de información.}\\
\small\textcolor{black!5}{\hspace{-0.4cm}Algoritmo de inferencia por pasaje de mensajes.} \\
\small\textcolor{black!5}{\hspace{-0.6cm}Ejemplos: modelos de estimación de habilidad.} \\
}
\end{turn}
\end{textblock}

\end{frame}



\begin{frame}[plain,noframenumbering]
\begin{textblock}{160}(0,-4.3) \centering
\includegraphics[width=1\textwidth]{../auxiliar/static/antartic}
\end{textblock}

\begin{textblock}{160}(0,0) \centering
\tikz{
\node[det, fill=black,draw=black] (k) {\textcolor{black}{--------------------------------------------------------------------------------------------------------------------------------------}} ;
}
\end{textblock}

\begin{textblock}{160}(5,0)
\tikz{
\node[det, fill=black,draw=black,text width=0.01cm] (k) {\textcolor{black}{--------------------------------------------------------------------------------------------------------------------------------------}} ;
}
\end{textblock}


\begin{textblock}{160}(0,4) \centering
\LARGE \hspace{1cm} \textcolor{black!20}{\fontsize{22}{0}\selectfont \textbf{Modelos de historia \\ \hspace{1cm} completa}}
\end{textblock}


\begin{textblock}{55}[0,1](8,70)
\begin{turn}{90}
\parbox{6cm}{\footnotesize
\textcolor{black!10}{Millones de km$^2$ de hielo Antártico}}
\end{turn}
\end{textblock}


\begin{textblock}{160}(20,63)
\textcolor{black!10}{Unidad \unidad \\ \small
Redes bayesianas de historia completa. \\
El problema de usar el posterior como prior del siguiente evento\\
El algoritmo de inferencia por loopy belief propagation. \\
Consideraciones de inferencia causal en series temporales. \\
}
\end{textblock}


\end{frame}


\begin{frame}[plain,noframenumbering]

\centering \LARGE
Taller 3. Toma de decisiones.

\end{frame}


\begin{frame}[plain,noframenumbering]

\begin{textblock}{160}(0,-15)
\includegraphics[width=1\textwidth]{../auxiliar/static/tsimane}
\end{textblock}


% VERSION 2
\begin{textblock}{160}(6,36)
\LARGE \rotatebox[origin=tr]{18}{\textcolor{black!95}{\fontsize{22}{0}\selectfont \textbf{La función}}}
\end{textblock}
\begin{textblock}{160}(41,32)
\LARGE \rotatebox[origin=tr]{23}{\textcolor{black!95}{\fontsize{22}{0}\selectfont \textbf{de}}}
\end{textblock}
\begin{textblock}{160}(50.5,23)
\LARGE \rotatebox[origin=tr]{28}{\textcolor{black!95}{\fontsize{22}{0}\selectfont \textbf{costo}}}
\end{textblock}
\begin{textblock}{160}(68,5.3)
\LARGE \rotatebox[origin=tr]{26}{\textcolor{black!95}{\fontsize{22}{0}\selectfont \textbf{epistémico}}}
\end{textblock}
\begin{textblock}{160}(104,5.5)
\LARGE \rotatebox[origin=tr]{8}{\textcolor{black!95}{\fontsize{22}{0}\selectfont \textbf{-}}}
\end{textblock}
\begin{textblock}{160}(110,3)
\LARGE \rotatebox[origin=tr]{-14}{\textcolor{black!95}{\fontsize{22}{0}\selectfont \textbf{evolutiva}}}
\end{textblock}


\begin{textblock}{55}[0,0](119,22)
\begin{turn}{-57}
\parbox{7cm}{\sloppy\setlength\parfillskip{0pt}
\textcolor{black!0}{\ \ \ \ \ Unidad \unidad} \\
\small\textcolor{black!5}{\hspace{-0.15cm} Apuestas óptimas.} \\
\small\textcolor{black!5}{\hspace{-0.85cm} Ventajas a favor de la:} \\
\small\textcolor{black!5}{\hspace{-1.45cm} Diversificación (propiedad epistémica)}\\
\small\textcolor{black!5}{\hspace{-1.7cm} Cooperación (propiedad evolutiva)}\\
\small\textcolor{black!5}{ \hspace{-1.75cm}Especialización (propiedad de especiación)} \\
\small\textcolor{black!5}{\hspace{-2cm} Heterogeniedad (propiedad ecológica).\\ }}
\end{turn}
\end{textblock}


\end{frame}


\begin{frame}[plain,noframenumbering]

\begin{textblock}{160}(0,11)  \centering
\includegraphics[width=0.40\textwidth]{../auxiliar/static/treeOfLife-liviano}
\end{textblock}

\begin{textblock}{160}(0,3) \centering
\LARGE \textcolor{black!85}{\rotatebox[origin=tr]{0}{\fontsize{22}{0}\selectfont \textbf{Apuestas de vida}}}
\end{textblock}


\begin{textblock}{55}(75,39)
\textcolor{black!85}{\normalsize El árbol de la vida \\
\fontsize{2}{0}\selectfont Synthesis of phylogeny and taxonomy into a comprehensive tree of life \\}
\end{textblock}


\begin{textblock}{55}(3,81)
\textcolor{black!85}{Unidad \unidad}
\end{textblock}

\begin{textblock}{55}(25,81.3)
\begin{turn}{0}
\parbox{15cm}{\small \textcolor{black!85}{Presentación de una competencia de inferecia con apuestas e intercambio de recursos.}
}
\end{turn}
\end{textblock}

\end{frame}

\end{document}
